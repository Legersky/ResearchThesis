\documentclass[11pt]{beamer}
% 
\usepackage[czech]{babel}
\usepackage[utf8]{inputenc}
\usepackage[IL2]{fontenc}
\usepackage{amsmath, amsthm}
\usepackage{mathtools, mathdots}
\usepackage{color}
\usepackage{graphicx}
\usepackage{wrapfig}
\usepackage{pgffor}

% \usepackage{appendixnumberbeamer}

% \usepackage{graphicx}
% \usepackage{wrapfig}
% \usepackage{eurosym}
% \usepackage{hyperref}

\newcommand{\Zomega}{\mathbb{Z}[\omega]}

\newcommand{\ZZ}{\mathbb{Z}}
\newcommand{\CC}{\mathbb{C}}
\newcommand{\NN}{\mathbb{N}}
\newcommand{\RR}{\mathbb{R}}

\newcommand{\OO}{\mathbb{O}}
\newcommand{\II}{\mathbb{I}}

\newcommand{\A}{\mathcal{A}}
\newcommand{\B}{\mathcal{B}}
\newcommand{\Q}{\mathcal{Q}}

\newcommand{\fin}[1]{\text{Fin}_{#1}(\beta)}

\newcommand{\multMat}[1]{\sum_{i=0}^{d-1} {#1}_i S^i}

\newcommand{\vect}[1]{\begin{pmatrix}
            {#1}_0 \\
            {#1}_1 \\
            \vdots \\
            {#1}_{d-1} 
            \end{pmatrix}}

% \newenvironment{method}[2]{\rule{0mm}{0mm} \newline\textbf{#1}(\textit{#2}) \newline}{}


\def\changemargin#1#2{\list{}{\rightmargin#2\leftmargin#1}\item[]}
\let\endchangemargin=\endlist 

\newenvironment{method}[2]{
% \rule{0pt}{0pt} \newline
\noindent \textbf{#1(}\textit{#2}\textbf{)}
\vspace{-5pt}
\begin{changemargin}{3em}{0em}}
{\end{changemargin}}

\newcommand{\var}[1]{\textit{#1}}
\newcommand{\fun}[2]{\textbf{#1}(\var{#2})}


\usepackage{mathtools, mathdots}

\usepackage{algorithm}
\usepackage{algorithmic}
\renewcommand{\algorithmicrequire}{\textbf{Input:}}
\algsetup{indent=2em}


\usepackage{enumerate}


\mode<presentation>
{
  \usetheme{Copenhagen} %{}
  \useinnertheme{circles}
  \usecolortheme{crane}
}

\title{Konstrukce algoritmů pro paralelní sčítání}
\institute{TIGR \\
                \url{jan.legersky@gmail.com} \\
					\rule{0cm}{0mm} \\
					\rule{0cm}{0mm} \\
% 					Theoretical Informatics GRoup \\
%           	\rule{0cm}{0mm} \\
					Obhajoba výzkumného úkolu}
\author{Jan Legersk\'y}
\date{10. září 2015}

\setbeamertemplate{navigation symbols}{}

\setbeamertemplate{footline}{ \hfill \mbox{\insertframenumber\ /\ \inserttotalframenumber } \hspace{\rightmargin} }


\begin{document}

\begin{frame}
  \titlepage
\end{frame}

\begin{frame}
  \tableofcontents
\end{frame}



\begin{frame}
  \frametitle{Poziční soustava}
  Algebraické celé číslo $\omega$ stupně $d$.
  $$\Zomega= \left\{\sum_{j=0}^{n} a_j \omega^j\ \colon n\in\NN,  a_j \in \ZZ \right\}$$
  \pause
  Poziční soustava je dána:
  \begin{itemize}
    \item Báze $\beta \in \Zomega$, $|\beta|>1$.
    \item Abeceda $\A \subset \Zomega , 0\in \A$. 
  \end{itemize}
  
  Komplexní číslo $x$ má konečnou  $(\beta, \A)$-reprezentaci, pokud~$x=\sum_{j=-m}^n x_j \beta^j$ s koeficienty $x_j\in\A$.
  $$
    (x)_{\beta,\A}=x_n x_{n-1}\cdots x_1 x_0 \bullet x_{-1} x_{-2} \cdots x_{-m}
  $$ 
\end{frame}

\section{Paralelní sčítání}
% \subsection{Standard numeration systems}
\begin{frame}
  \frametitle{Sčítání}
    \begin{align*}
  (x)_{\beta,\A}=\;x_{n'} \;x_{{n'}-1}\cdots x_1 \;x_0 &\bullet x_{-1} \;x_{-2}\, \cdots x_{-m'} \\[-3pt]
  (y)_{\beta,\A}=\,y_{n'} \;y_{{n'}-1}\cdots y_1 \,\;y_0 &\bullet y_{-1} \;y_{-2} \;\cdots y_{-m'} \\[-7pt]
    \line(1,0){130} & \line(1,0){100} \\[-7pt]
  (w)_{\beta,\A+\A}=w_{n'} w_{{n'}-1}\cdots w_1 w_0 &\bullet w_{-1} w_{-2} \cdots w_{-m'}\,,
  \end{align*}
  kde
  $$
    w_j=x_j+y_j \in \A +\A\,.
  $$
  \pause
  Chceme najít $(\beta,\A)$-reprezentaci součtu
%   $w$, tzn. číslice $$z_{n'}, z_{n'-1},\dots, z_1, z_0, z_{-1}, z_{-2}, \dots, z_{-m'}\in\A$$ takovou, že
  $$
    z_{n} z_{n-1}\cdots z_1 z_0 \bullet z_{-1} z_{-2} \cdots z_{-m}=(w)_{\beta,\A}\,.
  $$ 
\end{frame}



\begin{frame}
    $$0=1\cdot \beta^j -\beta \cdot \beta^{j-1}$$%=1 (-\!\beta) 0 \cdots 0\bullet $$ 
    
    $\rightarrow$ přičteme $q_j\cdot 0$ pro každé $j$:
    
    $$0=q_j\cdot \beta^j -\beta  q_j \cdot \beta^{j-1}$$
    \begin{align*}
        w_{n'} w_{n'-1}&&&\cdots& &w_{j+1}&\!\! &\textcolor{red}{w_j}  & \!\!  &w_{j-1} &&\cdots &&w_1 w_0\bullet \hspace{200pt}\\[-5pt]
                   &&&&       &       & &     &   &q_{j-2} &&\iddots\\[-3pt] 
                   &&&&       &       & &\textcolor{red}{q_{j-1}}& -&\beta q_{j-1} \\[-3pt]
                   &&&&         &q_j&   \textcolor{red}{-}&\textcolor{red}{\beta q_j} &&\\[-8pt]
                   &&&  \iddots      &   -&\beta q_{j+1}&   &\ &&\\[-17pt]
    \intertext{\line(1,0){280}}
    \vspace{-15pt}
    \\[-30pt]
     z_{n} \cdots z_{n'} z_{n-1}&&&\cdots& &z_{j+1}& &\textcolor{red}{z_j}& &z_{j-1} &&\cdots &&z_1 \; z_0\bullet                            
    \end{align*}
    \pause
    Jak volit váhové koeficienty $q_j$ tak aby
    $$
        \textcolor{red}{z_j=w_j + q_{j-1} - q_j \beta} \in \A\,?
    $$
        
%     \pause
%   Weight coefficient $q_j$ is chosen such that it returns the sum of the digit $w_j$  and the right carry $q_{j-1}$  back to the alphabet~$\A$.
\end{frame}



\begin{frame}
    $$
        \textcolor{red}{z_j=w_j + q_{j-1} - q_j \beta}\,
    $$
  Standardní sčítání:% v soustavě $(\beta, \A)$:
%   $$
%     \beta \in \NN\,,\beta  \geq 2\,, \A = \{0, 1, 2,\dots, \beta -1 \}
%   $$ 
  \begin{align*}
    w_n w_{n-1}\cdots w_{j+1}& \textcolor{red}{w_j w_{j-1} \cdots w_1 w_0}\bullet& \,,w_i &\in \A+\A\,,    \\
    \longrightarrow z_{n+1}\; z_{n}\; z_{n-1}\;\cdots z_{j+1} &\textcolor{red}{z_j} \; z_{j-1}\; \cdots \;z_1 \; z_0\bullet& \,,z_i &\in \A\,.
  \end{align*}
    
  \pause 
   Paralelní sčítání (Avizienis, 1961):
  \begin{align*}
    \cdots w_{j+t+1}\only<2>{\textcolor{red}{w_{j+t} \cdots w_{j+1}}} \only<3->{w_{j+t} \cdots w_{j+1}}& \textcolor{red}{w_j w_{j-1} \cdots w_{j-r}}w_{j-r-1} \cdots &,\, w_i &\in \A + \A\,,    \\
    \longrightarrow \cdots z_{j+t+1}\;z_{j+t} \; \cdots \; z_{j+1} &\textcolor{red}{z_j} \; z_{j-1}\; \cdots z_{j-r}\;z_{j-r-1}\; \cdots &,\, z_i &\in \A\,.
  \end{align*}

%   \pause
%   Například:
%   $$
%   \beta \in \NN, \beta \geq 3, \A=\{-a, \dots, 0, \dots a\}, b/2 <a \leq b-1\,. 
%   $$
    
\end{frame}

\begin{frame}
Jak najít váhové koeficienty $q_j$ takové, že 
    $$
        z_j=\underbrace{w_j}_{\in \A +\A} + q_{j-1} - q_j \beta \in \A 
    $$
    pro všechny vstupy $(w)_{\beta,\A+\A}$ a každou pozici $j$,
    tzn. algoritmus pro paralelní sčítání?
    
    \rule{0cm}{0cm}
    
     \url{https://cloud.sagemath.com/projects}
\end{frame}


\section{Extending window method}
\begin{frame}
    \frametitle{Extending window method}
    
    Hledáme šířku okna $M \in \NN$ a váhovou funkci $q:(\A+\A)^{M} \rightarrow \Q \subset \Zomega$ takovou, že $q_j=q(w_j, \dots, w_{j-M+1})$.
        
    \pause
    \vspace{20pt}
    Metoda:
    \begin{enumerate}
        \item Najdeme množinu váhových koeficientů $\Q \subset \Zomega$.
        \item Zvětšujeme šířku okna $M$ a pro všechny $(w_j,w_{j-1}, \dots , w_{j-M+1}) \in (\A+\A)^{M}$ zkoušíme najít váhový koeficient z množiny $\Q$ pro definování váhové funkce $q$.
    \end{enumerate}
\end{frame}

\subsection{Fáze 1 -- množina váhových koeficientů}
\begin{frame}
    \frametitle{Fáze 1 -- hledání množiny váhových koeficientů}
    Hledáme množinu váhových koeficientů $\Q \subset \Zomega$ takovou, že
    $$
    \underbrace{(\A+\A)}_{\uncover<2->{w_j \in}}+ \underbrace{\Q}_{\uncover<2->{q_{j-1} \in}} \subset \underbrace{\A}_{\uncover<2->{z_j \in}} + \underbrace{\beta \Q}_{\uncover<2->{\beta q_j \in}}
    $$
    \pause
    Odtud, pro všechny $q_{j-1} \in \Q$ a $w_j \in \A+\A$ existuje $q_j \in \Q$ takové, že
    $$
    z_j=w_j + q_{j-1} - q_j \beta \in \A \,.
    $$
\end{frame}


\begin{frame}
    Konstruujeme $\Q$ iterativně:
    \begin{block}{Fáze 1}
     $k:=0$
     
    $\Q_0:=\{0\}$
      
      \pause
      Repeat:
      \begin{itemize}
          \item rozšiř $\Q_k$ na $\textcolor{red}{\Q_{k+1}}$ tak, že
           $$
              (\A+\A)+ \Q_k \subset \A + \beta \textcolor{red}{\Q_{k+1}}\,,
           $$
           \item $k:=k+1$
      \end{itemize}
      \pause
      until $\Q_k = \Q_{k+1}$.
      
      \pause
      \vspace{7pt}
      $\Q:=\Q_k$
    \end{block}
\end{frame}

\begin{frame}
    \frametitle{Příklad -- fáze 1}
    \begin{block}{Eisensteinova báze}
        \begin{itemize}
            \item Báze $\beta = \omega - 1 $, kde $\omega=\exp(\frac{2 \pi i}{3}), \omega^2+\omega+1=0$.
            \item Minimální polynom báze je $ \beta^{2} + 3\beta + 3 $.
            \item Abeceda $ \A=\left\{0, 1, -1, \omega, -\omega, -\omega - 1, \omega + 1\right\} \subset \Zomega$.
        \end{itemize}
    \end{block}
\end{frame}

\begin{frame}
\foreach \n in {1,...,13} {%
      \only<\n>{%
            \includegraphics[height=0.8\textheight]{img/eisenstein/phase1_image_\n} \hfill
            \vfill
          }  
    }
\end{frame}

\subsection{Fáze 2 -- váhová funkce}
\begin{frame}
    \frametitle{Fáze 2 -- hledání váhové funkce}

    Hledáme šířku okna $M$ a váhovou funkci $q:(\A+\A)^{M} \to \Q$.
    \pause
    
    Předpokládejme, že šířka okna je $m$.
    
    Zkontrolujeme všechny přenosy zprava $q_{j-1}$ a určíme $q_j\in\Q$ takové, že 
    $$
    z_j=w_j + q_{j-1} - q_j \beta \in \A \,.
    $$
    Množinu všech možných hodnot $q_j$ označíme $\Q_{[w_{j},\dots, w_{j-m+1}]}\subset \Q$.
        
    \pause
    Šířka okna $M$ a váhová funkce $q$ je nalezena když 
    $$
    \#\Q_{[w_{j},\dots, w_{j-M+1}]}=1
    $$
    pro všechny  $w_{j},\dots, w_{j-M+1} \in (\A+\A)^M$.
\end{frame}    


\begin{frame}
    \begin{block}{Fáze 2}
      $m:=1$
      
      Pro každé $w_j \in \A+\A$ najdi množinu $\Q_{[w_j]} \subset \Q$ takovou, že
      $$
      w_j + \Q \subset \A + \beta \Q_{[w_j]}
      $$
    \pause
      While $(\max\{\#\Q_{[w_j,\dots, w_{j-m+1}]}:(w_j,\dots, w_{j-m+1}) \in (\A+\A)^m \} > 1)$ do:
      \begin{itemize}
          \pause
          \item $m:= m +1$
          \pause
          \item Pro všechny $(w_j,\dots, w_{j-m+1}) \in (\A+\A)^{m}$ najdi množinu $\textcolor{red}{\Q_{[w_j,\dots, w_{j-m+1}]}} \subset \Q_{[w_j,\dots, w_{j-m+2}]}$ takovou, že
          $$
          w_j + \Q_{[w_{j-1},\dots, w_{j-m+1}]} \subset \A + \beta \textcolor{red}{\Q_{[w_j,\dots, w_{j-m+1}]}}\,,
          $$
      \end{itemize}
      \pause
      $M:= m$ 
      
      \pause
      $q(w_j,\dots, w_{j-M+1}):=$ jediný prvek $\Q_{[w_j,\dots, w_{j-M+1}]}$
    \end{block}
\end{frame}


\begin{frame}
    \frametitle{Příklad -- fáze 2}
    Předpokládejme vstup  $(\omega\, 1\, 2)$.
\end{frame}

\begin{frame}
% Vstup: $(\omega\, 1\, 2)$
\foreach \n in {1,...,7} {%
      \only<\n>{%
        \includegraphics[height=0.8\textheight]{img/eisenstein/phase2_image_\n} \hfill
        \vfill
          }  
    }
\end{frame}

\section{Konvergence}
\subsection{Abeceda}
\begin{frame}
    \frametitle{Nutná podmínka na abecedu}
    Pro existenci algoritmu pro paralelní sčítání je nezbytné, aby abeceda $\A$ obsahovala:
    \begin{itemize}
        \item všechny reprezentanty  modulo $\beta$,
        \item všechny reprezentanty  modulo $\beta-1$.
    \end{itemize}
\end{frame}

\subsection{Fáze 1}
\begin{frame}
    \frametitle{Fáze 1 -- postačující podmínka konvergence}
    Pokud je algebraické celé číslo $\omega$ stupně 1 nebo je komplexní stupně 2, fáze 1 konverguje.
\end{frame}

\subsection{Fáze 2}
\begin{frame}
    \frametitle{Fáze 2 -- nutná podmínka konvergence}
    Pokud algoritmus pro paralelní sčítání existuje, fáze 2 konverguje pro vstupy $(b,\dots, b)$ pro všechny $b\in \A+\A$.
    \rule{0cm}{0cm}
    
    Máme algoritmus, které určí, jestli je fáze 2 konečná pro vstupy tohoto tvaru.
\end{frame}

\section{Výsledky}
\begin{frame}
\fontsize{8pt}{10}\selectfont
    \frametitle{Testované příklady}
    \begin{tabular}{l|c cc c c} 
      Jméno &   Abec. & Post. p. & Fáze 1 & Nut. p. & Fáze 2 \\ \hline
      Eisenstein\_1-block\_complex &   yes & yes & \checkmark & \checkmark & \checkmark \\
      Eisenstein\_1-block\_integer &   yes & yes & \checkmark & \xmark & --\\
      Eisenstein\_1-block\_small\_complex &  no & -- & -- & -- & -- \\
      Eisenstein\_2-block &  no & -- & -- & -- & -- \\
      Eisenstein\_2-block\_4elements &   yes & yes & \checkmark & \xmark & --\\
      \hline
      Penney\_1-block\_complex &   yes & yes & \checkmark & \checkmark & \checkmark \\
      Penney\_1-block\_small\_complex &  no & -- & -- & -- & -- \\
      Penney\_1-block\_integer &   yes & yes & \checkmark & \xmark & --\\
      Penney\_2-block\_integer &   yes & yes & \checkmark & \checkmark & \checkmark \\
      \hline
      Quadratic+1-2+2\_1-block\_complex &   yes & yes& \checkmark & \checkmark & \checkmark \\
        Quadratic+1-2+2\_1-block\_integer &   yes & yes & \checkmark & \xmark & --\\
      \hline
      Quadratic+1+4+5\_1-block\_complex &   yes & yes & \checkmark & \checkmark & \checkmark \\ 
      \hline
      Quadratic+1+3+5\_1-block\_complex &   yes & yes & \checkmark & \checkmark & \xmark \\
      \hline
      Quadratic+1-5+3\_1-block\_integer &   yes & no & \xmark & -- & --\\
      \hline
      Quadratic+1-5+5\_1-block\_real &   yes & no & \checkmark & \xmark & --\\
      \hline
      base\_2 &   yes & yes & \checkmark & \checkmark & \checkmark \\
        base\_4 &   yes & yes & \checkmark & \checkmark & \checkmark \\
      \hline
      Cubic+1+1-1+1\_complex &   yes & no & \xmark & -- & --\\
        Cubic+1+1-5+5\_complex &   yes & no & \checkmark & \xmark & --\\
  \end{tabular}

\end{frame}

\begin{frame}
    \frametitle{Výsledky}    
    \begin{itemize}
        \item Implementace v SageMath: \url{https://cloud.sagemath.com/projects}
        \item Vstupní kontrola abecedy.
        \item Algoritmus pro kontrolu nutné podmínky konvergence fáze 2.
        \item Zkoušení různých modifikací výběru prvků ve fázi 2.
        \item Testování příkladů.
    \end{itemize}
\end{frame}


\begin{frame}
\fontsize{14pt}{10}\selectfont
\begin{center}
Děkuji

\end{center}    

\end{frame}
\end {document}


