
\begin{defn}
  Let $\beta \in \CC, |\beta|>1$ and $\A \subset \CC$ be a finite set containing 0. A pair $(\beta, \A)$ is called a \emph{positional numeration system} with \emph{base} $\beta$ and \emph{digit set} $\A$, usually called \emph{alphabet}.
\end{defn}
DECREASING INDICES

\begin{defn}
  We say that a complex number $x$ has a \emph{$(\beta, \A)$-representation} if~ there exist digits $x_n,x_{n-1}, x_{n-2},\dots \in\A, n\geq 0$ such that $x=\sum_{j=-\infty}^n x_j \beta^j$.
\end{defn}
We associate a number $x$ with representation as a bi-infinite string 
  $$
    (x)_{\beta,\A}=0^\omega x_n x_{n-1}\cdots x_1 x_0 \bullet x_{-1} x_{-2} \cdots\,,
  $$
  where $0^\omega$ denotes the infinite sequence of zeros.

\begin{defn}
Let $(\beta, \A)$ be a positional numeration system. The set of all complex numbers with a finite $(\beta, \A)$-representation is defined by
$$
    \fin{\A}:=\left\{\sum_{j=-m}^n x_j \beta^j : n, m \in \NN, x_j \in \A \right\}\,.
$$
\end{defn}
A representation of $x\in\fin{\A}$ is 
$$
(x)_{\beta,\A}= 0^\omega x_n x_{n-1}\cdots x_1 x_0 \bullet x_{-1} x_{-2} \cdots x_{-m} 0^\omega\,.
$$ 
We omit starting and ending $0^\omega$ when we work with numbers in $\fin{\A}$ in the following.
\begin{defn}
Let $\A$ and $\B$ be alphabets. A function $\varphi:\B^\ZZ \rightarrow \A^\ZZ$ is said to be \emph{$p$-local} if there exist $r,t\in\NN_0$ satisfying $p=r+t+1$ and a function $\phi: \B^p \rightarrow \A$ such that, for any $w=(w_j)_{j\in\ZZ}\in\B^\ZZ$ and its image $z=\varphi(w)=(z_j)_{j\in\ZZ}\in\A^\ZZ$, we have $z_j=\phi(w_{j+t},\cdots,w_{j-r})$ for every $j\in\ZZ$. The parameter $t$, resp. $r$, is called \emph{anticipation}, resp. \emph{memory}.
\end{defn}
  
\begin{defn}
Let $\beta$ be a base and $\A$ and $\B$ two alphabets containing 0. A function $\varphi:\B^\ZZ\rightarrow \A^\ZZ$ such that
  \begin{enumerate}
      \item for any $w=(w_j)_{j\in\ZZ}\in\B^\ZZ$ with finitely many non-zero digits, $z=\varphi(w)=(z_j)_{j\in\ZZ}\in\A^\ZZ$ has only finite number of non-zero digits, and
      \item $\sum_{j\in\ZZ} w_j \beta^j= \sum_{j\in\ZZ} z_j \beta^j$
  \end{enumerate}
  is called \emph{digit set conversion} in base $\beta$. Such a conversion $\varphi$ is said to be \emph{computable in parallel} if it is $p$-local function for some $p\in\NN$. 
\end{defn}


\begin{defn}
Let $\alpha$ be a complex number. A set of all polynomials with integer coefficients evaluated in $\omega$ is denoted by
$$
    \ZZ[\alpha] =\left\{\sum_{i=0}^n a_i \alpha^i : n\in\NN, a_i\in\ZZ \right\}\,.
$$
\end{defn}
Notice that $\ZZ[\alpha]$ is a ring.


\begin{defn}
Let $\omega$ be an algebraic integer with a minimal polynomial $p(x)=x^d +p_{d-1}x^{d-1}+ \cdots + p_1 x+p_0 \in \ZZ[x]$. A matrix 
$$
S_\omega := \begin{pmatrix}
            0 & 0 & \cdots & 0 & -p_0 \\
            1 & 0 & \cdots & 0 & -p_1 \\
            0 & 1 & \cdots & 0 & -p_2 \\
            \vdots &   & \ddots & & \vdots \\
            0 & 0 & \cdots & 1 & -p_{d-1} 
            \end{pmatrix} \in \ZZ^d
$$
is called a \emph{companion matrix} of the minimal polynomial of $\omega$.
\end{defn}

\begin{defn}
Let $\omega$ be an algebraic integer and $d\in\NN$. We define a mapping $\odot_\omega: \ZZ^d \times \ZZ^d \rightarrow \ZZ^d$ by 
$$
a \odot_\omega b := \left(\multMat{a}\right)\cdot \vect{b} \text{ for all } a=\vect{a}, b=\vect{b} \in \ZZ^d\,.
$$ 
\end{defn}


\begin{lem}
Let  $\omega$ be an algebraic integer of degree $d$. Then
$$
\Zomega =\left\{\sum_{i=0}^{d-1} a_i \omega^i : a_i\in\ZZ \right\}
$$ 
and $\Zomega$ is isomorphic to $(\ZZ^d,+,\odot_\omega)$ by a mapping $\pi:\Zomega \rightarrow \ZZ^{d}$ defined by 
$$
\pi(x)=\vect{x}
$$
 for every $x=\sum_{i=0}^{d-1} x_i \omega^i \in \Zomega$. 
\end{lem}
\begin{proof}
Obviously, $\Zomega \supset \left\{\sum_{i=0}^{d-1} a_i \omega^i : a_i\in\ZZ \right\}$. Let $p(x)=x^d +p_{d-1}x^{d-1}+ \dots p_1 x+p_0$ be the minimal polynomial of $\omega$. Assume $u\in \Zomega$, $x=\sum_{i=0}^n u_i \omega^i$ for some $n\geq d$.  By $p(\omega)=0$, we have an equation $\omega^d =-p_{d-1}\omega^{d-1}- \dots -p_1\omega-p_0$ which enables us to write
$$
u=u_n\omega^n + \sum_{i=0}^{n-1} u_i \omega^i=u_n \omega^{n-d}(-b_{d-1}\omega^{d-1}- \dots -b_1\omega-b_0)+ \sum_{i=0}^{n-1} u_i \omega^i\,.
$$
Thus $x\in \left\{\sum_{i=0}^{d-1} a_i \omega^i : a_i\in\ZZ \right\}$ by  induction with respect to $n$.

Let us check now that the mapping $\pi$ is well-define. Assume on contrary that there exists $v\in \Zomega$ and $i_0\in\{0,1,\dots,d-1\}$ such that $v=\sum_{i=0}^{d-1} v_i \omega^i=\sum_{i=0}^{d-1} v'_i \omega^i$ and $v_{i_0} \neq v'_{i_0}$. Then
$$
    \sum_{i=0}^{d-1} (v'_i-v_i) \omega^i=0
$$
and $\sum_{i=0}^{d-1} (v'_i-v_i) x^i \in \ZZ[x]$ is non-zero polynomial of degree smaller than degree of minimal polynomial $p$, a contradiction.

Clearly, $\pi$ is bijection. Let $u=\sum_{i=0}^{d-1} u_i \omega^i$ and $v=\sum_{i=0}^{d-1} v_i \omega^i$ be elements of $\Zomega$. Consider
\begin{align*}
\omega v&=\omega \sum_{i=0}^{d-1} v_i \omega^i = \sum_{i=0}^{d-2} v_i \omega^{i+1} + v_{d-1}(\underbrace{-p_{d-1}\omega^{d-1}- \dots -p_1\omega-p_0}_{=\omega^d}) \\
&= -p_0 v_{d-1} + \sum_{i=1}^{d-1} (v_{i-1}- v_{d-1} p_i) \omega^i\,.
\end{align*}
Hence
\begin{align*}
\pi(\omega v)&=\begin{pmatrix}
              -v_{d-1}p_0 \\
              v_0-v_{d-1}p_1 \\
              v_1-v_{d-1}p_2 \\
              \vdots \\
              v_{d-2}-v_{d-1}p_{d-1}
              \end{pmatrix}=\begin{pmatrix}
            0 & 0 & \cdots & 0 & -p_0 \\
            1 & 0 & \cdots & 0 & -p_1 \\
            0 & 1 & \cdots & 0 & -p_2 \\
            \vdots &   & \ddots & & \vdots \\
            0 & 0 & \cdots & 1 & -p_{d-1} 
            \end{pmatrix}\cdot\vect{v} \\
            &=S_\omega \vect{v} =\begin{pmatrix}
              0\\
              1 \\
              0 \\
              \vdots \\
              0
              \end{pmatrix}\odot_\omega \vect{v}=\pi(\omega)\odot_\omega\pi(v)\,.
\end{align*}
By induction and the last equation we have 
$$
\pi(\omega^{i}v)=\pi(\omega(\omega^{i-1} v))=\pi(\omega)\odot_\omega\pi(\omega^{i-1} v)=\pi(\omega)\odot_\omega(\pi(\omega))^{i-1}\odot_\omega \pi(v)=(\pi(\omega))^{i}\odot_\omega \pi(v)\,.
$$
Now we multiply $v$ by $m\in\ZZ\subset\Zomega$:
\begin{align*}
\pi(m v)&=\pi(m \sum_{i=0}^{d-1} v_i \omega^i)=\pi(\sum_{i=0}^{d-1} m v_i \omega^i)=\vect{mv}=m \mathbb{I} \cdot \vect{v}=\begin{pmatrix}
              m\\
              0 \\
              \vdots \\
              0
              \end{pmatrix}\odot_\omega \vect{v} \\
        &= \pi(m)\odot_\omega\pi(v)\,.
\end{align*}
We conclude by using previous results and obvious aditivity of $\pi$:
\begin{align*}
\pi(uv)&=\pi\left(\sum_{i=0}^{d-1} u_i \omega^i v\right)=\sum_{i=0}^{d-1}\pi(\omega^i u_i  v)=\sum_{i=0}^{d-1}\pi(\omega)^i \odot_\omega\left(\pi(u_i)\odot_\omega\pi(v)\right) \\
    &=\sum_{i=0}^{d-1}\pi(\omega^i u_i)\odot_\omega \pi(v)=\left(\pi(\sum_{i=0}^{d-1}u_i\omega^i)\right)\odot_\omega\pi(v)=\pi(u)\odot_\omega \pi(v)\,.
\end{align*}

\cite{horn}
\end{proof}


























 \section{Previous results}
 
    {Parallel Addition}
    Introduced by Avizienis in 1961:
  \begin{align*}
    \cdots w_{j+t+1}\textcolor{red}{w_{j+t} \cdots w_{j+1}}& \textcolor{red}{w_j w_{j-1} \cdots w_{j-r}}w_{j-r-1} \cdots &,\, w_i &\in \A + \A\,,    \\
    \longrightarrow \cdots z_{j+t+1}\;z_{j+t} \; \cdots \; z_{j+1} &\textcolor{red}{z_j} \; z_{j-1}\; \cdots z_{j-r}\;z_{j-r-1}\; \cdots &,\, z_i &\in \A\,.
  \end{align*}
  Digit conversion is a mapping $(\A+\A)^{t+r+1} \to \A$:%so called \textit{$p$-local function} with the window of the~fixed length $p=t+r+1$: 
  $$
    \textcolor{red}{z_j=z_j(w_{j+t} \cdots w_{j+1}w_j w_{j-1} \cdots w_{j-r})}\,.
  $$
  
  Thus the parallel addition is done in constant time but the numeration system must be redundant -- a number has more than one admissible representation.
  
  
  For example:
  $$
  \beta \in \NN, \beta \geq 3, \A=\{-a, \dots, 0, \dots a\}, b/2 <a \leq b-1\,. 
  $$
    
    
    
        {Non-standard numeration systems}
    
    Integer alphabets:
    \begin{itemize}
        \item Base $\beta \in \CC, |\beta|>1$.
        
        \item Addition is computable in parallel if and only if $\beta$ is an~algebraic number with no conjugate of modulus 1 [Frougny, Heller, Pelantov\'a, S. ].
        
        \item Algorithms are known but with large alphabets.
    \end{itemize}
    
    
    Non-integer alphabets:
    \begin{itemize}
        \item Base $\beta \in \Zomega= \left\{\sum_{j=0}^{d-1} a_j \omega^j\ : a_j \in \ZZ \right\}$, where $\omega \in \CC$ is an~algebraic integer of degree $d$.
        
        \item Alphabet $\A \subset \Zomega , 0\in \A$.
        
        \item Only few manually found algorithms. 
    \end{itemize}
    
    PRO VSECHNY DEFINICE A VETY NENI-LI UVEDENO JINAK
     Let $\omega$ be an algebraic integer and $(\beta,\A)$ be a numeration system such that a base $\beta \in \Zomega$ and an alphabet $\A$ is a finite subset of $\Zomega$. 