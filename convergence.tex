Unfortunately, the extending window method is not always finite. The algorithm may lead to an infinite loop in both phases. 
However, the following lemmma gives a sufficient condition for convergence of the Phase 1. 
\begin{lem}
\label{lem:weightCoefSet}
    Let $\A$ and $\B$ be finite subsets of $\Zomega$ such that $\A$ contains at least one representative of each congruence class modulo $\beta$ in $\Zomega$. NEBO ZBeta???? Then, there exists a subset $\Q\subset\Zomega$ such that $ \B + \Q \subset \A + \beta \Q$ and all elements of $\Q$ are limited by constant $R\in \RR^+$ in modulus.
\end{lem}
\begin{proof}
 Denote $A:=\max\{|a|:a \in \A\}$ and $B:=\max\{|b|:b\in\B\}$. Consequently, set $R:=\frac{A+B}{|\beta|-1}$ and $\Q:=\{q\in\Zomega:|q|\leq R\}$. The set $\Q$ is nonempty because obviously $A>0$. Any element $x=b+q$ with $b\in\B$ and $q\in\Q$ can be written as $x=a+\beta q'$ for some $a\in\A$ due to existence of representative of each congruence class in $\A$. We prove that $|q'|\leq R$:
 $$
    |q'|=\frac{|b+q-a|}{|\beta|}\leq \frac{B+R+A}{|\beta|} \leq \frac{1}{|\beta|}\left(A+B+\frac{A+B}{|\beta|-1}\right)  =\frac{A+B}{|\beta|}\left(\frac{\beta}{|\beta|-1}\right)=R\,.
 $$ 
 Hence $q'\in\Q$ and thus  $x=b+q \in \A + \beta \Q$.
\end{proof}
We plug in the alphabet $\A$ and input alphabet $\B$. Because of the choice of the smallest elements in Algorithm \ref{alg:extendWeightCoefSet}, we know by the lemma that the weight coefficient set $\Q$ constructed in Phase~1 has elements bounded by constant $R$. 
\begin{theo}
    Let $\omega$ be an algebraic integer such that any complex circle centered at 0 contains only finitely  many elements of $\Zomega$. Let $\beta \in \Zomega, |\beta|>1$. Let $\A$ and $\B$ be finite subsets of $\Zomega$ such that $\A$ contains at least one representative of each congruence class modulo $\beta$ in $\Zomega$. NEBO ZBeta???? Then, there exists a finite subset $\Q\subset\Zomega$ such that $ \B + \Q \subset \A + \beta \Q$.
\end{theo} 
\begin{proof}
    Directly from Lemma \ref{lem:weightCoefSet} and assumption on $\Zomega$.
\end{proof}
Therefore, Phase 1 succesfully ends if there can be only finitely many elements in $\Zomega$ NEBO Zbeta??? bounded by constant $R$ as intermediate weight coefficient sets $\Q_k$ have elements bounded by $R$ for all $k$. 
We categorize an algebraic integer $\omega$ which generates $\Zomega \ni \beta$ as follows:
\begin{itemize}
    \item $\omega \in \ZZ$ implies $\Zomega=\ZZ$ which has required property and thus Phase~1 converges.
    \item $\omega \in \RR\setminus\ZZ$ implies $\Zomega$ being dense in $\RR$. CITACE NEBO DUKAZ??? Thus the convergence of Phase~1 is not guaranteed and we have an example for which it does not converge.
    \item $\omega \in \CC\setminus\RR$, $\omega$ being quadratic algebraic integer implies that $\Zomega$ is not dense in $\CC$ and thus Phase~1 can converge. CO NEJAKY DUKAZ NEBO LEPSI ZDUVODNENI?????? NOT DENSE PORAD JESTE NEZNAMENA, ZE JICH NEMUZE BYT NEKONECNE...
    \item $\omega \in \CC\setminus\RR$, $\omega$ being algebraic integer of degree $\geq 3$ implies $\Zomega$ is  dense in $\CC$ and therefore the convergence of Phase~1 is not guaranteed.
\end{itemize}

We focus on Phase~2 now. For shorter notation, set 
$$
\Q^m_b:=\Q_{[\underbrace{b,\dots,b}_m]}
$$ for $m \in \NN$ and $b\in\B$.

Obviously, finitness of Phase~2 implies that there exists a length of window $M$ such that the set $\Q^M_b$ contains only one element for all $b\in\B$. 

\begin{algorithm}
  \caption{Check input $bb\dots b$}
    \label{alg:oneletterSets}
  \begin{algorithmic}[1]
    \REQUIRE{Weight coefficient set $\Q$, digit $b\in\B$}
    \STATE $m:=1$
    \STATE Find minimal set $\Q^1_b \subset \Q$ such that
      $$
      b + \Q \subset \A + \beta \Q^1_b\,.
      $$
      \vspace{-20pt}
    \WHILE{$\#\Q^m_b > 1$}
        \STATE $m:= m +1$
        \STATE By Algorithm \ref{alg:minimalSet}, find minimal set $\Q^m_b \subset \Q^{m-1}_b$ such that
          $$
          b + \Q^{m-1}_b \subset \A + \beta \Q^m_b\,.
          $$  
          \vspace{-20pt}
        \IF{$\#\Q^m_b=\#\Q^{m-1}_b$}
            \RETURN Phase 2 does not converge for input $bb\dots b$.
        \ENDIF
    \ENDWHILE  
    \RETURN Weight coefficient for input $bb\dots b$ is the only element of $\Q^m_b$.
  \end{algorithmic}
\end{algorithm}

For arbitrary $m$, sets $\Q^m_b$  can be easily constructed separately for each $b\in\B$. Using Lemma \ref{lem:stoppingCondition}, Algorithm \ref{alg:oneletterSets} checks whether Phase~2 stops processing input digits $bb\dots b$. Thus,  non-finitness of Phase~2 can be revealed by running it for each input digit $b\in\B$.
\begin{lem}
\label{lem:stoppingCondition}
Let $m_0 \in \NN$ be a length of window and $b\in\B$ be a letter of an input alphabet such that sets $\Q^{m_0}_b$ and $\Q^{{m_0}-1}_b$ produced by Algorithm \ref{alg:minimalSet} within Phase~2 have the same size. Then
$$
    \#\Q^m_b = \#\Q^{m_0}_b \qquad \forall m\geq m_0-1\,.
$$ 
Particulary, if $\#\Q^{m_0}_b\geq 2$, Phase~2 does not converge.
\end{lem}
\begin{proof}
It is enough to prove the base case of an induction as the inductive step is analogic. The set $\Q^{m_0+1}_b$ is find by Algorithm \ref{alg:minimalSet} such that 
$$
b + \Q^{m_0}_b \subset \A + \beta \Q^{m_0+1}_b\,
$$
and set $\Q^{m_0}_b$ was found by the same algorithm such that
$$
b + \Q^{m_0-1}_b \subset \A + \beta \Q^{m_0}_b\,.
$$
As $\Q^{m_0}_b \subset \Q^{{m_0}-1}_b$, the assumption of the same size implies
$$
    \Q^{m_0}_b = \Q^{{m_0}-1}_b\,.
$$
It means that Algorithm \ref{alg:minimalSet} runs with the same input and hence
$$
\Q^{m_0+1}_b=\Q^{m_0}_b\,.
$$
Phase 2 ends when there is only one element in $\Q_{[w_j,\dots, w_{j-m+1}]}$ for all $(w_j,\dots, w_{j-m+1}) \in \B^m$ for some fixed length of window $m$. But if $\#\Q^{m_0}_b\geq 2$, size of $\Q_{[b,\dots,b]}$ does not decrease despite of extending the length of window.
\end{proof}
