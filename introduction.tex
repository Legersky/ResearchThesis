Algorithms for parallel addition have been studied to improve performance of computer power units as well as for the theoretical reasons. Addition is the important part of algorithms for multiplication and division. Thus the linear time of the standard addition algorithms is a serious drawback. A parallel algorithm to compute the sum of $x=x_n x_{n-1}\cdots x_1 x_0 \bullet$ and $y=y_n y_{n-1}\cdots y_1 y_0 \bullet$ determines the $j$-th digit of the sum just from the knowledge of fixed number of digits around $x_j$ and $y_j$. Thus it avoids a carry propagation and all output digits can be determined at the same time. In contrary, the carry propagation in the standard algorithms requires to compute digits one by one.

The parallel addition algorithm for an integer base $\beta\geq3$ was introduced by A. Avizienis in \cite{avizienis} in 1961. The algorithm works on the alphabet $\{-a, \dots, 0, \dots a\}$ where $a\in\NN$ is such that $\beta/2 <a \leq \beta-1$. Later, C. Y. Chow and J. E. Robertson \cite{chow} found an algorithm for the base 2 and the alphabet $\{-1,0,1\}$ in 1978.   

So called non-standard numeration systems, where the base $\beta$ is not a positive integer, have been extensively studied. The reason of this interest is for instance the precise arithmetic in $\QQ(\beta)$. Also, complex bases allow to represent any complex number without separating the real and imaginary part. The example of such system is Penny numeration system with the base $\imath -1$ and the alphabet $\{0,1\}$.

Some redundancy of the numeration is required in order to construct a parallel addition algorithm. It means that numbers may have more than one represenation. P. Kornerup studied the necessary amount of redundancy in \cite{kornerup}. 

 The parallel addition algorithms for several bases (negative integer, complex numbers $-1+\imath, 2\imath$ and $\sqrt{2}\imath$, quadratic Pisot unit and the non-integer rational base) are given in \cite{minAlph}     