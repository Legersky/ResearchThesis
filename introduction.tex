The algorithms for parallel addition have been studied to improve performance of computer power units as well as for the theoretical reasons. Addition is the important part of algorithms for multiplication and division. Thus the linear time of the standard addition algorithms is a serious drawback. A parallel algorithm to compute the sum of $x=x_n x_{n-1}\cdots x_1 x_0 \bullet$ and $y=y_n y_{n-1}\cdots y_1 y_0 \bullet$ determines the $j$-th digit of the sum just from the knowledge of fixed number of digits around $x_j$ and $y_j$. Thus we avoid carry propagation and all output digit can be determined at the same time.In contrary, the carry propagation in the standard algortihms requires to compute digits one by one.






Now we recall known algorithms for parallel addition in different numeration systems. 
DODELAT 
    {Parallel Addition}
    Introduced by Avizienis in 1961:
  
  ROBERTSON
  For example:
  $$
  \beta \in \NN, \beta \geq 3, \A=\{-a, \dots, 0, \dots a\}, b/2 <a \leq b-1\,. 
  $$  
  