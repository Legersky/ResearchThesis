The main goal of this thesis was to design and implement the extending window method in SageMath. In order to do that, we have recalled the definitions and the previous results in the field of parallel addition algorithms. We have proved Theorem \ref{thm:divisibility} which is necessary tool for computation in $\Zomega$.

From the general concept of construction of parallel addition algorithm, we have designed both phases of the extending window method for the rewriting rule $x-\beta$. The sufficient condition for the convergence of Phase 1, i.e., the search for the weight coefficients set $\Q$, have been introduced in Theorem \ref{thm:suffCondPhase1}. Algebraic integers $\omega$ have been categorized according to this sufficient condition. Next, we have developed Algorithm \ref{alg:oneletterSets} which checks the necessary condition for the convergence of Phase 2, i.e. the search for the weight function $q$. This algorithm is based on Theorem \ref{thm:stoppingCondition} which we have proved.

Both phases were implemented in SageMath. The provided graphic user interface is  to be used in SageMath Cloud and the shell interface enables us to compute the weight function for more demanding numeration systems.

We have tested several examples of numeration systems. Our program found the weight function successfully for many of them. We have also examples for which Phase 1 converges, although the sufficient condition given by Theorem \ref{thm:suffCondPhase1} does not hold. 

Many questions remains open:
\begin{itemize}
\item What is the necessary condition of convergence of Phase 1? Is there any connection with modulus of the conjugates of the base?
\item Can we ensure the convergence of Phase 1 for wider class of numeration systems, for instance by using different metric in $\CC$?
% \item Is there any example when necessary condition of Phase 2 is satisfied, but Phase 2 does not converge?
\item How can we improve Phase 2 to converge for more numeration systems?
\end{itemize}



We have observed that the results depend on the way how we pick an element in Algorithm \ref{alg:minimalSet} -- 