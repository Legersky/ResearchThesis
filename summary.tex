The main goal of this thesis was to design and implement the extending window method in SageMath. In order to do that, we have recalled the definitions and the previous results in the field of parallel addition algorithms. We have proved Theorem \ref{thm:divisibility} which is necessary tool for computation in $\Zomega$.

From the general concept of construction of parallel addition algorithm, we have designed both phases of the extending window method for the rewriting rule $x-\beta$. The sufficient condition for the convergence of Phase 1, i.e., the search for the weight coefficients set $\Q$, have been introduced in Theorem \ref{thm:suffCondPhase1}. Algebraic integers $\omega$ have been categorized according to this sufficient condition. Next, we have developed Algorithm \ref{alg:oneletterSets} which checks the necessary condition for the convergence of Phase 2, i.e. the search for the weight function $q$. This algorithm is based on Theorem \ref{thm:stoppingCondition} which we have proved.

Both phases were implemented in SageMath. The provided graphic user interface is  to be used in SageMath Cloud and the shell interface enables us to compute the weight function for more demanding numeration systems.

We have tested several examples of numeration systems. Our program found the weight function successfully for many of them. We have also examples for which Phase 1 converges, although the sufficient condition given by Theorem \ref{thm:suffCondPhase1} does not hold. We think that it caused by the fact that there is a claim similar to Lemma \ref{lem:suffCondPhase1} which guarantees the convergence for bases with all conjugates greater then 1 in modulus by using different norm than absolute value.

We have tried many ways to pick elements in Algorithm \ref{alg:minimalSet} in Phase 2 to obtain convergence of the given examples with the possibly smallest window. We have observed that the chosen way influences results a lot. For example, reducing  size of sets of possible weight coefficients as much as possible in each iteration does not lead to a positive control of the necessary condition for some numeration system whereas less strict choice does. There is still space for further improvements.

Many questions remain open:
\begin{itemize}
\item Can we characterize all bases and alphabets for which Phase 1 converges?
\item How can we improve Phase 2 to converge for more numeration systems? Can the mentioned different norm for Phase 1 help also in Phase 2? 
\item How can be revealed that there is an infinite loop during computation of Phase 2?
\item May non-deterministic choice of elements work in Phase 2?
\item Is there any sufficient condition of convergence of any modification of Phase 2? 
\end{itemize}
