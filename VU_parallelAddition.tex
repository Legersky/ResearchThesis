\documentclass[a4paper, 11pt]{report}

\usepackage{a4wide,cite}
\usepackage[english]{babel}
\usepackage[utf8]{inputenc}
% \usepackage[IL2]{fontenc}  
% \usepackage{amsmath,amsthm} 
% \usepackage{amsfonts}
\usepackage{amsmath, amsthm, amssymb, units, dsfont}

\usepackage[fixlanguage]{babelbib}
\selectbiblanguage{english}

\usepackage[pdfauthor={Jan Legersk\'y},
            pdfproducer={Jan Legersk\'y},
            pdfcreator={pdflatex},
            pdfencoding=unicode]{hyperref}
\usepackage{bookmark}
\hypersetup{pdftitle={Konstrukce algoritmu pro paraleln\'i sc\'it\'an\'i}}
\hypersetup{pdfsubject={V\'yzkumn\'y \'ukol, Jan Legersk\'y}}
\hypersetup{pdfkeywords={parallel addition, non-standard numeration systems}}            

\usepackage[pdftex]{graphicx,color}
\usepackage{fancyhdr}
\pagestyle{fancy}
\fancyhf{}
\fancyhead[L]{\slshape \nouppercase{\rightmark}} % \leftmark .. kapitola
\fancyhead[R]{\thepage}
% \oddsidemargin=10mm   % jednostranný tisk !!!
\setlength{\headheight}{16pt} 


\newcommand{\Zomega}{\mathbb{Z}[\omega]}

\newcommand{\ZZ}{\mathbb{Z}}
\newcommand{\CC}{\mathbb{C}}
\newcommand{\NN}{\mathbb{N}}
\newcommand{\RR}{\mathbb{R}}

\newcommand{\OO}{\mathbb{O}}
\newcommand{\II}{\mathbb{I}}

\newcommand{\A}{\mathcal{A}}
\newcommand{\B}{\mathcal{B}}
\newcommand{\Q}{\mathcal{Q}}

\newcommand{\fin}[1]{\text{Fin}_{#1}(\beta)}

\newcommand{\multMat}[1]{\sum_{i=0}^{d-1} {#1}_i S^i}

\newcommand{\vect}[1]{\begin{pmatrix}
            {#1}_0 \\
            {#1}_1 \\
            \vdots \\
            {#1}_{d-1} 
            \end{pmatrix}}

\usepackage{mathtools, mathdots}

\usepackage{algorithm}
\usepackage{algorithmic}

\usepackage{enumerate}

\renewcommand{\algorithmicrequire}{\textbf{Input:}}
\algsetup{indent=2em}


\newtheorem{theo}{Theorem}[chapter]
\newtheorem{lem}[theo]{Lemma}

\theoremstyle{definition}
\newtheorem{defn}{Definition}[chapter]
\newtheorem{exmp}{Example}[chapter]



\input{listing.tex}

\setcounter{tocdepth}{1}

%%% VYHLEDAT A OPRAVIT !!! %%%%%%%%%%%%%%%%%%%%%%%%%%%%%%%%%%%%%%%%%%%%%
% vlna !!
% taky opravit N-1 na N\!-\!1
% zkontrolovat \min \; ...
%%%%%%%%%%%%%%%%%%%%%%%%%%%%%%%%%%%%%%%%%%%%%%%%%%%%%%%%%%%%%%%%%%%%%%%%

%%% k vyplnění ZAČÁTEK -------------------------------------------------
\newcommand{\cvut}{ČESKÉ VYSOKÉ UČENÍ TECHNICKÉ V~PRAZE}
\newcommand{\fjfi}{Fakulta jaderná a fyzikálně inženýrská}
\newcommand{\km}{Katedra matematiky}
\newcommand{\obor}{Inženýrská informatika}
\newcommand{\zamereni}{Matematická informatika}

\newcommand{\nazevcz}{Konstrukce algoritm\r u pro paraleln\'i s\v c\'it\'an\'i}
\newcommand{\nazeven}{Construction of algorithms for parallel addition}
\newcommand{\autor}{Jan Legersk\'y}
\newcommand{\vedouci}{Ing. \v St\v ep\'an Starosta, Ph.D.}
\newcommand{\pracovisteVed}{KAM FIT, \v CVUT v~Praze}
\newcommand{\konzultant}{---}

\newcommand{\klicova}{Paraleln\'i s\v c\'it\'an\'i, nestandardn\'i numera\v cn\'i syst\'emy, extending window method.}  % max. 5 klíčových slov
\newcommand{\keyword}{Parallel addition, non-standard numeration systems, extending window method.}
\newcommand{\abstrCZ}%{\begin{minipage}{0.7\textwidth}
{Pr\'{a}ce je v\v{e}nov\'{a}na konstrukci algoritm\r{u} pro paraleln\'{\i} s\v{c}\'{\i}t\'{a}n\'{\i} v r\r{u}zn\'{y}ch numera\v{c}n\'{\i}ch soustav\'{a}ch. Zam\v{e}\v{r}uje se hlavn\v{e} na nestandardn\'{\i} numera\v{c}n\'{\i} syst\'{e}my s necelo\v{c}\'{\i}selnou, obecn\v{e} komplexn\'{\i}, b\'{a}z\'{\i} $\beta\in\Zomega$ a obecn\v{e} necelo\v{c}\'{\i}selnou abecedou $\A\subset\Zomega$ pro n\v{e}jak\'{e} algebraick\'{e} cel\'{e} \v{c}\'{\i}slo $\omega$. Navrhujeme takzvanou \emph{extending window method} s pou\v{z}it\'{\i}m z\'{a}kladn\'{\i}ho p\v{r}episovac\'{\i}ho pravidla $x-\beta$, kter\'{a} pro danou b\'{a}zi $\beta$ a abecedu $\A$ hled\'{a} algoritmus pro paraleln\'{\i} s\v{c}\'{\i}t\'{a}n\'{\i}. Metoda se skl\'{a}d\'{a} ze dvou f\'{a}z\'{\i}. Pro prvn\'{\i} z nich uv\'{a}d\'{\i}me posta\v{c}uj\'{\i}c\'{\i} podm\'{\i}nku konvergence, pro druhou m\'{a}me algoritmus, kter\'{y} kontroluje nutnou podm\'{\i}nku konvergence. Tuto metodu implementujeme v programovac\'{\i}m jazyce SageMath a uv\'{a}d\'{\i}me mno\v{z}stv\'{\i} otestovan\'{y}ch numera\v{c}n\'{\i}ch syst\'{e}m\r{u}.} 
% \end{minipage}}
\newcommand{\abstrEN}{We focus on the construction of algorithms for parallel addition in different numeration systems, especially non-standard ones. The base $\beta$ is an element of $\Zomega$ and the alphabet, in general non-integer, is a subset of $\Zomega$ for some algebraic integer $\omega$. We design so-called extending window method with the basic rewriting rule $x-\beta$. The method searchs for a parallel addition algorithm for a given base $\beta$ and alphabet $\A$. It consists of two phases. We have the sufficient condition of convergence of Phase 1. We introduce the algorithm which verifies the necessary condition of convergence of Phase 2. The method is implemented in SageMath and we provide several tested examples.}
%%% k vyplnění KONEC ---------------------------------------------------


\begin{document}

\pagenumbering{roman}
\begin{titlepage}

%% 1. úvodní strana
% \thispagestyle{empty}
% \begin{center}
% 	{\Large \cvut \\[12pt] \fjfi\\}   % 260pt od nadpisu k "Bak. práce"
% 	\vspace{45pt}
% 	\includegraphics[height=100pt]{img/logoCVUT.pdf}\\  %bylo  75
% 	\vspace{160pt}
% 	{\Huge V\'YZKUMN\'Y \'UKOL}
% \end{center}
% \vfill
% {
% 	\Large 2015 \hfill Jan Legersk\'y
% }
% \newpage


%%% 2. úvodní strana
\thispagestyle{empty}
\begin{center}
	{\Large \cvut \\[10pt] \fjfi \\[10pt] \km\\}
	\vspace{45pt} %120
	\includegraphics[height=100pt]{img/logoCVUT.pdf}\\
	\vspace{90pt}
	{\Large V\'YZKUMN\'Y \'UKOL}
	\vspace{90pt}
	
	{\Large\bf \nazevcz}
	\vspace{30pt}
	
	{\Large\bf \nazeven}
\end{center}
\vfill
{
	\Large
	\begin{tabular}{ll}
	Vypracoval: & \autor\\[3pt]
	\v Skolitel: & \vedouci\\[3pt]
	Akademick\'y rok: & 2014/2015
	\end{tabular}
}
\newpage


%%% sem přijde ofic. zadání
% \thispagestyle{empty}
% \Large
% Na toto místo přijde svázat \textbf{zadání diplomové práce}!
% 
% \vspace{4mm}
% V~jednom z~výtisků musí být \textbf{originál} zadání, v~ostatních kopie.
% \normalsize
\rule{0cm}{0cm}
\thispagestyle{empty}
\newpage


%%% prohlášení
\thispagestyle{empty}
~
\vfill
\noindent\textbf{Čestné prohlášení}
\vspace{0.5cm}

Prohla\v suji na tomto m\'ist\v e, \v ze jsem p\v redlo\v zenou pr\'aci vypracoval samostatn\v e a \v ze jsem uvedl ve\v skerou pou\v zitou literaturu.
\vspace{1.5cm}

\noindent
\vspace{5mm}V Praze dne 2. 9. 2015\hfill
	\begin{tabular}{c}
	\\
	\autor
	\end{tabular}
\newpage


%%% poděkování
\thispagestyle{empty}
~
\vfill
\noindent\textbf{Pod\v ekov\'an\'i}
\vspace{0.5cm}

Děkuji Ing. Štěpánu Starostovi, Ph.D., za vedení mého výzkumného úkolu. Dále děkuji Ing. Mileně Svobodové, Ph.D., a prof. Ing. Editě Pelantové, CSc., za podrobné vysvětlení implementované metody a podnětné diskuze.

\begin{flushright}
Jan Legersk\'y
\end{flushright}
\newpage


%%% abstrakt atp.
\thispagestyle{empty}

\begin{tabular}{lp{0.76\textwidth}}
  {\em N\'azev pr\'ace:} & \bf \nazevcz \\[4mm]
  {\em Autor:} & \autor \\[4mm]
  {\em Obor:} & \obor \\[4mm]
  {\em Zam\v e\v ren\'i:} & \zamereni \\[4mm]
  {\em Druh pr\'ace:} & V\'yzkumn\'y \'ukol \\[4mm]
  {\em Vedoucí práce:} & \vedouci, \pracovisteVed \\[4mm]
  {\em Konzultant:} & \konzultant \\[4mm]
  {\em Abstrakt:} & \abstrCZ \\[4mm]
  {\em Kl\'i\v cov\'a slova:} & \klicova \\[20mm]

  {\em Title:} & \bf \nazeven \\[4mm]
  {\em Author:} & \autor \\[4mm]
  {\em Abstract:} & \abstrEN \\[4mm]
  {\em Key words:} & \keyword
\end{tabular}
\newpage

\rule{0cm}{0cm}
\thispagestyle{empty}
\newpage

%%% obsah
\renewcommand\contentsname{\vspace{-3cm} Contents \vspace{-0.8cm}}
\tableofcontents
\thispagestyle{empty}

\end{titlepage}
\pagenumbering{arabic}


%%%%%%%%%%%%%%%%%%%%%%%%%%%%%%%%%%%%%%%%%%%%%%%%%%%%%%%%%%%%%%%%%%%%%%%%
%     Z A Č Á T E K   P R Á C E                                        %
%%%%%%%%%%%%%%%%%%%%%%%%%%%%%%%%%%%%%%%%%%%%%%%%%%%%%%%%%%%%%%%%%%%%%%%%
\chapter*{List of symbols}
% \centering
\begin{tabular}{ll}
Symbol        & Description \\ \hline
$\NN$         & set of nonnegative integers $\{0,1,2,3,\dots\}$   \\
$\ZZ$         & set of integers $\{\dots,-2,-1,0,1,2,\dots\}$ \\
$\RR$           & set of real numbers \\
$\CC$           & set of complex numbers \\
$\QQ(\beta)$    &the smallest field containing the set of rational numbers $\QQ$ and $\beta$ \\
$\#S$          & number of elements of the finite set $S$ \\
\rule{0cm}{0cm}& \\
$(\beta,\A)$            & numeration system with the base $\beta$ and the alphabet $\A$\\
$(x)_{\beta,\A}$    &$(\beta,\A)$-representation of the number $x$\\
$\fin{\A}$          &set of all complex numbers with a finite $(\beta,\A)$-representation \\
$\A^\ZZ$        &set of all bi-infinite sequences of digits in $\A$\\
$\Zomega$       &set of values of all polynomials with integer coefficients evaluated in $\omega$\\
$\pi$           &isomorphism from $\Zomega$ to $\ZZ^d$ \\
\rule{0cm}{0cm}& \\
$\B$            &alphabet of input digits\\
$q_j$           &weight coefficient for the $j$-th position \\
$\Q$            &weight coefficients set\\
$\Q_{[w_{j},\dots, w_{j-m+1}]}$ &set of possible weight coefficients for the input digits $w_{j},\dots, w_{j-m+1}$ \\
\rule{0cm}{0cm}& \\
$\lfloor x \rfloor$ & floor function of the number $x$ \\  
$\Re x$           & real part of the complex number $x$ \\
$\Im x$           & imaginary part of the complex number $x$
\end{tabular}
\addcontentsline{toc}{chapter}{List of symbols}

\chapter*{Introduction}
\addcontentsline{toc}{chapter}{Introduction}
Now we recall known algorithms for parallel addition in different numeration systems. 
DODELAT 
    {Parallel Addition}
    Introduced by Avizienis in 1961:
  
  ROBERTSON
  For example:
  $$
  \beta \in \NN, \beta \geq 3, \A=\{-a, \dots, 0, \dots a\}, b/2 <a \leq b-1\,. 
  $$  
  

\chapter{Preliminaries}
\label{chap:preliminaries}
In this chapter, we recall few definitions and results connected to numeration systems and parallelism. We define the set $\Zomega$ for an algebraic integer $\omega$ and we prove that $\Zomega$ is isomorphic to $\ZZ^d$. This property is used in Theorem \ref{thm:divisibility} which is an important tool for divisibility in $\Zomega$. Division in $\Zomega$ is necessary for the extending window method described in Chapter \ref{chap:methodDescription}.

\section{Numeration systems}
Firstly, we give a general definition of numeration system.
\begin{defn}
  Let $\beta \in \CC, |\beta|>1$ and $\A \subset \CC$ be a finite set containing 0. A pair $(\beta, \A)$ is called a \emph{positional numeration system} with \emph{base} $\beta$ and \emph{digit set} $\A$, usually called \emph{alphabet}.
\end{defn}
So-called standard numeration systems have an integer base $\beta$ and an alphabet $\A$ which is a set of contiguous integers. We restrict ourselves to base $\beta$ which is an algebraic integer and possibly non-integer alphabet $\A$. 
\begin{defn}
Let $(\beta, \A)$ be a positional numeration system.  We say that a complex number $x$ has a \emph{$(\beta, \A)$-representation} if~ there exist digits $x_n,x_{n-1}, x_{n-2},\dots \in\A, n\geq 0$ such that $x=\sum_{j=-\infty}^n x_j \beta^j$.
\end{defn}
 We write briefly a \emph{representation} instead of a $(\beta, \A)$-representation if the base $\beta$ and the alphabet $\A$ follow from context. 

\begin{defn}
Let $(\beta, \A)$ be a positional numeration system. The set of all complex numbers with a finite $(\beta, \A)$-representation is defined by
$$
    \fin{\A}:=\left\{\sum_{j=-m}^n x_j \beta^j\colon n, m \in \NN, x_j \in \A \right\}\,.
$$
\end{defn}
   
For  $x\in\fin{\A}$, we write 
$$
(x)_{\beta,\A}= 0^\omega x_n x_{n-1}\cdots x_1 x_0 \bullet x_{-1} x_{-2} \cdots x_{-m} 0^\omega\,,
$$ 
where $0^\omega$ denotes right, respectively left-infinite sequence of zeros. Notice that indices are decreasing from left to right as it is usual to write the most significant digits first. In what follows, we omit the starting and ending $0^\omega$ when we work with numbers in $\fin{\A}$. We remark that existence of an algorithm (standard or parallel) producing a finite $(\beta,\A)$-representation of $x+y$ where $x,y\in\fin{\A}$ implies that the set $\fin{\A}$ is closed under addition, i.e.,
$$
\fin{\A} + \fin{\A} \subset \fin{\A}\,.
$$ 

Designing an algorithm for parallel addition requires some redundancy in numeration system. According to \cite{redundant}, a numeration system $(\beta,\A)$ is called \emph{redundant} if there exists $x \in \fin{\A}$ which has two different $(\beta,\A)$-representations. For instance, the number 1 has $(2,\{-1,0,1\})$-representations $1\bullet$ and $1(-1)\bullet$.
Redundant numeration system can enable us to avoid carry propagation in addition. On the other hand, there are some disadvantages. For example, comparison is problematic.  


\section{Parallel addition}
A local function, which is also often called sliding block code, is used to mathematically formalize parallelism. 
\begin{defn}
Let $\A$ and $\B$ be alphabets. A function $\varphi:\B^\ZZ \rightarrow \A^\ZZ$ is said to be \emph{$p$-local} if there exist $r,t\in\NN$ satisfying $p=r+t+1$ and a function $\phi: \B^p \rightarrow \A$ such that, for any $w=(w_j)_{j\in\ZZ}\in\B^\ZZ$ and its image $z=\varphi(w)=(z_j)_{j\in\ZZ}\in\A^\ZZ$, we have $z_j=\phi(w_{j+t},\cdots,w_{j-r})$ for every $j\in\ZZ$. The parameter $t$, resp. $r$, is called \emph{anticipation}, resp. \emph{memory}.
\end{defn}
This means that each digit of the image $\varphi(w)$ is computed from $p$ digits of $w$ in a sliding window. Suppose that there is a processor on  each position with access to $t$ input digits on the left and $r$ input digits on the right. Then computation of $\varphi(w)$, where $w$ finite sequence, can be done in constant time independent on the length of $w$.   
  
\begin{defn}
\label{def:digitSetConversion}
Let $\beta$ be a base and $\A$ and $\B$ two alphabets containing 0. A function $\varphi:\B^\ZZ\rightarrow \A^\ZZ$ such that
  \begin{enumerate}
      \item for any $w=(w_j)_{j\in\ZZ}\in\B^\ZZ$ with finitely many non-zero digits, $z=\varphi(w)=(z_j)_{j\in\ZZ}\in\A^\ZZ$ has only finite number of non-zero digits, and
      \item $\sum_{j\in\ZZ} w_j \beta^j= \sum_{j\in\ZZ} z_j \beta^j$
  \end{enumerate}
  is called \emph{digit set conversion} in base $\beta$ from $\B$ to $\A$. Such a conversion $\varphi$ is said to be \emph{computable in parallel} if it is $p$-local function for some $p\in\NN$. 
\end{defn}
In fact, addition on $\fin{\A}$ can be performed in parallel if there is digit set conversion from $\A+\A$ to $\A$ computable in  parallel as we can easily output digitwise sum of two $(\beta,\A)$-representations in parallel.   


We recall few results about parallel addition in a numeration system with an integer alphabet. C. Frougny, E. Pelantov\'a and M. Svobodov\'a proved in \cite{parAddNS} the following sufficient condition of existence of an algorithm for parallel addition.
  \begin{theo}
  \label{thm:suffConjugates}
  Let $\beta\in\CC$ be an algebraic number such that $|\beta|>1$ and all its conjugates in modulus differ from 1. There exists an alphabet $\A$ of contiguous integers containing 0 such that addition on $\fin{\A}$ can be performed in parallel.
  \end{theo}
  The proof of the theorem provides the algorithm for the alphabet of the form $\{-a,-a+1, \dots,0,\dots,a-1,a\}$. But in general, $a$ is not minimal.
    
The same authors showed in \cite{kBlock} that the condition on the conjugates of the base $\beta$ is also necessary:
  \begin{theo}
  Let the base $\beta\in\CC, |\beta|>1,$ be an algebraic number with a conjugate $\beta'$ such that $|\beta'|=1$. Let $\A\subset\ZZ$ be an alphabet of contiguous integers containing 0. Then addition on $\fin{\A}$ cannot be computable in parallel.
  \end{theo}
  
The question of minimality of the alphabet is studied in \cite{minAlph}. The following lower bound for the size of the alphabet is provided:
  \begin{theo}
  \label{thm:lowerBoundAlphabet}
  Let $\beta\in\CC, |\beta|>1,$  be an algebraic integer with the minimal polynomial $p$. Let $\A\subset\ZZ$ be an alphabet of contiguous integers containing 0 and 1. If addition on $\fin{\A}$ is computable in parallel, then $\#\A \geq |p(1)|$. Moreover, if $\beta$ is a positive real number, $\beta>1$, then $\#\A \geq  |p(1)|+2$.
  \end{theo}
  

In this thesis, we work in a more general concept as we consider also non-integer alphabets. First, we recall the following definition.
\begin{defn}
Let $\omega$ be a complex number. The set of values of all polynomials with integer coefficients evaluated in $\omega$ is denoted by
$$
    \ZZ[\omega] =\left\{\sum_{i=0}^n a_i \omega^i\colon n\in\NN, a_i\in\ZZ \right\} \subset \QQ(\omega)\,.
$$
\end{defn}
 Notice that $\ZZ[\omega]$ is a commutative ring (for our purposes, a ring is associative under multiplication and there is a multiplicative identity).     
    
From now on, let $\omega$ be an algebraic integer  which generates the set $\Zomega$ and let the base $\beta\in\Zomega$ be such that $|\beta|>1$. We remark that $\beta$ is also an algebraic integer as all elements of $\Zomega$ are algebraic integers. Finally, let the alphabet $\A$ be a finite subset of $\Zomega$ such that $0\in\A$.

Few parallel addition algorithms for such numeration system with a non-integer alphabet were found ad hoc. We introduce the method for construction of the parallel addition algorithm for a given numeration system $(\beta,\A)$ in Chapter \ref{chap:methodDescription}. 
  


\section{\texorpdfstring{Isomorphism of $\Zomega$ and $\ZZ^{d}$}{Isomorphism of Z[omega] and Zd}}
The goal of this section is to show a connection between the ring $\Zomega$ and the set $\ZZ^d$. Using Theorem \ref{thm:divisibility}, division in $\Zomega$ can be replaced by searching for an integer solution of a linear system. This is used for the implementation of the extending window method.

First we recall the notion of companion matrix which we use to define multiplication in $\ZZ^d$. By the minimal polynomial of an algebraic integer, we always mean the monic minimal polynomial.  
\begin{defn}
Let $\omega$ be an algebraic integer of degree $d\geq 1$ with the  minimal polynomial $p(x)=x^d +p_{d-1}x^{d-1}+ \cdots + p_1 x+p_0 \in \ZZ[x]$. The matrix 
$$
S := \begin{pmatrix}
            0 & 0 & \cdots & 0 & -p_0 \\
            1 & 0 & \cdots & 0 & -p_1 \\
            0 & 1 & \cdots & 0 & -p_2 \\
            \vdots &   & \ddots & & \vdots \\
            0 & 0 & \cdots & 1 & -p_{d-1} 
            \end{pmatrix} \in \ZZ^{d\times d}
$$
is called \emph{companion matrix} of the minimal polynomial of $\omega$.
\end{defn}
In what follows, the standard basis vectors of $\ZZ^d$  are denoted by 
$$
e_0=\begin{pmatrix}
              1 \\
              0 \\
              0 \\
              \vdots \\
              0
              \end{pmatrix}, \\
e_1=\begin{pmatrix}
              0 \\
              1 \\
              0 \\
              \vdots \\
              0
              \end{pmatrix}, \dots ,\\
e_{d-1}=\begin{pmatrix}
              0 \\        
              \vdots \\
              0 \\
              0\\
              1
              \end{pmatrix}\,.             
$$
% We remark that 1 in $e_i$ is in the $(i+1)$-st row because the index corresponds to the power of a companion matrix in the following definition. 

\begin{defn}
Let $\omega$ be an algebraic integer of degree $d\geq 1$, let $p$ be its minimal polynomial and let $S$ be its companion matrix. We define the mapping $\odot_\omega: \ZZ^d \times \ZZ^d \rightarrow \ZZ^d$ by 
$$
u \odot_\omega v := \left(\multMat{u}\right)\cdot \vect{v} \quad \text{ for all } u=\vect{u}, v=\vect{v} \in \ZZ^d\,.
$$ 
and we define powers of $u \in \ZZ^d$ by
\begin{align*}
    u^0&=e_0, \\
    u^{i}&= u^{i-1} \odot_\omega u \text{ for } i\in\NN\,.
\end{align*}
\end{defn}

We will see later that $\ZZ^d$ equipped with elementwise addition and multiplication $\odot_\omega$ builds a commutative ring. 
% It will follow from the isomorphism with $\Zomega$. 
Let us first recall an important property of a companion matrix  -- it is a root of its defining polynomial.
\begin{lem}
\label{lem:compMatrixIsRoot}
Let $\omega$ be an algebraic integer with a minimal polynomial $p$ and let $S$ be its companion matrix. Then
$$
p(S)=0\,.
$$
\end{lem}
\begin{proof}
Following the proof in \cite{horn}, we have
\begin{align*}
e_0&=S^0 e_0\,, \\
S e_0= e_1&=S^1 e_0\,, \\
S e_1= e_2&=S^2 e_0\,, \\
S e_2= e_3&=S^3 e_0\,, \\
\vdots & \\
S e_{d-2}= e_{d-1}&=S^{d-1} e_0\,, \\
S e_{d-1} &= S^{d} e_0\,,
\end{align*}
where the middle column is obtained by multiplication and the right one by using the previous row. 
Also by multiplying and substituting
\begin{align*}
S^{d} e_0=S e_{d-1}&= -p_0e_0-p_1e_1-\cdots-p_{d-1}e_{d-1} \\
    &= -p_0 S^{0}e_0-p_1S^{1}e_0-\cdots-p_{d-1}S^{d-1}e_{0} \\
    &= (-p_0 S^{0}-p_1S^{1}-\cdots-p_{d-1}S^{d-1})e_{0} \\
    &=(S^{d}-p(S))e_0\,.
\end{align*}
Hence
$$
p(S)e_0=0\,.
$$
Moreover,
$$
p(S)e_k=p(S)S^k e_0=S^k p(S) e_0=0
$$
for $k=\{0,1,\dots,d-1\}$ which implies the statement.
\end{proof}

The following lemma summarizes basic properties of the mapping $\odot_\omega$ -- multiplication by an integer scalar, the identity element, the distributive law and a weaker form of associativity.
\begin{lem}
\label{lem:multInZd}
Let $\omega$ be an algebraic integer of degree $d$. The following statements hold for every $u,v,w\in \ZZ^d$ and $m\in\ZZ$:
\begin{enumerate}[(i)]
    \item $(mu)\odot_\omega v = u \odot_\omega (m v)= m (u\odot_\omega v)$,
    \item $e_0 \odot_\omega v= v \odot_\omega e_0 =v$,
    \item $(u \odot_\omega e_1^k)\odot_\omega v = u \odot_\omega (e_1^k\odot_\omega v)$ for $k\in\NN$,
    \item $(u+v)\odot_\omega w =u\odot_\omega w + v\odot_\omega w$ \ and \ $u \odot_\omega (v+w)= u \odot_\omega v +u\odot_\omega w$.
\end{enumerate}
\end{lem}
\begin{proof}
It is easy to see (i) as multiplication of a matrix by a scalar commutes and a scalar can be factored out of a sum. 

The first equality of (ii) follows from definition and
$$
v \odot_\omega e_0=\multMat{v}\cdot e_0= \sum_{i=0}^{d-1} v_i e_i = v\,.
$$
For (iii), we use Lemma \ref{lem:compMatrixIsRoot} and its proof. Assume $k=1$:
\begin{align*}
(u \odot_\omega e_1)\odot_\omega v &= \left(\multMat{u} \cdot e_1\right) \odot_\omega v = \left(\multMat{u} \cdot S e_0\right) \odot_\omega v \\
    &=\left(\sum_{i=0}^{d-2}u_i e_{i+1} + u_{d-1} S^d e_{0} \right)\odot_\omega v= \left(\sum_{i=1}^{d-1}u_{i-1} e_{i} - u_{d-1} \sum_{i=0}^{d-1} p_i e_{i} \right)\odot_\omega v \\
    &=\left(\sum_{i=1}^{d-1}u_{i-1}S^i - u_{d-1} \sum_{i=0}^{d-1}p_i S^i \right)\cdot v \\
    &=\left(\sum_{i=1}^{d-1}u_{i-1}S^i  + u_{d-1}S^d\right)\cdot v =\sum_{i=0}^{d-1}u_{i}S^i\cdot S\cdot v \\
    &=u \odot_\omega (S\cdot v)=u \odot_\omega (e_1\odot_\omega v)\,.
\end{align*}
Now we proceed by induction:
\begin{align*}
\left(u \odot_\omega e_1^k\right)\odot_\omega v &=\left(u \odot_\omega (e_1^{k-1}\odot_\omega e_1) \right)\odot_\omega v = \left((u \odot_\omega e_1^{k-1})\odot_\omega e_1 \right)\odot_\omega v \\
    &= (u \odot_\omega e_1^{k-1})\odot_\omega \left(e_1 \odot_\omega v \right)= u \odot_\omega \left( e_1^{k-1}\odot_\omega (e_1 \odot_\omega v )\right)\\
    &= u \odot_\omega \left(( e_1^{k-1}\odot_\omega e_1 )\odot_\omega v \right) = u \odot_\omega \left(e_1^k\odot_\omega v\right)\,.
\end{align*}
The statement (iv) follows easily from distributivity of matrix multiplication with respect to addition. 
\end{proof}




Now we can prove that there is a correspondence between elements of $\Zomega$ and $\ZZ^d$.

\begin{theo}
Let  $\omega$ be an algebraic integer of degree $d$. Then 
$$
\Zomega =\left\{\sum_{i=0}^{d-1} a_i \omega^i \colon a_i\in\ZZ \right\},
$$ 
$(\ZZ^d,+,\odot_\omega)$ is a commutative ring and the mapping $\pi:\Zomega \rightarrow \ZZ^{d}$ defined by 
$$
\pi(u)=\vect{u} \quad \text{ for every } u=\sum_{i=0}^{d-1} u_i \omega^i \in \Zomega
$$
is a ring isomorphism.
\end{theo}
\begin{proof}
Obviously, $\left\{\sum_{i=0}^{n} a_i \omega^i\colon n\in\NN, a_i\in\ZZ \right\}=\Zomega \supset \left\{\sum_{i=0}^{d-1} a_i \omega^i\colon a_i\in\ZZ \right\}$. We prove the opposite direction by the induction with respect to $n$. Assume $u\in \Zomega$, $u=\sum_{i=0}^n u_i \omega^i$ for some $n\in\NN$. We see that $u\in \left\{\sum_{i=0}^{d-1} a_i \omega^i\colon a_i\in\ZZ \right\}$ for all $n< d$. 

Suppose now that the claim holds for $n-1$ and consider $n\geq d$. Let $p(x)=x^d +p_{d-1}x^{d-1}+ \dots p_1 x+p_0$ be the minimal polynomial of $\omega$.  By $p(\omega)=0$, we have an equation $\omega^d =-p_{d-1}\omega^{d-1}- \dots -p_1\omega-p_0$ which enables us to write
\begin{align*}
u&=u_n\omega^n + \sum_{i=0}^{n-1} u_i \omega^i=u_n \omega^{n-d}(\underbrace{-p_{d-1}\omega^{d-1}- \dots -p_1\omega-p_0}_{\omega^d})+ \sum_{i=0}^{n-1} u_i \omega^i\\
    &=\sum_{i=0}^{n-d-1} u_i \omega^i+ \sum_{i=n-d}^{n-1} (u_i-u_n \cdot p_{i-n+d}) \omega^i=\sum_{i=0}^{n-1} u'_i \omega^i\,.
\end{align*}

Thus $u\in \left\{\sum_{i=0}^{d-1} a_i \omega^i \colon a_i\in\ZZ \right\}$ by the induction assumption.

Let us check now that the mapping $\pi$ is well-defined. Assume on contrary that there exists $v\in \Zomega$ and $i_0\in\{0,1,\dots,d-1\}$ such that $v=\sum_{i=0}^{d-1} v_i \omega^i=\sum_{i=0}^{d-1} v'_i \omega^i$ and $v_{i_0} \neq v'_{i_0}$. Then
$$
    \sum_{i=0}^{d-1} (v'_i-v_i) \omega^i=0
$$
and $\sum_{i=0}^{d-1} (v'_i-v_i) x^i \in \ZZ[x]$ is non-zero polynomial of degree smaller than the degree $d$ of minimal polynomial $p$, a contradiction.

Clearly, $\pi$ is bijection. 

Let $v=\sum_{i=0}^{d-1} v_i \omega^i$ be element of $\Zomega$. We prove by the induction that 
$$
\pi(\omega^{i} v)=(\pi(\omega))^{i}\odot_\omega \pi(v)\,.
$$ 
For $i=1$, consider
\begin{align*}
\omega v&=\omega \sum_{i=0}^{d-1} v_i \omega^i = \sum_{i=0}^{d-2} v_i \omega^{i+1} + v_{d-1}(\underbrace{-p_{d-1}\omega^{d-1}- \dots -p_1\omega-p_0}_{=\omega^d}) \\
&= -p_0 v_{d-1} + \sum_{i=1}^{d-1} (v_{i-1}- v_{d-1} p_i) \omega^i\,.
\end{align*}
Hence
\begin{align*}
\pi(\omega v)&= -p_0 v_{d-1} e_0 + \sum_{i=1}^{d-1} (v_{i-1}- v_{d-1} p_i) e_i = S \cdot \pi(v) \\
    &=e_1\odot_\omega \pi(v)=\pi(\omega)\odot_\omega\pi(v)\,.
\end{align*}
Suppose now for the induction that
$$
\pi(\omega^{i-1} v)=(\pi(\omega))^{i-1}\odot_\omega \pi(v)\,.
$$ 
Then
$$
\pi(\omega^{i}v)=\pi(\omega(\omega^{i-1} v))=\pi(\omega)\odot_\omega\pi(\omega^{i-1} v)=\pi(\omega)\odot_\omega\left((\pi(\omega))^{i-1}\odot_\omega \pi(v)\right)=(\pi(\omega))^{i}\odot_\omega \pi(v)\,,
$$
where we use (iii) of Lemma \ref{lem:multInZd} for the last equality.

Now we multiply $v$ by $m\in\ZZ\subset\Zomega$:
\begin{align*}
\pi(m v)&=\pi\left(m \sum_{i=0}^{d-1} v_i \omega^i\right)=\pi \left(\sum_{i=0}^{d-1} m v_i \omega^i\right)=m \pi(v)= (m e_0) \odot_\omega\pi(v)= \pi(m)\odot_\omega\pi(v)\,.
\end{align*}
% \begin{align*}
% \pi(m v)&=\pi(m \sum_{i=0}^{d-1} v_i \omega^i)=\pi(\sum_{i=0}^{d-1} m v_i \omega^i)=\vect{mv}=m \mathbb{I} \cdot \vect{v}=\begin{pmatrix}
%               m\\
%               0 \\
%               \vdots \\
%               0
%               \end{pmatrix}\odot_\omega \vect{v} \\
%         &= \pi(m)\odot_\omega\pi(v)\,.
% \end{align*}
Let $u=\sum_{i=0}^{d-1} u_i \omega^i\in\Zomega$. Since $\pi$ is obviously additive, we conclude:
\begin{align*}
\pi(uv)&=\pi\left(\sum_{i=0}^{d-1} u_i \omega^i v\right)=\sum_{i=0}^{d-1}\pi(\omega^i u_i  v)=\sum_{i=0}^{d-1}\pi(\omega)^i \odot_\omega\left(\pi(u_i)\odot_\omega\pi(v)\right) \\
    &=\sum_{i=0}^{d-1}\pi(\omega^i u_i)\odot_\omega \pi(v)=\pi\left(\sum_{i=0}^{d-1}u_i\omega^i\right)\odot_\omega\pi(v)=\pi(u)\odot_\omega \pi(v)\,.
\end{align*}
Now we can show that the operation $\odot_\omega$ is associative and commutative. Let $f,g,h\in \ZZ^d$ and $u,v,w\in\Zomega$ such that $f=\pi(u),g=\pi(v)$ and $h=\pi(w)$. Then
$$
f\odot_\omega(g\odot_\omega h)=f\odot_\omega\pi(vw)=\pi(u(vw))=\pi((uv)w)=\pi(uv)\odot_\omega h=(f\odot_\omega g)\odot_\omega h
$$
and
$$
g\odot_\omega h=\pi(vw)=\pi(wv)=h\odot_\omega g\,.
$$
Thus, $(\ZZ^d,+,\odot_\omega)$ is a commutative ring.
\end{proof}

Due to this theorem we may work with integer vectors instead of elements of $\Zomega$ and multiplication in $\Zomega$ is replaced by multiplying by an appropriate matrix. 

The last theorem of this section is a practical tool for divisibility in $\Zomega$. To check whether an element of $\Zomega$ is divisible by another element, we look for an integer solution of a linear system. Moreover, this solution provides a result of the division in the positive case. 
\begin{theo}
\label{thm:divisibility}
Let $\omega$ be an algebraic integer of degree $d$ and let $S$ be the companion matrix of its minimal polynomial. Let $\beta=\sum_{i=0}^{d-1} b_i \omega^i$ be a nonzero element of $\Zomega$. Then for every $u\in\Zomega$
$$
u\in\beta\Zomega \iff S_\beta^{-1}\cdot \pi(u) \in \ZZ^d\,,
$$
where $S_\beta=\multMat{b}$.
\end{theo}
\begin{proof}
Observe first that $S_\beta$ is nonsingular. Otherwise, there exists $y=\vect{y} \in \ZZ^d, y\neq \mathbf{0}$ such that $S_\beta \cdot y=0$. Thus
$$
\pi(\beta)\odot_\omega y=\mathbf{0} \iff \beta \pi^{-1}(y)=0\,.
$$
Since $\beta\neq 0$, we have
$$
0=\pi^{-1}(y)=\sum_{i=0}^{d-1} y_i \omega^i\,,
$$
which contradict that the degree of $\omega$ is $d$.

Now
\begin{align*}
u\in\beta\Zomega &\iff (\exists v \in \Zomega)(u=\beta v)\\
    &\iff  (\exists v \in \Zomega)(\pi(u)=\pi(\beta)\odot_\omega\pi(v)=S_\beta \cdot \pi(v))\\
    &\iff \pi(v)=S_\beta^{-1} \cdot \pi(u) \in \ZZ^d\,.
\end{align*} 
Clearly, if $u$ is divisible by $\beta$, then $v=u/\beta= \pi^{-1}(S_\beta^{-1} \cdot \pi(u))\in\Zomega$.
\end{proof}























  
   

\chapter{Design of extending window method}
\label{chap:methodDescription}

\section{Basic formula for digit set conversion}
The general concept of addition (standard or parallel) in any numeration system $(\beta,\A)$, such that $\fin{\A}$ is closed under addition, is following: we add numbers digitwise and then we convert the result into the alphabet $\A$. Obviously, digitwise addition is computable in parallel, thus the crucial point is the conversion of the obtained result. It can be easily done in a standard way but a parallel digit set conversion is nontrivial. However, formulas are basically same but the choice of coefficients differs.

Now we go step by step more precisely. Let $x=\sum_{-m'}^{n'} x_i\beta^i,y=\sum_{-m'}^{n'} y_i\beta^i \in \fin{\A}$ with $(\beta,\A)$-representantions padded by zeros to have the same length. We set 
  \begin{align*}
    w&=x+y =\sum_{-m'}^{n'} x_i\beta^i + \sum_{-m'}^{n'} y_i\beta^i = \sum_{-m'}^{n'} (x_i+y_i)\beta^i \\
    &=\sum_{-m'}^{n'} w_i\beta^i \,,
  \end{align*}
  where $w_j=x_j+y_j \in \A +\A$. Thus, $w_{n'} w_{{n'}-1}\cdots w_1 w_0 \bullet w_{-1} w_{-2} \cdots w_{-m'}$ is a  $(\beta, \A+\A)$-representation of $w\in \fin{\A+\A}$. 

We also use column notation of addition in the following, e.g.,     
  \begin{align*}
  x_{n'} \;x_{{n'}-1}\cdots x_1 \;x_0 &\bullet x_{-1} \;x_{-2}\, \cdots x_{-m'} \\[-3pt]
  y_{n'} \;y_{{n'}-1}\cdots y_1 \,\;y_0 &\bullet y_{-1} \;y_{-2} \;\cdots y_{-m'} \\[-7pt]
    \line(1,0){90} & \line(1,0){100} \\[-7pt]
  w_{n'} w_{{n'}-1}\cdots w_1 w_0 &\bullet w_{-1} w_{-2} \cdots w_{-m'}\,.
  \end{align*}
  
As we want to obtain a $(\beta,\A)$-representation of $w$, we search a sequence 
  $$z_{n} z_{n-1}\cdots z_1 z_0 z_{-1} z_{-2} \cdots z_{-m}$$ such that $z_j \in \A$ and
  $$
    z_{n} z_{n-1}\cdots z_1 z_0 \bullet z_{-1} z_{-2} \cdots z_{-m}=(w)_{\beta,\A}\,.
  $$
  In the next, we consider without lost of generality only $\beta$-integers since modification for representations with rational part is obvious:
  $$
  \beta^m \cdot z_{n} z_{n-1}\cdots z_1 z_0 \bullet z_{-1} z_{-2} \cdots z_{-m} = z_{n} z_{n-1}\cdots z_1 z_0 z_{-1} z_{-2} \cdots z_{-m} \bullet
  $$  
  Particulary, let $(w)_{\beta, \A+\A}=w_{n'} w_{{n'}-1}\cdots w_1 w_0 \bullet$. We search $n \in \NN$ and $z_{n}, z_{n-1},\dots, z_1, z_0 \in \A$ such that $(w)_{\beta, \A}=z_{n} z_{n-1}\cdots z_1 z_0 \bullet$.   
  
  We use suitable representation of zero to convert digits $w_j$ into the alphabet $\A$. 
  \begin{defn}
  For a base $\beta \in \Zomega$, a polynomial $R(x)=r_s x^s+r_{s-1}x^{s-1}+ \dots + r_1 x+r_0$ with coefficients $r_i \in \Zomega,$ such that $R(\beta)=0$, is called a \emph{rewriting rule}. MA TAM BYT Z[OMEGA] NEBO Z[BETA]???
  \end{defn}  
  
  An arbitrary rewriting rule may be used for conversion, then so called \emph{core coefficient}, i.e., one of the coefficients which is greatest in modulus, is applied to convert a digit $w_j$. For our purpose, we use the simplest possible rewriting rule
  $$
    R(x)=x-\beta \in \Zomega[x]\,.
  $$
As $0=\beta^{j} \cdot R(\beta)=1\cdot \beta^{j+1} -\beta \cdot \beta^{j}$, we have a representation of zero 
$$1 (-\!\beta) \underbrace{0 \cdots 0}_{j}\bullet = (0)_\beta\,. $$
for all $j \in \NN$. We multiply this representation by $q_j \in \Zomega$ which is called a \emph{weight coefficient} to obtain representation of zero 
$$q_j (-q_j\beta) \underbrace{0 \cdots 0}_{j}\bullet = (0)_\beta\,. $$ 
This is digitwise added to $w_{n} w_{n-1}\cdots w_1 w_0 \bullet$ to convert the digit $w_j$ into the alphabet $\A$. It causes a \emph{carry} $q_{j}$ on the $(j+1)$-th position from the conversion of $j$-th digit. The conversion runs from right ($j=0$) to left until all digits and carries are converted into the alphabet $\A$:
        \begin{align*}
            \hspace{100pt}  w_n w_{n-1}&&&\cdots& &w_{j+1}&\!\! &\textcolor{red}{w_j}  & \!\!  &w_{j-1} &&\cdots &&w_1 w_0\bullet \hspace{100pt} \\[-5pt]
                         &&&&       &       & &     &   &q_{j-2} &&\iddots\\[-3pt] 
                         &&&&       &       & &\textcolor{red}{q_{j-1}}& -&\beta q_{j-1} \\[-3pt]
                         &&&&         &q_j&   \textcolor{red}{-}&\textcolor{red}{\beta q_j} &&\\[-8pt]
                         &&&  \iddots      &   -&\beta q_{j+1}&   &\ &&\\[-17pt]
          \intertext{\hspace{60pt}\line(1,0){300}}
          \vspace{-15pt}
          \\[-30pt]
           z_{n+s} \cdots z_{n} z_{n-1}&&&\cdots& &z_{j+1}& &\textcolor{red}{z_j}& &z_{j-1} &&\cdots &&z_1 \; z_0\bullet                  
        \end{align*}
    Hence, the desired formula for conversion on the $j$-th position is 
    \begin{equation*}
        z_j=w_j + q_{j-1} - q_j \beta
    \end{equation*}
    for $j \in \NN_0$. We set $q_{-1}=0$ as there is no carry from the right on the 0-th position.
    
     Clearly, the value of $w$ is preserved:
\begin{align*}
    \sum_{j\geq 0} z_j \beta^j &=w_0 - \beta q_0 + \sum_{j> 0} (w_j + q_{j-1} - q_j \beta) \beta^j \\
    &=\sum_{j\geq 0} w_j \beta^j + \sum_{j>0} q_{j-1} \beta^j - \sum_{j\geq 0} q_j \cdot \beta^{j+1} \\
    &=\sum_{j\geq 0} w_j \beta^j + \sum_{j>0} q_{j-1} \beta^j - \sum_{j> 0} q_{j-1} \cdot \beta^j \\
    &=\sum_{j\geq 0} w_j \beta^j = w\,.
\end{align*}

     The weight coefficient $q_j$ must be chosen such that the converted digit is in the alphabet~$\A$, i.e., 
    \begin{equation}
    \label{eq:conversionFormula}
        z_j=w_j + q_{j-1} - q_j \beta \in \A\,.
    \end{equation}
    Choice of weight coefficients is the crucial part in the case of construction of parallel addition algorithms. The method determining weight coefficients for a given input is decribed in Section~\ref{sec:methodDescription}.
    
     On the other hand, it is trivial for standard numeration systems.  Notice that
    $$
        z_j \equiv w_j+q_{j-1} \mod \beta\,. 
    $$
  Assume now a standard numeration system $(\beta, \A)$, where
  $$
    \beta \in \NN\,,\beta  \geq 2\,, \A = \{0, 1, 2,\dots, \beta -1 \}\,.
  $$ 
  There is only one representative of each class modulo  $\beta$. Therefore, the digit $z_j$ is uniquely determined for a given digit $w_j \in \A+\A$ and carry $q_{j-1}$ and thus so is the weight coefficient $q_j$. This means that $q_j=q_j(w_j,q_{j-1})$ for all $j\geq 0$. Generally,
  $$
  q_j=q_j(w_j,q_{j-1}(w_{j-1},q_{j-2}))=\dots =q_j(w_j ,\dots , w_1, w_0)
  $$
  and
  $$
  z_j=z_j(w_j ,\dots , w_1, w_0)\,,
  $$
  which implies that addition runs in linear time.
  
  We want to the digit set conversion from $\A+\A$ into $\A$ be computable in parallel, i.e., there exist constants $r,t \in \NN_0$ such that for all $j\geq 0$ is $z_j=z_j(w_{j+r},\dots,w_{j-t})$. To avoid the dependency on all less, respectivelly more, significant digits, we need variety in the choice of weight coefficient $q_j$. This implies that the used numeration system must be redundant.
  
  
  
  NEJAKE LEMMA O TOM, ZE NEREDUNDANTNI SYSTEM VEDE NA LINEARNI CAS???


\section{Method}
\label{sec:methodDescription}
In order to construct a parallel digit set conversion in numeration system $(\beta,\A)$ we consider more general case of conversion from an \emph{input alphabet} $\B$ such that $\A \subsetneq \B \subset \A +\A$ instead of the alphabet $\A+\A$.
As menshioned above, the key problem is to find for every $j\geq 0$ a weight coefficient $q_j$ such that 
    $$
        z_j=\underbrace{w_j}_{\in \B} + q_{j-1} - q_j \beta \in \A 
    $$  
    for any input $w\in \fin{\B}, (w)_{\beta,\B}=w_{n'}w_{n'-1}\dots w_1 w_0 \bullet$. We remark that the weight coefficient $q_{j-1}$ is determined by the input $w$. For digit set conversion to be computable in parallel we demand to digit $z_j=z_j(w_{j+r},\dots,w_{j-t})$ for a fixed anticipation $r$ and memory $t$ in $\NN_0$.
    
    We introduce following definitions to obtain the desired digit set conversion. 
    \begin{defn}
        Let $\B$ be a set such that $\A \subsetneq \B \subset \A +\A$. Then any set $\Q\subset\Zomega$ containing~0 such that 
        $$
            \B + \Q \subset \A + \beta \Q
        $$  
        is called a \emph{set of weight coefficients}.
    \end{defn}
    We see that
        $$
        (\forall w_j \in \B)(\forall q_{j-1}\in\Q)(\exists q_j \in \Q )(w_j + q_{j-1} - q_j \beta \in \A )\,.
        $$
    Accordingly, there is a weight coefficient $q_j \in \Q$ for any carry from the right $q_{j-1}\in \Q$ and any digit $w_j$ in the input alphabet $\B$. I.e., we  satisfy basic conversion formula (\ref{eq:conversionFormula}). Notice that $q_{-1}=0$ is in $\Q$ by definition. Thus, all weight coeficients may be chosen from $\Q$ inductively.
    \begin{defn}
    Let $M$ be an integer and $q:\B^{M} \rightarrow \Q$ be a mapping such that 
    $$
    w_j+ q(w_{j-1}, \dots, w_{j-M}) - \beta q(w_j, \dots, w_{j-M+1}) \in \A
    $$
    for all $w_j,w_{j-1}, \dots, w_{j-M} \in \B$. Then $q$ is called a \emph{weight function} and $M$ is called a \emph{length of window}.    
    \end{defn}
 
 JE TREBA POZADOVAT q(0,...,0)=0 NEBO TO Z NECEHO PLYNE?

 Having a weight function $q$, we define a function $\phi:\B^{M+1}\rightarrow \A$ by
    $$
        \phi(w_{j}, \dots, w_{j-M})=w_j+ \underbrace{q(w_{j-1}, \dots, w_{j-M})}_{=q_{j-1}} - \beta \underbrace{q(w_j, \dots, w_{j-M+1})}_{=q_j}=:z_j\,,
    $$ 
    which verifies that the conversion is indeed $(M+1)$-local function with anticipation $r=0$ and memory $t=M$.
    
The construction of a parallel conversion algorithm by so-called \emph{extending window method} consists of two phases. In the first one, we find a minimal possible weight coefficient set $\Q$. It serves as the starting point for the second phase in which e increment the expected length of the window $M$ until the weight function $q$ is uniquely defined for each $(w_j,w_{j-1}, \dots , w_{j-M+1}) \in \B^{M}$.

\subsection{Phase 1 -- Weight coefficient set}
The goal of the first phase is to compute a weight coefficient set $\Q$, i.e., to find a set $\Q \ni 0$ such that 
$$
    \B + \Q \subset \A + \beta \Q\,.
$$  
We build $\Q$ iteratively so that we extend $\Q$ in a way to cover all elements on the left-hand side with original $\Q$ by elements on the right-hand side with extended $\Q$. This procedure is repeated until the extended weight coefficient set is the same as original one. 

In other words, we start with $\Q=\{0\}$ meaning that we search all weight coefficients necessary for conversion for the case where there is no carry from the right. We add them to weight coeffcient set. These weight coefficients now may appear as a carry. If there are not suitable weight coefficients  in weight coefficient set to satisfy formula (\ref{eq:conversionFormula}) for all combinations of added coefficients and digits of input alphabet, we extend $\Q$ by appropriate ones. And so on until there is no need to add more elements.
    
The precise description of the semi-algorithm in a pseudocode is following: 
    
\begin{algorithm}
  \caption{Search for weight coefficient set (Phase 1)}
    \label{alg:weightCoefSet}
  \begin{algorithmic}[1]
    \STATE $k:=0$ 
    \STATE $Q_0:=\{0\}$
    \REPEAT
     \STATE  Extend $\Q_k$ to $\Q_{k+1}$ in a minimal possible way so that $$\B+ \Q_k \subset \A + \beta \Q_{k+1}$$
     \vspace{-20pt}
      \STATE  $k:=k+1$
      \UNTIL{$\Q_k = \Q_{k+1}$}      
      \STATE $\Q:=\Q_k$
    \RETURN $\Q$
  \end{algorithmic}
\end{algorithm}

\begin{algorithm}
  \caption{Extending intermediate weight coefficient set}
    \label{alg:extendWeightCoefSet}
  \begin{algorithmic}[1]
    \REQUIRE{\verb+candidates+, previous weight coefficient set $\Q_{k-1}$}
    \STATE $\Q_k:=\Q_{k-1}$
    \FORALL{\texttt{cand\_for\_x} in \texttt{candidates}}
        \IF{no element of \texttt{cand\_for\_x} in $\Q_{k-1}$}
            \STATE Add the smallest element (in absolute value) of  \verb+cand_for_x+ to $\Q_{k}$  
        \ENDIF
    \ENDFOR
    \RETURN $\Q_k$
  \end{algorithmic}
\end{algorithm}
    

\begin{algorithm}
  \caption{Search for candidates}
    \label{alg:searchCand}
  \begin{algorithmic}[1]
    \REQUIRE{previous weight coefficient set $\Q_{k-1}$}
    \STATE \verb+candidates+:=[ ]
    \FORALL{$x \in \B + \Q_{k}$}
      \STATE \verb+cand_for_x+:=[ ]
      \FORALL{$a \in \A$}
          \IF{$(x-a)$ is divisible by $\beta$ in $\Zomega$ NEBO ZBETA???}
              \STATE Append $\frac{x-a}{\beta}$ to \verb+cand_for_x+
            \ENDIF
      \ENDFOR 
      \STATE Append \verb+cand_for_x+ to \verb+candidates+
  \ENDFOR
  \RETURN \verb+candidates+
  \end{algorithmic}
\end{algorithm}    
    
    





\subsection{Phase 2 -- Weight function}

    {Phase 2 -- searching for a weight function}

    We want to find a length of the window $M$ and a weight function $q:(\A+\A)^{M} \to \Q$.% have $q_j=q(w_j,w_{j-1},\dots, w_{j-M+1})$, where $q$ is a weight function:
%     \begin{align*}
%         \cdots\; &w_{j+1}&\!\! &\textcolor{red}{w_j}  & \!\!  &\textcolor{red}{w_{j-1}\cdots w_{j-M+1}} w_{j-M} \cdots\\[-3pt]
%                          & &       &q_{j-1}& -&\beta q_{j-1} \\[-1pt]
%                            &q_j&   -&\beta \textcolor{red}{q_j} &&\\[-19pt]      
%     \intertext{\line(1,0){280}}
%     \\[-30pt]
%      \cdots\; &z_{j+1}& &z_j& &z_{j-1} \cdots z_{j-M+1}\; z_{j-M}\cdots                     
%     \end{align*}
    
    
    Suppose that the length of the window is $m$.
    
    The idea is to check all possible right carries $q_{j-1}$ and determine values $q_j$ such that 
    $$
    z_j=w_j + q_{j-1} - q_j \beta \in \A \,.
    $$
    The set of all such needed values of $q_j$ is denoted by $\Q_{[w_{j},\dots, w_{j-m+1}]}\subset \Q$
        
    
    The length $M$ and weight function $q$ is found when 
    $$
    \#\Q_{[w_{j},\dots, w_{j-M+1}]}=1
    $$
    for all $w_{j},\dots, w_{j-M+1} \in (\A+\A)^M$.
%     \begin{equation*}
%     w_j + \underbrace{q(w_{j-1},w_{j-2},\dots, w_{j-M})}_{\in \Q_{[w_{j-1},w_{j-2},\dots, w_{j-M}]}  \subset \Q }=z_j + \beta \cdot \underbrace{q(w_j,w_{j-1},\dots, w_{j-M+1})}_{\in \Q_{[w_j,w_{j-1},\dots, w_{j-M+1}]} \subset \Q}
%     \end{equation*}
    



    {Phase 2}
      $m:=1$
      
      For each $w_j \in \A+\A$ find minimal set $\Q_{[w_j]} \subset \Q$ such that
      $$
      w_j + \Q \subset \A + \beta \Q_{[w_j]}
      $$
    
      While $(\max\{\#\Q_{[w_j,\dots, w_{j-m+1}]}:(w_j,\dots, w_{j-m+1}) \in (\A+\A)^m \} > 1)$ do:
      \begin{itemize}
          
          \item $m:= m +1$
          
          \item For each $(w_j,\dots, w_{j-m+1}) \in (\A+\A)^{m}$ find minimal set $\textcolor{red}{\Q_{[w_j,\dots, w_{j-m+1}]}} \subset \Q_{[w_j,\dots, w_{j-m+2}]}$ such that
          $$
          w_j + \Q_{[w_{j-1},\dots, w_{j-m+1}]} \subset \A + \beta \textcolor{red}{\Q_{[w_j,\dots, w_{j-m+1}]}}\,,
          $$
      \end{itemize}
      
      $M:= m$ 
      
      
      $q(w_j,\dots, w_{j-M+1}):=$ only element of $\Q_{[w_j,\dots, w_{j-M+1}]}$
    



        Now we have parallel conversion algorithm:
    \begin{align*}
    z_j&=w_j + q_{j-1} - q_j \beta = \\
       &=w_j + q(w_{j-1},w_{j-2},\dots, w_{j-M}) - \beta q(w_j,w_{j-1},\dots, w_{j-M+1}) = \\
       &= z_j(w_{j},w_{j-1},\dots, w_{j-M})\,.
    \end{align*}







\chapter{Convergence}
\label{chap:convergence}

Unfortunately, the extending window method does not always converge. The algorithm may lead to an infinite loop in both phases. 
In this chapter, we introduce a sufficient condition for convergence of Phase 1 in Theorem \ref{thm:suffCondPhase1} and we categorize the algebraic number $\omega$ according to this condition. Theorem \ref{thm:stoppingCondition} enables us to construct an algorithm which checks the necessary condition for convergence of Phase 2.

% However,  and Theorem \ref{thm:stoppingCondition} enables us to construct Algorithm \ref{alg:oneletterSets} which checks the necessary condition for convergence of Phase 2.  

\section{Convergence of Phase 1}
\label{sec:convergencePhase1}
The following theorem gives the condition on $\omega$ which is sufficient for convergence of the Phase 1.
\begin{theo}
\label{thm:suffCondPhase1}
    Let $\omega$ be an algebraic integer. Let $\A$ and $\B$ be finite subsets of $\Zomega$ such that $\A$ contains at least one representative of each congruence class modulo $\beta$ in $\Zomega$. Then, there exists a set $\Q\subset\Zomega$ such that $ \B + \Q \subset \A + \beta \Q$ and all elements of $\Q$ are limited by constant $R\in \RR^+$ in modulus.
    
    Moreover, if $\omega$ is such that any complex circle centered at 0 contains only finitely  many elements of $\Zomega$, the set $\Q$ is finite. 
\end{theo}
\begin{proof}
 Denote $A:=\max\{|a|\colon a \in \A\}$ and $B:=\max\{|b|\colon b\in\B\}$. Consequently, set $R:=\frac{A+B}{|\beta|-1}$ and $\Q:=\{q\in\Zomega\colon |q|\leq R\}$. Since $A>0$ and $|\beta|>1$, the set $\Q$ is not empty. Any element $x=b+q \in \Zomega$ with $b\in\B$ and $q\in\Q$ can be written as $x=a+\beta q'$ for some $a\in\A$ due to existence of representative of each congruence class in $\A$. We prove that $|q'|\leq R$:
 $$
    |q'|=\frac{|b+q-a|}{|\beta|}\leq \frac{B+R+A}{|\beta|} \leq \frac{1}{|\beta|}\left(A+B+\frac{A+B}{|\beta|-1}\right)  =\frac{A+B}{|\beta|}\left(\frac{|\beta|}{|\beta|-1}\right)=R\,.
 $$ 
 Hence $q'\in\Q$ and thus  $x=b+q \in \A + \beta \Q$. 
 
 Obviously, the set $\Q$ is finite if there are only finitely many elements of $\Zomega$ bounded by the constant $R$.
\end{proof}
We plug in the alphabet $\A$ and input alphabet $\B$. Because of adding of the smallest elements in Algorithm \ref{alg:extendWeightCoefSet}, we see from the proof of Theorem \ref{thm:suffCondPhase1} that the weight coefficient set~$\Q$ constructed in Phase~1 has elements bounded by the constant $R$. 

Therefore, the intermediate weight coefficient sets $\Q_k$ in Algorithm \ref{alg:weightCoefSet} have elements bounded by $R$ for all $k\in\NN$. Hence, Phase 1 succesfully ends if there are only finitely many elements in $\Zomega$ bounded by the constant $R$. 


The following two lemmas characterizes the algebraic integer $\omega$ according to the number of elements of $\Zomega$ in a complex circle centered around 0. 
\begin{lem}
\label{lem:numElemRR}
Let $\omega\in\RR$ be an algebraic integer and $R$ be a positive constant. There are only finitely many elements in the set $\left\{x \in\Zomega \colon |x|<R \right\}$ if and only if $\omega \in \ZZ$. 
\end{lem}
\begin{proof}
If $\omega \in \ZZ$, then $\Zomega = \ZZ$. Trivially, there are only finitely many integers bounded by any constant $R$.

Suppose now that $\omega \notin \ZZ$ and we show that 
$$
(\forall R>0) (\exists_\infty x\in \Zomega)(|x|<R)\,.
$$ 

The degree of $\omega$ is at least two as the only algebraic integers of degree one are integers, i.e. $\omega \in \RR \setminus \QQ$. 
Set $x:= \omega- \lfloor \omega \rfloor$. We see that $0<x<1$ as $\omega \notin \ZZ$ and $x \in \Zomega$ as $\omega \in \Zomega$, $\lfloor \omega \rfloor \in  \ZZ\subset\Zomega$ and $\Zomega$ is a ring. 
Hence, the sequence $(x^n)_{n\in \NN}$ is strictly decreasing and its limit is 0 which implies the claim.
\end{proof}


\begin{lem}
\label{lem:numElemCC}
Let $\omega\in \CC\setminus\RR$ be an algebraic integer and $R$ be a positive constant. There are only finitely many elements in the set $\left\{x \in\Zomega \colon |x|<R\right\}$ if and only if the degree of $\omega$ is two.  
\end{lem}
\begin{proof}
If the degree of $\omega$ is two, the set $\Zomega$ is generated by integer combinations of 1 and $\omega$, i.e. it is the lattice in $\CC$. Thus the number of elements of the set $\left\{x \in\Zomega \colon |x|<R\right\}$ is finite.

Suppose now that the degree of $\omega$ is at least three as there are no complex algebraic integers with the degree one. 


We recall that for all $r,s\in\RR\setminus\{0\}, |r|\leq|s|$, there exists $k\in\ZZ\setminus\{0\}$ such that $|k\cdot r -s|\leq |r|/2$. It follows from the fact that $l\cdot |r| \leq |s| <(l+1)|r| $ for some $l\in\ZZ\setminus\{0\}$. We choose $k$ from $\{\pm l,\pm(l+1)\}$ accordingly. 
 
Now, if $|\Im \omega|\leq|\Im \omega^2|$, set $z_0:=\omega$ and $w:=\omega^2$. Otherwise, set $z_0:=\omega^2$ and $w:=\omega$.   

We build the sequence $(z_i)_{i\in\NN}$ recurrently:
$$
z_{i+1}:=k_{i+1} \cdot z_{i}-w\, 
$$
where $k_{i+1}\in\ZZ$ is such that 
$$
|\Im z_{i+1}|=|k_{i+1} \cdot\Im z_{i}-\Im w|\leq \frac{|\Im z_{i}|}{2}\,.
$$
We can find such number $k_{i+1}$ under assumption that $\Im z_{i} \neq 0$.

We prove by induction that $|\Im z_i|\leq |\Im z_0|/2^i$. Clearly, it holds for $i=0$. Assume now that it holds for $i$. Then 
$$
|\Im z_{i+1}|= |k_{i+1} \cdot \Im z_{i}- \Im w|\leq \frac{1}{2}|\Im z_i|\leq \frac{|\Im z_0|}{2^{i+1}}\,. 
$$
Hence, $k_i\neq 0$ for all $i\in\NN$ as $|\Im z_i|\leq |\Im z_0|/2^i \leq |\Im w|$.

Next, we show that
$$
z_i=z_0 \cdot \prod_{j=1}^i k_j- l_i\cdot w
$$
for some $l_i\in\ZZ$. 

Obviously, $z_0=z_0 \prod_{j=1}^0 k_j - 0\cdot w$. Assume now that it holds for $i$ and consider
$$
z_{i+1}=k_{i+1} \cdot z_{i}-w= k_{i+1}\left(z_0 \cdot \prod_{j=1}^i k_j- l_i\cdot w\right) -w=z_0 \cdot \prod_{j=1}^{i+1} k_j - (\underbrace{k_{i+1}\cdot l_i +1}_{=:l_{i+1}})w\,.
$$
Thus, $z_i\in\Zomega$. Moreover, $z_i\notin \ZZ$. Assume in contrary that 
$$
z_0 \cdot \prod_{j=1}^i \underbrace{k_j}_{\neq0}- l_i\cdot w = z_i\in \ZZ\,.
$$
Since $z_0,w\in\{\omega, \omega^2\}, z_0\neq w$, we have the integer polynomial of degree two with zero $\omega$. It contradicts that the degree of $\omega$ is at least three.
% Then $z_{i+1}\in\Zomega$ as $z_{i+1}=k_{i+1} \cdot z_{i}-w$ which is an integer combination of numbers from $\Zomega$. Also,



Now we take $i_0\in\NN$ such that $|\Im z_0|/2^{i_0}< 1/2$. Let $k\in\ZZ$ be such that $|\Re z_{i_0} -k|\leq 1/2$. Set $x:= z_{i_0}-k$. Then
$$
|x|\leq |\Re x| +|\Im x| \leq  |\Re z_{i_0} -k| + |\Im z_{i_0}| \leq \frac{1}{2} + \frac{|\Im z_0|}{2^{i_0}} < \frac{1}{2} +\frac{1}{2}=1\,.
$$
We know that $\Im z_{i_0} \neq 0$ and thus $0<|x|<1$.

Consider now the case that we cannot produce $z_{i_1+1}$  in the sequence $(z_i)_{i\in\NN}$ as there is ${i_1}$ such that $\Im z_{i_1} = 0$, i.e. $z_{i_1}\in\RR$. In the same manner as before, we may prove that  $z_{i_1} \in \Zomega\setminus \ZZ$. Set $x:= z_{i_1}- \lfloor z_{i_1} \rfloor$. Obviously, $0<|x|<1$ and $x\in\Zomega$.

For both cases, the sequence $(x^n)_{n\in\NN}$ has the limit 0 and $0\neq x^n\in\Zomega$ for all $n\in\NN$. Thus there are infinitely many elements of $\Zomega$ bounded by any positive constant $R$. 
\end{proof}


Using Theorem \ref{thm:suffCondPhase1} and Lemma \ref{lem:numElemRR} and \ref{lem:numElemCC}, we categorize an algebraic integer $\omega$ which generates $\Zomega \ni \beta$ as follows:

\begin{itemize}
    \item $\omega \in \ZZ \implies$ Phase~1 converges.
    \item $\omega \in \RR\setminus\ZZ$ -- the sufficient condition does not hold and we we have an example for which Phase 1 does not converge (see Section \ref{sec:example_nonoconvergence_phase1}).
    \item $\omega \in \CC\setminus\RR$, $\omega$ being quadratic algebraic integer $\implies$ Phase~1 converges.
    \item $\omega \in \CC\setminus\RR$, $\omega$ being algebraic integer of degree $\geq 3$ -- the sufficient condition does not hold  and therefore the convergence of Phase~1 is not guaranteed.
\end{itemize}

\section{Convergence of Phase 2}
\label{sec:convergencePhase2}
For shorter notation, set 
$$
\Qb{m}:=\Q_{[\underbrace{\scriptstyle b,\dots,b}_m]}
$$ for $m \in \NN$ and $b\in\B$.

Obviously, finitness of Phase~2 implies that there exists a length of window $M$ such that the set $\Qb{m}$ contains only one element for all $b\in\B$. 
The following theorem is used for the construction of algorithm which checks this necessary condition. 

\begin{theo}
\label{thm:stoppingCondition}
Let $m_0 \in \NN$ and $b\in\B$ be such that sets $\Qb{m_0}$ and $\Qb{m_0-1}$ produced by Algorithm \ref{alg:minimalSet} within Phase~2 have the same size. Then
$$
    \#\Qb{m} = \#\Qb{m_0} \qquad \forall m\geq m_0-1\,.
$$ 
Particularly, if $\#\Qb{m_0}\geq 2$, Phase~2 does not converge.
\end{theo}
\begin{proof}
We prove the base case of induction with respect to $m$. For $m=m_0+1$, the set $\Qb{m_0+1}$ is found by Algorithm \ref{alg:minimalSet} such that 
$$
b + \Qb{m_0} \subset \A + \beta \Qb{m_0+1}\,
$$
and set $\Qb{m_0}$ is found by the same algorithm such that
$$
b + \Qb{m_0-1} \subset \A + \beta \Qb{m_0}\,.
$$
As $\Qb{m_0} \subset \Qb{m_0-1}$, the assumption of the same size implies
$$
    \Qb{m_0} = \Qb{m_0-1}\,.
$$
It means that Algorithm \ref{alg:minimalSet} runs with the same input and hence
$$
\Qb{m_0+1}=\Qb{m_0}\,.
$$
The inductive step of the proof for $m+1$ is analogous to the base case.

Phase 2 ends when there is only one element in $\Q_{[w_j,\dots, w_{j-m+1}]}$ for all $(w_j,\dots, w_{j-m+1}) \in \B^m$ for some fixed length of window $m$. But if $\#\Qb{m_0}\geq 2$, size of $\Q_{[b,\dots,b]}$ does not decrease despite of extending the length of window.
\end{proof}



\begin{algorithm}
  \caption{Check input $bb\dots b$}
    \label{alg:oneletterSets}
  \begin{algorithmic}[1]
    \REQUIRE{Weight coefficient set $\Q$, digit $b\in\B$}
    \STATE $m:=1$
    \STATE Find minimal set $\Qb{1} \subset \Q$ such that
      $$
      b + \Q \subset \A + \beta \Qb{1}\,.
      $$
      \vspace{-20pt}
    \WHILE{$\#\Qb{m} > 1$}
        \STATE $m:= m +1$
        \STATE By Algorithm \ref{alg:minimalSet}, find minimal set $\Qb{m} \subset \Qb{m-1}$ such that
          $$
          b + \Qb{m-1} \subset \A + \beta \Qb{m}\,.
          $$  
          \vspace{-20pt}
        \IF{$\#\Qb{m}=\#\Qb{m-1}$}
            \RETURN Phase 2 does not converge for input $bb\dots b$.
        \ENDIF
    \ENDWHILE  
    \RETURN Weight coefficient for input $bb\dots b$ is the only element of $\Qb{m}$.
  \end{algorithmic}
\end{algorithm}

For arbitrary $m$, sets $\Qb{m}$  can be easily constructed separately for each $b\in\B$. Using Theorem \ref{thm:stoppingCondition}, Algorithm \ref{alg:oneletterSets} checks whether Phase~2 stops processing input digits $bb\dots b$. Thus,  non-finiteness of Phase~2 can be revealed by running it for each input digit $b\in\B$.


\chapter{Design and implementation}
\label{chap:implementation}
The designed method requires to compute arithmetic operations in $\Zomega$. Therefore, we have chosen Python-based programming language SageMath for the implementation of the extending window method as it contains many ready-to-use mathematical structure.  Using SageMath is very convenient as it also offers easily usable data structures or tools for plotting. Thus the code is more readable and we may focus on the algorithmic part of problem. On the other hand, SageMath is considerably slower than for example C++ or other low-level languages. Nevertheless, it is sufficient for our purpose.

The implementation is object-oriented. It consists of four classes. Class \emph{AlgorithmForParallelAddition} contains structures for computations in $\Zomega$. Specifically, we use the provided class \emph{PolynomialQuotientRing} to represent elements of $\Zomega$ and  \emph{NumberField} for obtaining numerical complex value of them. The class also links necessary instances and functions to construct algorithm for parallel addition by the extending window method for an algebraic integer $\omega$ given by its minimal polynomial $p$ and approximate complex value, a base $\beta\in\Zomega$, an alphabet $\A\subset\Zomega$ and input alphabet $\B$. Phase 1 of the extending window method is implemented in class \emph{WeightCoefficientsSetSearch} and Phase 2 in \emph{WeightFunctionSearch}. Class \emph{WeightFunction} holds the weight function $q$ computed in Phase 2. All classes are described in the following sections including lists of the important methods  with a short description. Sometimes, the notation from Chapter \ref{chap:methodDescription} is used for better understanding. For all implemented methods, see commented source code.  

Basically, weight function can be found just by creating an instance of \emph{AlgorithmForParallelAddition} and calling \textbf{findWeightFunction()}. For more comfortable usage, our implementation includes two interfaces -- a shell version and graphic user interface using interact in SageMath Cloud. The whole implementation is on the attached DVD or it can be downloaded from  \url{https://github.com/Legersky/ParallelAddition}.

\section{Class AlgorithmForParallelAddition}
Constructor:

\begin{method}{\_\_init\_\_}{minPol\_str, embd, alphabet, base, name='NumerationSystem', inputAlphabet='', printLog=True, printLogLatex=False, verbose=0}
Take \var{minPol\_str} which is symbolic expression in the variable $x$ of minimal polynomial $p$. The closest root of  \var{minPol\_str} to \var{embd} is used as the ring generator $\omega$. The structures for $\Zomega$ are constructed as described above. Setters \fun{setAlphabet}{alphabet}, \fun{setInputAlphabet}{A} and \fun{setBase}{base} are called. Messages saved to logfile during existence of an instance are printed (using \LaTeX) on standard output depending on \var{printLog} and \var{printLogLatex}. The level of messages for a development is set by \var{verbose}. 
\end{method}

Setters and getters:

\begin{method}{setAlphabet}{A}
Check if \var{A} is subset of $\Zomega$. Set alphabet $\A$:=\var{A}
\end{method}

\begin{method}{setInputAlphabet}{B}
If \var{B} is empty, $\A+\A$ is used. Set the input alphabet $\B$:=\var{B}. Check if $\A\subsetneq\B\subset\A+\A$. 
\end{method}


-----------------------------EXTENDING WINDOW METHOD------------------------------------------------------------------------

\begin{method}{\_findWeightCoefSet}{ max\_iterations, method\_number}
Create an instance of \emph{WeightCoefficientsSetSearch(method\_number)} and call its method \fun{findWeightCoefficientsSet}{max\_iterations} to obtain a weight coefficients set $\Q$.
\end{method}

\begin{method}{addWeightCoefSetIncrement}{ increment}
Save \var{increment} from extending intermediate weight coefficients set $\Q_{k}$ to $\Q_{k+1}$.
\end{method}

\begin{method}{\_findWeightFunction}{ max\_input\_length,method\_number}
Create an instance of \emph{WeightFunctionSearch(method\_number)} and call its methods \fun{check\_one\_letter\_inputs}{max\_input\_length} and \fun{findWeightFunction}{max\_input\_length} to obtain a weight function $q$.
\end{method}


\begin{method}{findWeightFunction}{ max\_iterations, max\_input\_length, method\_weightCoefSet=2, method\_weightFunSearch=4}
Return the weight function $q$ obtaind by calling \fun{\_findWeightCoefSet}{max\_iterations,method\_weightCoefSet} and \fun{\_findWeightFunction}{max\_input\_length, method\_weightFunSearch}.
\end{method}


-----------------------------PARALLEL ADDITION AND CONVERSION---------------------------------------------------------------

\begin{method}{addParallel}{a,b}
Sum up numbers represented by lists of digits \var{a} and \var{b} digitwise and convert the result by \fun{parallelConversion}{}. 
\end{method}


\begin{method}{parallelConversion}{\_w}
Return $(\beta,\A)$-representation of number represented by list \var{\_w} of digits in input alphabet $\B$. It is computed locally according to the equation \ref{eq:conversionFormula} and using weight function $q$.
\end{method}


\begin{method}{localConversion}{w}

\end{method}


-----------------------------SANITY CHECK-----------------------------------------------------------------------------------

\begin{method}{sanityCheck\_conversion}{ num\_digits}

\end{method}


-----------------------------RING CONVERSIONS, AUXILIARY RING FUNCTIONS-----------------------------------------------------

\begin{method}{\_computeInverseBaseCompanionMatrix}{}

\end{method}


\begin{method}{divideByBase}{divided\_number}

\end{method}




\section{Class WeightCoefficientsSetSearch}
-----------------------------CONSTRUCTOR, SETTER-------------------------------------------------------------------

\begin{method}{\_\_init\_\_}{ algForParallelAdd, method}

\end{method}


\begin{method}{\_\_repr\_\_}{}

\end{method}


\begin{method}{setVerbose}{verb}

\end{method}


-----------------------------FIND CANDIDATES-------------------------------------------------------------------

\begin{method}{\_findCandidates}{C}

\end{method}


-----------------------------SEARCH FOR WEIGHT COEFFICIENT SET--------------------------------------------------

\begin{method}{\_getQk}{C}

\end{method}


\begin{method}{\_chooseQk\_FromCandidates}{cand\_for\_all}

\end{method}


\begin{method}{findWeightCoefficientsSet}{ maxIterations}

\end{method}


-----------------------------AUXILIARY FUNCTIONS-------------------------------------------------------------------




\begin{method}{\_findSmallest}{list\_from\_Ring}

\end{method}



\section{Class WeightFunctionSearch}
-----------------------------CONSTRUCTOR, SETTERS-------------------------------------------------------------------

\begin{method}{\_\_init\_\_}{ algForParallelAdd, weightCoefSet, method}

\end{method}

-----------------------------SEARCH FOR WEIGHT FUNCTION-------------------------------------------------------------------

\begin{method}{\_find\_weightCoef\_for\_comb\_B}{ combinations, max\_len}

\end{method}


\begin{method}{\_findQw}{w\_tuple}

\end{method}


\begin{method}{findWeightFunction}{ max\_input\_length}

\end{method}


\begin{method}{check\_one\_letter\_inputs}{ max\_input\_length}

\end{method}



\section{Class WeightFunction}
-----------------------------CONSTRUCTOR, GETTERS-------------------------------------------------------------------

\begin{method}{\_\_init\_\_}{B}

\end{method}


\begin{method}{\_\_repr\_\_}{}

\end{method}


\begin{method}{getMaxLength}{}

\end{method}


\begin{method}{getMapping}{}

\end{method}


-----------------------------ADDING INPUTS, CALL FUNCTION-------------------------------------------------------------------

\begin{method}{\_\_call\_\_}{ input\_tuple}

\end{method}


\begin{method}{getWeightCoef}{ w}

\end{method}


\begin{method}{addWeightCoefToInput}{\_input, coef}

\end{method}


-----------------------------PRINT FUNCTIONS-------------------------------------------------------------------

\begin{method}{printInfo}{}

\end{method}


\begin{method}{printLatexInfo}{}

\end{method}


\begin{method}{printMapping}{}

\end{method}


\begin{method}{printLatexMapping}{}

\end{method}


\begin{method}{printCsvMapping}{}

\end{method}


\section{User interfaces}
We provide two interfaces for running of the implemented extending window method -- the~shell version and graphic user interface.

\subsection{Shell}
SageMath must be installed. The implementation of the extending window method is launched in a shell by typing \verb+sage extending_window_method.sage <input_sample.sage>+. The parameters of the numeration system and setting of outputs and computation is given by the SageMath file \verb+input_sample.sage+. See Appendix \ref{app:inputSample} for an example of such a file.

The name of the numeration system, minimal polynomial of generator $\omega$, an approximate value of $\omega$, the base $\beta$, alphabet $\A$ and input alphabet $\B$ are set in the part INPUTS. The maximum number of iterations in Phase 1, maximal length of the window in Phase 2 and the launching of the sanity check are set in SETTING. 

The boolean values in the part SAVING determines which formats of the outputs are saved. All outputs are saved in the folder \verb+./outputs/<name>/+. General information about the computation can be saved in the TeX format, the computed weight function and local digit set conversion in the CSV file format. An inputs setting saved as a dictionary can be loaded by the interact interface. The log of the whole computation can be saved as a text file. There is also an option to save unsolved combinations in Phase 2 in the CSV file format in the case of the interruption of the program.

According to the boolean values in the part IMAGES, figures of the alphabet, input alphabet, weight coefficients set or part of the set $\Zomega$ with marked alphabet shifted into points which are divisible by the base $\beta$ are saved in the PNG format to the folder \verb+./outputs/<name>/img/+. Optionally, the weight coefficients set is plotted with the  bound given by the proof of Theorem~\ref{thm:suffCondPhase1}. Images of individual steps of both phases of the extending window method can be saved, too. For Phase 2, the search for the weight coefficient  is plotted for the digits given by \verb+phase2_input+.  

The program prints out all inputs and then it computes the weight function $q$ by calling \fun{findWeightFunction}{ max\_iterations, max\_input\_length}. The increments of the weight coefficients set in each iteration of Phase 1 are printed and then also the obtained weight coefficients set $\Q$. The longest tested combinations given by repetition of one letter are printed after the computation of \fun{check\_one\_letter\_inputs}{max\_input\_length}. During computing of Phase 2, the current length of window and the number of saved combinations are printed. At the end, the final length of window, elapsed time and info about saved outputs are printed.  

It is possible to batch process all input files in one folder by executing the bash script \verb+ewm_batch <folder_name>+.  

\subsection{Interact in SageMath Cloud}
The graphic user interface is implemented using an interact in SageMath Cloud. The parts of the interact are on Figure \ref{fig:interact1}, \ref{fig:interact2} and \ref{fig:interact3} in Appendix \ref{app:interact}. The functionality is basically the same as the shell version. An account on the website \url{https://cloud.sagemath.com} is needed to use the interact. Create a new project and load files \verb+extending_window_method_GUI.sagews+ and \verb+interact_ewm.sage+. After executing of the cell by Shift+Enter in the first one, the parameters of the numeration system are filled in the corresponding spaces or one of the previous settings is loaded by typing its name.  By default, the last inputs are shown in the form. The inputs are submitted by pressing the button Update. Using check-boxes, the formats for saving of the output are chosen and the search for the weight function is launched by pressing second button Update.

The printed output is similar to the shell output. In addition, it contains figures and it is formatted using \LaTeX. Moreover, the sanity check can be run for a given length, the weight coefficient for a tuple of  input digits is returned or images of individual steps of both phases are shown and saved.

\chapter{Testing}
\label{chap:testing}




%%%%%%%%%%%%%%%%%%%%%%%%%%%%%%%%%%%%%%%%%%%%%%%%%%%%%%%%%%%%%%%%%%%%%%%%%%%%%%%%%%%%%%%%%

\subsection{Eisenstein base $\beta = -1 + \omega$ with $\omega = \exp{\frac{2 \pi \imath}{3}} = -\frac{1}{2} + \frac{\imath \sqrt{3}}{2}$}
KAZDEMU NEJAKY TAKOVYTO VYSTUP:

Numeration System: eisenstein 

Minimal polynomial of $\omega$: $ t^{2} + t + 1 $

Base $\beta= \omega - 1 $

Minimal polynomial of base: $ x^{2} + 3x + 3 $

Alphabet $\mathcal{A} =\left\{0, 1, -1, \omega, -\omega, -\omega - 1, \omega + 1\right\}$

Input alphabet $\mathcal{B} =\mathcal{A}+ \mathcal{A}$

Phase 1 was succesfull. 

Weight Coefficient Set:
\begin{equation*}
 \mathcal{Q}=\left\{0, 1, 2, -\omega, 2\omega, 2\omega + 1, 2\omega + 2, \omega - 1, \omega, \omega + 1, \omega + 2, -\omega - 1, -\omega - 2, -2\omega, -\omega + 1, -1, -2\omega - 2, -2\omega - 1, -2\right\}
\end{equation*}


Number of elements in the weight coefficient set $\mathcal{Q}$ is: $ 19 $

Weight function Info:

Phase 2 was succesfull. 

Maximal input length: $ 3 $


Number of inputs: $ 6085 $


\subsection{Penney base $\beta = -1 + \imath$ with $\imath = \exp{\frac{2 \pi \imath}{4}}$}


\subsection{Base $\beta = 1 + \imath$}


\subsection{Base $\beta = -\frac{3}{2} + \frac{\imath \sqrt{11}}{2}$}


\subsection{Base $\beta = -2 + \imath$}




\chapter*{Summary}
\addcontentsline{toc}{chapter}{Summary}
The main goal of this thesis was to design and implement the extending window method in SageMath. In order to do that, first, we have recalled the definitions and the previous results in the field of parallel addition algorithms. We have proved Theorem \ref{thm:divisibility} which is necessary tool for computation in $\Zomega$.

From the general concept of construction of parallel addition algorithm, we have designed both phases of the extending window method for rewriting rule $x-\beta$. The sufficient condition for the convergence of Phase 1, i.e., the search for the weight coefficients set $\Q$, have been introduced in Theorem \ref{thm:suffCondPhase1}. The algebraic integer $\omega$ have been categorized according to this sufficient condition. Next, we have developed Algorithm \ref{alg:oneletterSets} which checks the necessary condition for the convergence of Phase 2, i.e. the search for the weight function $q$. This algorithm is based on Theorem \ref{thm:stoppingCondition} which we have proved.

Both phases were implemented in SageMath. The graphic user interface is provided for comfortable using in SageMath Cloud and the shell iterface enables us to compute the weight function for the more demanding numeration systems.

We have tested several examples of numeration systems. Our program found the weight function successfuly for many of them. We have also examples for which Phase 1 converges, although the sufficient condition given by Theorem \ref{thm:suffCondPhase1} does not hold. 

Many questions remains open:
\begin{itemize}
\item What is the necessary condition of convergence of Phase 1? Is there any connection with modulus of the conjugates of the base?
\item Can we ensure the convergence of Phase 1 for wider class of numeration systems, for instance by using different metric in $\CC$?
\item Is there any example when necessary condition of Phase 2 is satisfied, but Phase 2 does not converge?
\item How can we improve Phase 2 to converge for more numeration systems?
\end{itemize}
We focus on these questions in the future work.


% \newpage
\bibliography{literatura}
\addcontentsline{toc}{chapter}{References}
\bibliographystyle{amsplain}


\appendix
\chapter*{Appendices}
\pagenumbering{Roman}
\addcontentsline{toc}{chapter}{Appendices}
\renewcommand{\thesection}{\Alph{section}}
% \section{Illustration of Phase 1}
\label{app:phase1}   
Figures \ref{img:phase1img1} -- \ref{img:phase1img13} illustrates the construction of the weight coefficients set $\Q$ for the Eisenstein numeration system with complex alphabet (see Example \ref{ex:Eisenstein1-blockcomplex}).

\figurehascaptionOne{1 = The starting set $\Q_0{=}\{0\}$.}
\figurehascaptionOne{2 = The set $\B+\Q_0$ need to be covered.}
\figurehascaptionOne{3 = The set $\Q_0$ does not cover the set $\B+\Q_0${,} i.e.{,} the set $\A+\beta \cdot \Q_0$ is not superset of $\B+\Q_0$.}
\figurehascaptionOne{4 = The set $\Q_0$ is extended to $\Q_1$ to cover all elements of $\B+\Q_0$.}
\figurehascaptionOne{5 = The set $\B+\Q_1$ need to be covered.}
\figurehascaptionOne{6 = The set $\Q_1$ does not cover the set $\B+\Q_1${,} i.e.{,} the set $\A+\beta \cdot \Q_1$ is not superset of $\B+\Q_1$.}
\figurehascaptionOne{7 = The set $\Q_1$ is extended to $\Q_2$ to cover all elements of $\B+\Q_1$.}
\figurehascaptionOne{8 = The set $\B+\Q_2$ need to be covered.}
\figurehascaptionOne{9 = The set $\Q_2$ does not cover the set $\B+\Q_2${,} i.e.{,} the set $\A+\beta \cdot \Q_2$ is not superset of $\B+\Q_2$.}
\figurehascaptionOne{10 = The set $\Q_2$ is extended to $\Q_3$ to cover all elements of $\B+\Q_2$.}
\figurehascaptionOne{11 = The set $\B+\Q_3$ need to be covered.}
\figurehascaptionOne{12 = The set $\Q_3$ covers the set $\B+\Q_3${,} i.e.{,} the set $\A+\beta \cdot \Q_3$ is superset of $\B+\Q_2$.}
\figurehascaptionOne{13 = The final weight coefficients $\Q{=}\Q_3$.}


\foreach \n in {1,...,13} {%
\begin{SCfigure}[][htbp]
    \centering
    \caption{\getcaptionOne{\n}}
    \label{img:phase1img\n}
    \includegraphics[height=0.3\textheight]{img/eisenstein/phase1_image_\n.png}
\end{SCfigure}
    }

\newpage
\figurehascaptionTwo{1 = Phase 2 starts with the weight coefficients set $\Q$ from Phase 1.}
\figurehascaptionTwo{2 = The set $\omega+\Q$ need to be covered.}
\figurehascaptionTwo{3 = The elements of $\omega+\Q$ are covered by the set $\Q_{[\omega]}$.}
\figurehascaptionTwo{4 = The set $\omega+\Q_{[1]}$ need to be covered.}
\figurehascaptionTwo{5 = The elements of $\omega+\Q_{[1]}$ are covered by the set $\Q_{[\omega,1]}$.}
\figurehascaptionTwo{6 = The set $\omega+\Q_{[1,2]}$ need to be covered.}
\figurehascaptionTwo{7 = The elements of $\omega+\Q_{[1]}$ are covered by the set $\Q_{[\omega,1,2]}$ which has only one element{.} This element is the output of the weight function $q{(\omega,1,2)}$.}


\section{Illustration of Phase 2}
The construction of set $\Q_{[\omega,1,2]}$ for the Eisenstein numeration system (see Example \ref{ex:Eisenstein1-blockcomplex}) is illustrated on Figures \ref{img:phase2img1} -- \ref{img:phase2img7}.
\label{app:phase2}    

\foreach \n in {1,...,7} {%
\begin{SCfigure}[][htbp]
    \centering
    \caption{\getcaptionTwo{\n}}
    \label{img:phase2img\n}
    \includegraphics[height=0.3\textheight]{img/eisenstein/phase2_image_\n.png}
\end{SCfigure}
    }
    


\newpage
\section{Sample of input file for shell}
File \verb+input_sample.sage+:
\label{app:inputSample}

\lstinputlisting[language=Python]{input_sample.sage}

\section{GUI in SageMath Cloud}
\label{app:interact}
\begin{figure}[!htbp]
  \centering
  \includegraphics[width=\textwidth]{img/interact1.png}
  \caption{The interact after loading inputs.}
  \label{fig:interact1}
\end{figure}

\begin{figure}[htbp]
  \centering
  \includegraphics[width=\textwidth]{img/interact2.png}
  \caption{The output of the extending window method in the interact}
  \label{fig:interact2}
\end{figure}

\begin{figure}[htbp]
  \centering
  \includegraphics[width=\textwidth]{img/interact3.png}
  \caption{The part of the interact for the sanity check, calling of the weight function and plotting of images of steps of both phases.}
  \label{fig:interact3}
\end{figure}


%%%%%%%%%%%%%%%%%%%%%%%%%%%%%%%%%%%%%%%%%%%%%%%%%%%%%%%%%%%%%%%%%%%%%%%%
%     K O N E C   P R Á C E                                            %
%%%%%%%%%%%%%%%%%%%%%%%%%%%%%%%%%%%%%%%%%%%%%%%%%%%%%%%%%%%%%%%%%%%%%%%%


\end{document}