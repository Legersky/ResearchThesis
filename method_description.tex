\section{Method deduction}
The general concept of addition (standard or parallel) in any numeration system $(\beta,\A)$ is following: we add numbers digitwise and then we convert the result into the alphabet $\A$. Obviously, digitwise addition can be run in parallel, thus the crucial point is the conversion of the obtained result. It can be easily done in a standard way but a parallel conversion is nontrivial. However, formulas are basically same but the choice of coefficients differs.

Now we go step by step more precisely. Let $x=\sum_{-m'}^{n'} x_i\beta^i,y=\sum_{-m'}^{n'} y_i\beta^i \in \fin{\A}$ with $(\beta,\A)$-representantions padded by zeros to have the same length. We set 
  \begin{align*}
    w&=x+y =\sum_{-m'}^{n'} x_i\beta^i + \sum_{-m'}^{n'} y_i\beta^i = \sum_{-m'}^{n'} (x_i+y_i)\beta^i \\
    &=\sum_{-m'}^{n'} w_i\beta^i \,,
  \end{align*}
  where $w_j=x_j+y_j \in \A +\A$. Thus, $w_{n'} w_{{n'}-1}\cdots w_1 w_0 \bullet w_{-1} w_{-2} \cdots w_{-m'}$ is a  $(\beta, \A+\A)$-representation of $w\in \fin{\A+\A}$. 

We also use column notation of addition in the following, e.g.,     
  \begin{align*}
  x_{n'} \;x_{{n'}-1}\cdots x_1 \;x_0 &\bullet x_{-1} \;x_{-2}\, \cdots x_{-m'} \\[-3pt]
  y_{n'} \;y_{{n'}-1}\cdots y_1 \,\;y_0 &\bullet y_{-1} \;y_{-2} \;\cdots y_{-m'} \\[-7pt]
    \line(1,0){90} & \line(1,0){100} \\[-7pt]
  w_{n'} w_{{n'}-1}\cdots w_1 w_0 &\bullet w_{-1} w_{-2} \cdots w_{-m'}\,.
  \end{align*}
  

  
  
  
  
  
  Goal -- find $(\beta,\A)$-representation of $w$, i.e. a sequence $$z_{n'} z_{n'-1}\cdots z_1 z_0 z_{-1} z_{-2} \cdots z_{-m'}$$ such that $z_j \in \A$ and
  $$
    z_{n'} z_{n'-1}\cdots z_1 z_0 \bullet z_{-1} z_{-2} \cdots z_{-m'}=w
  $$ 





    We have $0=1\cdot \beta^j -\beta \cdot \beta^{j-1}=1 (-\!\beta) 0 \cdots 0\bullet $ 
    
    $\rightarrow$ We add $q_j\cdot 0$ for each $j$:
    
    $0=q_j\cdot \beta^j -\beta  q_j \cdot \beta^{j-1}$
    \begin{align*}
        w_n w_{n-1}&&&\cdots& &w_{j+1}&\!\! &\textcolor{red}{w_j}  & \!\!  &w_{j-1} &&\cdots &&w_1 w_0\bullet \hspace{200pt}\\[-5pt]
                   &&&&       &       & &     &   &q_{j-2} &&\iddots\\[-3pt] 
                   &&&&       &       & &\textcolor{red}{q_{j-1}}& -&\beta q_{j-1} \\[-3pt]
                   &&&&         &q_j&   \textcolor{red}{-}&\textcolor{red}{\beta q_j} &&\\[-8pt]
                   &&&  \iddots      &   -&\beta q_{j+1}&   &\ &&\\[-17pt]
    \intertext{\line(1,0){280}}
    \vspace{-15pt}
    \\[-30pt]
     z_{n+1} z_{n} z_{n-1}&&&\cdots& &z_{j+1}& &\textcolor{red}{z_j}& &z_{j-1} &&\cdots &&z_1 \; z_0\bullet                            
    \end{align*}
    $\implies$
    $$
        \textcolor{red}{z_j=w_j + q_{j-1} - q_j \beta} \in \A\,
    $$
        
%     
%   Weight coefficient $q_j$ is chosen such that it returns the sum of the digit $w_j$  and the right carry $q_{j-1}$  back to the alphabet~$\A$.





  Assume now a standard numeration system $(\beta, \A)$:
  $$
    \beta \in \NN\,,\beta  \geq 2\,, \A = \{0, 1, 2,\dots, \beta -1 \}
  $$ 
  
  Conversion runs from right to left:
  \begin{align*}
    w_n w_{n-1}\cdots w_{j+1}& \textcolor{red}{w_j w_{j-1} \cdots w_1 w_0}\bullet& \,,w_i &\in \A+\A\,,    \\
    \longrightarrow z_{n+1}\; z_{n}\; z_{n-1}\;\cdots z_{j+1} &\textcolor{red}{z_j} \; z_{j-1}\; \cdots \;z_1 \; z_0\bullet& \,,z_i &\in \A\,.
  \end{align*}
    $$
        \textcolor{red}{z_j=w_j + q_{j-1} - q_j \beta}\,
    $$
   
  
    The weight coefficients $q_j$ and digits $z_j$ are unique in the standard numeration system 
    
    $$
    \implies \textcolor{red}{z_j=z_j(w_j ,\dots , w1, w_0)}
    $$  
    
    Thus, addition is linear in length of inputs.




    {Parallel Addition}
    Introduced by Avizienis in 1961:
  \begin{align*}
    \cdots w_{j+t+1}\textcolor{red}{w_{j+t} \cdots w_{j+1}}& \textcolor{red}{w_j w_{j-1} \cdots w_{j-r}}w_{j-r-1} \cdots &,\, w_i &\in \A + \A\,,    \\
    \longrightarrow \cdots z_{j+t+1}\;z_{j+t} \; \cdots \; z_{j+1} &\textcolor{red}{z_j} \; z_{j-1}\; \cdots z_{j-r}\;z_{j-r-1}\; \cdots &,\, z_i &\in \A\,.
  \end{align*}
  Digit conversion is a mapping $(\A+\A)^{t+r+1} \to \A$:%so called \textit{$p$-local function} with the window of the~fixed length $p=t+r+1$: 
  $$
    \textcolor{red}{z_j=z_j(w_{j+t} \cdots w_{j+1}w_j w_{j-1} \cdots w_{j-r})}\,.
  $$
  
  Thus the parallel addition is done in constant time but the numeration system must be redundant -- a number has more than one admissible representation.
  
  
  For example:
  $$
  \beta \in \NN, \beta \geq 3, \A=\{-a, \dots, 0, \dots a\}, b/2 <a \leq b-1\,. 
  $$
    


\subsection{Non-standard numeration systems}

    {Non-standard numeration systems}
    
    Integer alphabets:
    \begin{itemize}
        \item Base $\beta \in \CC, |\beta|>1$.
        
        \item Addition is computable in parallel if and only if $\beta$ is an~algebraic number with no conjugate of modulus 1 [Frougny, Heller, Pelantov\'a, S. ].
        
        \item Algorithms are known but with large alphabets.
    \end{itemize}
    
    
    Non-integer alphabets:
    \begin{itemize}
        \item Base $\beta \in \Zomega= \left\{\sum_{j=0}^{d-1} a_j \omega^j\ : a_j \in \ZZ \right\}$, where $\omega \in \CC$ is an~algebraic integer of degree $d$.
        
        \item Alphabet $\A \subset \Zomega , 0\in \A$.
        
        \item Only few manually found algorithms. 
    \end{itemize}
    
    
    
    How do we find systematically algorithms with minimal alphabets?  



\section{Method}

    {Method for construction of parallel addition algorithms}
    Key problem -- find weight coefficients $q_j$ such that 
    $$
        z_j=\underbrace{w_j}_{\in \A +\A} + q_{j-1} - q_j \beta \in \A 
    $$  
    for any input $w$ and every $j$.
    
    
    In order to do that we need to find $M \in \NN$ and $q:(\A+\A)^{M} \rightarrow \Q \subset \Zomega$ such that $q_j=q(w_j, \dots, w_{j-M+1})$.
    
    The mapping $q$ is called the \textit{weight function}.% $q:(\A+\A)^{M} \rightarrow \Q \subset \Zomega$. 
    
    
    \vspace{20pt}
    Our method:
    \begin{enumerate}
        \item Find set $\Q \subset \Zomega$ of possible weight coefficients.
        \item Increment $M$ and for each $w_j,w_{j-1}, \dots , w_{j-M+1} \in (\A+\A)^{M}$ try to find a weight coefficient from $\Q$ to define $q$.
    \end{enumerate}


\subsection{Phase 1 -- Set of weight coefficients}

    {Phase 1 -- searching for the set of weight coefficients}
    We want to find set of weight coefficients $\Q \subset \Zomega$ such that
    $$
    \underbrace{(\A+\A)}_{w_j \in}+ \underbrace{\Q}_{q_{j-1} \in} \subset \underbrace{\A}_{z_j \in} + \underbrace{\beta \Q}_{\beta q_j \in}
    $$
    
    It implies that there is $q_j \in \Q$ such that
    $$
    z_j=w_j + q_{j-1} - q_j \beta \in \A \,.
    $$
    for all $q_{j-1} \in \Q$ and $w_j \in \A+\A$.




    We build $\Q$ iteratively:
    
    
    {Phase 1}
     $k:=0$
     
      Set $\Q_0:=\{0\}$
      
      
      Repeat:
      \begin{itemize}
          \item extend $\Q_k$ to $\textcolor{red}{\Q_{k+1}}$ in a minimal possible way so that
           $$
              (\A+\A)+ \Q_k \subset \A + \beta \textcolor{red}{\Q_{k+1}}\,,
           $$
           \item $k:=k+1$
      \end{itemize}
      
      until $\Q_k = \Q_{k+1}$.
      
      
      \vspace{7pt}
      $\Q:=\Q_k$
    





\subsection{Phase 2 -- Weight function}

    {Phase 2 -- searching for a weight function}

    We want to find a length of the window $M$ and a weight function $q:(\A+\A)^{M} \to \Q$.% have $q_j=q(w_j,w_{j-1},\dots, w_{j-M+1})$, where $q$ is a weight function:
%     \begin{align*}
%         \cdots\; &w_{j+1}&\!\! &\textcolor{red}{w_j}  & \!\!  &\textcolor{red}{w_{j-1}\cdots w_{j-M+1}} w_{j-M} \cdots\\[-3pt]
%                          & &       &q_{j-1}& -&\beta q_{j-1} \\[-1pt]
%                            &q_j&   -&\beta \textcolor{red}{q_j} &&\\[-19pt]      
%     \intertext{\line(1,0){280}}
%     \\[-30pt]
%      \cdots\; &z_{j+1}& &z_j& &z_{j-1} \cdots z_{j-M+1}\; z_{j-M}\cdots                     
%     \end{align*}
    
    
    Suppose that the length of the window is $m$.
    
    The idea is to check all possible right carries $q_{j-1}$ and determine values $q_j$ such that 
    $$
    z_j=w_j + q_{j-1} - q_j \beta \in \A \,.
    $$
    The set of all such needed values of $q_j$ is denoted by $\Q_{[w_{j},\dots, w_{j-m+1}]}\subset \Q$
        
    
    The length $M$ and weight function $q$ is found when 
    $$
    \#\Q_{[w_{j},\dots, w_{j-M+1}]}=1
    $$
    for all $w_{j},\dots, w_{j-M+1} \in (\A+\A)^M$.
%     \begin{equation*}
%     w_j + \underbrace{q(w_{j-1},w_{j-2},\dots, w_{j-M})}_{\in \Q_{[w_{j-1},w_{j-2},\dots, w_{j-M}]}  \subset \Q }=z_j + \beta \cdot \underbrace{q(w_j,w_{j-1},\dots, w_{j-M+1})}_{\in \Q_{[w_j,w_{j-1},\dots, w_{j-M+1}]} \subset \Q}
%     \end{equation*}
    



    {Phase 2}
      $m:=1$
      
      For each $w_j \in \A+\A$ find minimal set $\Q_{[w_j]} \subset \Q$ such that
      $$
      w_j + \Q \subset \A + \beta \Q_{[w_j]}
      $$
    
      While $(\max\{\#\Q_{[w_j,\dots, w_{j-m+1}]}:(w_j,\dots, w_{j-m+1}) \in (\A+\A)^m \} > 1)$ do:
      \begin{itemize}
          
          \item $m:= m +1$
          
          \item For each $(w_j,\dots, w_{j-m+1}) \in (\A+\A)^{m}$ find minimal set $\textcolor{red}{\Q_{[w_j,\dots, w_{j-m+1}]}} \subset \Q_{[w_j,\dots, w_{j-m+2}]}$ such that
          $$
          w_j + \Q_{[w_{j-1},\dots, w_{j-m+1}]} \subset \A + \beta \textcolor{red}{\Q_{[w_j,\dots, w_{j-m+1}]}}\,,
          $$
      \end{itemize}
      
      $M:= m$ 
      
      
      $q(w_j,\dots, w_{j-M+1}):=$ only element of $\Q_{[w_j,\dots, w_{j-M+1}]}$
    



        Now we have parallel conversion algorithm:
    \begin{align*}
    z_j&=w_j + q_{j-1} - q_j \beta = \\
       &=w_j + q(w_{j-1},w_{j-2},\dots, w_{j-M}) - \beta q(w_j,w_{j-1},\dots, w_{j-M+1}) = \\
       &= z_j(w_{j},w_{j-1},\dots, w_{j-M})\,.
    \end{align*}




