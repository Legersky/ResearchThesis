The designed method requires the arithmetics in $\Zomega$. Therefore, we have chosen Python-based programming language Sage for the implementation of method as it contains many ready-to-use mathematical structure. Specifically, we use the provided class \emph{PolynomialQuotientRing} to represent elements of $\Zomega$ and  \emph{NumberField} for obtaining numerical complex value of them. Using Sage is very convenient as it also offers easily usable datastructures or tools for plotting. Thus the code is more readible and we may focus on the algorithmic part of problem. On the other hand, Sage is considerably slower than for C++ or other low-level languages. Nevertheless, it is sufficient for our purpose.

The implementation is object-oriented. It consists of four classes. Class \emph{AlgorithmForParallelAddition} contains structures for computations in $\Zomega$ and links neccessary instances and functions to construct algorithm for parallel addition by the extending window method for an algebraic integer $\omega$ given by its minimal polynomial $p$ and approximate complex value, a base $\beta\in\Zomega$, an alphabet $\A\subset\Zomega$ and an input alphabet $\B$. Phase 1 of the extending window method is implemented in class \emph{WeightCoefficientsSetSearch} and Phase 2 in \emph{WeightFunctionSearch}. Class \emph{WeightFunction} holds the weight function $q$ computed in Phase 2. All classes are described in the following sections including lists of all method with short desription. Sometimes, notation from Chapter \ref{chap:methodDescription} is used for better understanding.

Basically, weight function can be found just by creating an instance of \emph{AlgorithmForParallelAddition} and calling \textbf{findWeightFunction()}. For more comfortable usage, our implementation includes two interfaces -- command line version and graphic user interface using interact in SageMath Cloud. The whole implementation is on the attached CD or it can be downloaded from  \url{https://github.com/Legersky/ParallelAddition}.

\section{Class AlgorithmForParallelAddition}
This class constructs neccessary structures for computation in $\Zomega$. It is \emph{PolynomialQuotientRing} obtained as a \emph{PolynomialRing} over integers factored by polynomial $p$. This is used for representation of elements of $\Zomega$ and arithmetics. We remark that it is independent on the choice of root of the  minimal polynomial $p$. But as we need also comparisons of numbers in $\Zomega$ in modulus, we specify $\omega$ by its approximate complex value and we form a factor ring of rational polynomials by using class \emph{NumberField}. This enables us to get absolute values of elements of $\Zomega$ which can be then compared.

Method \textbf{findWeightFunction()} links together both phases of the extending window method to find the weight function $q$. That is used in the methods for addition and digit set conversion to process them as local functions. There are also verification methods.

Moreover, many methods for printing, plotting and saving outputs are implemented.

Constructor:

\begin{method}{\_\_init\_\_}{minPol\_str, embd, alphabet, base, name='NumerationSystem', inputAlphabet='', printLog=True, printLogLatex=False, verbose=0}
Take \var{minPol\_str} which is symbolic expression in the variable $x$ of minimal polynomial $p$. The closest root of  \var{minPol\_str} to \var{embd} is used as the ring generator $\omega$. The structures for $\Zomega$ are constructed as described above. Setters \fun{setAlphabet}{alphabet}, \fun{setInputAlphabet}{A} and \fun{setBase}{base} are called. Messages saved to logfile during existence of an instance are printed (using \LaTeX) on standard output depending on \var{printLog} and \var{printLogLatex}. The level of messages for a development is set by \var{verbose}. 
\end{method}

Setters and getters:

\begin{method}{setAlphabet}{A}
Check if \var{A} is subset of $\Zomega$. Set alphabet $\A$:=\var{A}
\end{method}

\begin{method}{setInputAlphabet}{B}
If \var{B} is empty, $\A+\A$ is used. Set the input alphabet $\B$:=\var{B}. Check if $\A\subsetneq\B\subset\A+\A$. 
\end{method}


-----------------------------EXTENDING WINDOW METHOD------------------------------------------------------------------------

\begin{method}{\_findWeightCoefSet}{ max\_iterations, method\_number}
Create an instance of \emph{WeightCoefficientsSetSearch(method\_number)} and call its method \fun{findWeightCoefficientsSet}{max\_iterations} to obtain a weight coefficients set $\Q$.
\end{method}

\begin{method}{addWeightCoefSetIncrement}{ increment}
Save \var{increment} from extending intermediate weight coefficients set $\Q_{k}$ to $\Q_{k+1}$.
\end{method}

\begin{method}{\_findWeightFunction}{ max\_input\_length,method\_number}
Create an instance of \emph{WeightFunctionSearch(method\_number)} and call its methods \fun{check\_one\_letter\_inputs}{max\_input\_length} and \fun{findWeightFunction}{max\_input\_length} to obtain a weight function $q$.
\end{method}


\begin{method}{findWeightFunction}{ max\_iterations, max\_input\_length, method\_weightCoefSet=2, method\_weightFunSearch=4}
Return the weight function $q$ obtaind by calling \fun{\_findWeightCoefSet}{max\_iterations,method\_weightCoefSet} and \fun{\_findWeightFunction}{max\_input\_length, method\_weightFunSearch}.
\end{method}


-----------------------------PARALLEL ADDITION AND CONVERSION---------------------------------------------------------------

\begin{method}{addParallel}{a,b}
Sum up numbers represented by lists of digits \var{a} and \var{b} digitwise and convert the result by \fun{parallelConversion}{}. 
\end{method}


\begin{method}{parallelConversion}{\_w}
Return $(\beta,\A)$-representation of number represented by list \var{\_w} of digits in input alphabet $\B$. It is computed locally according to the equation \ref{eq:conversionFormula} and using weight function $q$.
\end{method}


\begin{method}{localConversion}{w}

\end{method}


-----------------------------SANITY CHECK-----------------------------------------------------------------------------------

\begin{method}{sanityCheck\_conversion}{ num\_digits}

\end{method}


-----------------------------RING CONVERSIONS, AUXILIARY RING FUNCTIONS-----------------------------------------------------

\begin{method}{\_computeInverseBaseCompanionMatrix}{}

\end{method}


\begin{method}{divideByBase}{divided\_number}

\end{method}




\section{Class WeightCoefficientsSetSearch}
-----------------------------CONSTRUCTOR, SETTER-------------------------------------------------------------------

\begin{method}{\_\_init\_\_}{ algForParallelAdd, method}

\end{method}


\begin{method}{\_\_repr\_\_}{}

\end{method}


\begin{method}{setVerbose}{verb}

\end{method}


-----------------------------FIND CANDIDATES-------------------------------------------------------------------

\begin{method}{\_findCandidates}{C}

\end{method}


-----------------------------SEARCH FOR WEIGHT COEFFICIENT SET--------------------------------------------------

\begin{method}{\_getQk}{C}

\end{method}


\begin{method}{\_chooseQk\_FromCandidates}{cand\_for\_all}

\end{method}


\begin{method}{findWeightCoefficientsSet}{ maxIterations}

\end{method}


-----------------------------AUXILIARY FUNCTIONS-------------------------------------------------------------------




\begin{method}{\_findSmallest}{list\_from\_Ring}

\end{method}



\section{Class WeightFunctionSearch}
-----------------------------CONSTRUCTOR, SETTERS-------------------------------------------------------------------

\begin{method}{\_\_init\_\_}{ algForParallelAdd, weightCoefSet, method}

\end{method}

-----------------------------SEARCH FOR WEIGHT FUNCTION-------------------------------------------------------------------

\begin{method}{\_find\_weightCoef\_for\_comb\_B}{ combinations, max\_len}

\end{method}


\begin{method}{\_findQw}{w\_tuple}

\end{method}


\begin{method}{findWeightFunction}{ max\_input\_length}

\end{method}


\begin{method}{check\_one\_letter\_inputs}{ max\_input\_length}

\end{method}



\section{Class WeightFunction}
-----------------------------CONSTRUCTOR, GETTERS-------------------------------------------------------------------

\begin{method}{\_\_init\_\_}{B}

\end{method}


\begin{method}{\_\_repr\_\_}{}

\end{method}


\begin{method}{getMaxLength}{}

\end{method}


\begin{method}{getMapping}{}

\end{method}


-----------------------------ADDING INPUTS, CALL FUNCTION-------------------------------------------------------------------

\begin{method}{\_\_call\_\_}{ input\_tuple}

\end{method}


\begin{method}{getWeightCoef}{ w}

\end{method}


\begin{method}{addWeightCoefToInput}{\_input, coef}

\end{method}


-----------------------------PRINT FUNCTIONS-------------------------------------------------------------------

\begin{method}{printInfo}{}

\end{method}


\begin{method}{printLatexInfo}{}

\end{method}


\begin{method}{printMapping}{}

\end{method}


\begin{method}{printLatexMapping}{}

\end{method}


\begin{method}{printCsvMapping}{}

\end{method}


\section{User interfaces}