The designed method requires the arithmetics in $\Zomega$. Therefore, we have chosen Python-based programming language SageMath for the implementation of method as it contains many ready-to-use mathematical structure.  Using SageMath is very convenient as it also offers easily usable datastructures or tools for plotting. Thus the code is more readible and we may focus on the algorithmic part of problem. On the other hand, SageMath is considerably slower than for example C++ or other low-level languages. Nevertheless, it is sufficient for our purpose.

The implementation is object-oriented. It consists of four classes. Class \emph{AlgorithmForParallelAddition} contains structures for computations in $\Zomega$. Specifically, we use the provided class \emph{PolynomialQuotientRing} to represent elements of $\Zomega$ and  \emph{NumberField} for obtaining numerical complex value of them. The class also links neccessary instances and functions to construct algorithm for parallel addition by the extending window method for an algebraic integer $\omega$ given by its minimal polynomial $p$ and approximate complex value, a base $\beta\in\Zomega$, an alphabet $\A\subset\Zomega$ and an input alphabet $\B$. Phase 1 of the extending window method is implemented in class \emph{WeightCoefficientsSetSearch} and Phase 2 in \emph{WeightFunctionSearch}. Class \emph{WeightFunction} holds the weight function $q$ computed in Phase 2. All classes are described in the following sections including lists of the important methods  with short desription. Sometimes, notation from Chapter \ref{chap:methodDescription} is used for better understanding. For all implemented methods, see commented source code.  

Basically, weight function can be found just by creating an instance of \emph{AlgorithmForParallelAddition} and calling \textbf{findWeightFunction()}. For more comfortable usage, our implementation includes two interfaces -- shell version and graphic user interface using interact in SageMath Cloud. The whole implementation is on the attached CD or it can be downloaded from  \url{https://github.com/Legersky/ParallelAddition}.

\section{Class AlgorithmForParallelAddition}
Constructor:

\begin{method}{\_\_init\_\_}{minPol\_str, embd, alphabet, base, name='NumerationSystem', inputAlphabet='', printLog=True, printLogLatex=False, verbose=0}
Take \var{minPol\_str} which is symbolic expression in the variable $x$ of minimal polynomial $p$. The closest root of  \var{minPol\_str} to \var{embd} is used as the ring generator $\omega$. The structures for $\Zomega$ are constructed as described above. Setters \fun{setAlphabet}{alphabet}, \fun{setInputAlphabet}{A} and \fun{setBase}{base} are called. Messages saved to logfile during existence of an instance are printed (using \LaTeX) on standard output depending on \var{printLog} and \var{printLogLatex}. The level of messages for a development is set by \var{verbose}. 
\end{method}

Setters and getters:

\begin{method}{setAlphabet}{A}
Check if \var{A} is subset of $\Zomega$. Set alphabet $\A$:=\var{A}
\end{method}

\begin{method}{setInputAlphabet}{B}
If \var{B} is empty, $\A+\A$ is used. Set the input alphabet $\B$:=\var{B}. Check if $\A\subsetneq\B\subset\A+\A$. 
\end{method}


-----------------------------EXTENDING WINDOW METHOD------------------------------------------------------------------------

\begin{method}{\_findWeightCoefSet}{ max\_iterations, method\_number}
Create an instance of \emph{WeightCoefficientsSetSearch(method\_number)} and call its method \fun{findWeightCoefficientsSet}{max\_iterations} to obtain a weight coefficients set $\Q$.
\end{method}

\begin{method}{addWeightCoefSetIncrement}{ increment}
Save \var{increment} from extending intermediate weight coefficients set $\Q_{k}$ to $\Q_{k+1}$.
\end{method}

\begin{method}{\_findWeightFunction}{ max\_input\_length,method\_number}
Create an instance of \emph{WeightFunctionSearch(method\_number)} and call its methods \fun{check\_one\_letter\_inputs}{max\_input\_length} and \fun{findWeightFunction}{max\_input\_length} to obtain a weight function $q$.
\end{method}


\begin{method}{findWeightFunction}{ max\_iterations, max\_input\_length, method\_weightCoefSet=2, method\_weightFunSearch=4}
Return the weight function $q$ obtaind by calling \fun{\_findWeightCoefSet}{max\_iterations,method\_weightCoefSet} and \fun{\_findWeightFunction}{max\_input\_length, method\_weightFunSearch}.
\end{method}


-----------------------------PARALLEL ADDITION AND CONVERSION---------------------------------------------------------------

\begin{method}{addParallel}{a,b}
Sum up numbers represented by lists of digits \var{a} and \var{b} digitwise and convert the result by \fun{parallelConversion}{}. 
\end{method}


\begin{method}{parallelConversion}{\_w}
Return $(\beta,\A)$-representation of number represented by list \var{\_w} of digits in input alphabet $\B$. It is computed locally according to the equation \ref{eq:conversionFormula} and using weight function $q$.
\end{method}


\begin{method}{localConversion}{w}

\end{method}


-----------------------------SANITY CHECK-----------------------------------------------------------------------------------

\begin{method}{sanityCheck\_conversion}{ num\_digits}

\end{method}


-----------------------------RING CONVERSIONS, AUXILIARY RING FUNCTIONS-----------------------------------------------------

\begin{method}{\_computeInverseBaseCompanionMatrix}{}

\end{method}


\begin{method}{divideByBase}{divided\_number}

\end{method}




\section{Class WeightCoefficientsSetSearch}
-----------------------------CONSTRUCTOR, SETTER-------------------------------------------------------------------

\begin{method}{\_\_init\_\_}{ algForParallelAdd, method}

\end{method}


\begin{method}{\_\_repr\_\_}{}

\end{method}


\begin{method}{setVerbose}{verb}

\end{method}


-----------------------------FIND CANDIDATES-------------------------------------------------------------------

\begin{method}{\_findCandidates}{C}

\end{method}


-----------------------------SEARCH FOR WEIGHT COEFFICIENT SET--------------------------------------------------

\begin{method}{\_getQk}{C}

\end{method}


\begin{method}{\_chooseQk\_FromCandidates}{cand\_for\_all}

\end{method}


\begin{method}{findWeightCoefficientsSet}{ maxIterations}

\end{method}


-----------------------------AUXILIARY FUNCTIONS-------------------------------------------------------------------




\begin{method}{\_findSmallest}{list\_from\_Ring}

\end{method}



\section{Class WeightFunctionSearch}
-----------------------------CONSTRUCTOR, SETTERS-------------------------------------------------------------------

\begin{method}{\_\_init\_\_}{ algForParallelAdd, weightCoefSet, method}

\end{method}

-----------------------------SEARCH FOR WEIGHT FUNCTION-------------------------------------------------------------------

\begin{method}{\_find\_weightCoef\_for\_comb\_B}{ combinations, max\_len}

\end{method}


\begin{method}{\_findQw}{w\_tuple}

\end{method}


\begin{method}{findWeightFunction}{ max\_input\_length}

\end{method}


\begin{method}{check\_one\_letter\_inputs}{ max\_input\_length}

\end{method}



\section{Class WeightFunction}
-----------------------------CONSTRUCTOR, GETTERS-------------------------------------------------------------------

\begin{method}{\_\_init\_\_}{B}

\end{method}


\begin{method}{\_\_repr\_\_}{}

\end{method}


\begin{method}{getMaxLength}{}

\end{method}


\begin{method}{getMapping}{}

\end{method}


-----------------------------ADDING INPUTS, CALL FUNCTION-------------------------------------------------------------------

\begin{method}{\_\_call\_\_}{ input\_tuple}

\end{method}


\begin{method}{getWeightCoef}{ w}

\end{method}


\begin{method}{addWeightCoefToInput}{\_input, coef}

\end{method}


-----------------------------PRINT FUNCTIONS-------------------------------------------------------------------

\begin{method}{printInfo}{}

\end{method}


\begin{method}{printLatexInfo}{}

\end{method}


\begin{method}{printMapping}{}

\end{method}


\begin{method}{printLatexMapping}{}

\end{method}


\begin{method}{printCsvMapping}{}

\end{method}


\section{User interfaces}
We provide two interfaces for running of the implemented extending window method -- the~shell version and graphic user interface.

\subsection{Shell}
The implementation of the extending window method is launched in a shell by typping \verb+sage parallelAdd.sage <input_sample.sage>+. The parameters of the numeration system and the setting of outputs and computation is given by SageMath file \verb+input_sample.sage+. See Appendix \ref{app:inputSample} for the example of such a file with parameters of Eisenstein numeration system.

The name of the numeration system, minimal polynomial of generator $\omega$, an approximate value of $\omega$, the base $\beta$, alphabet $\A$ and input alphabet $\B$ are set in the part INPUTS. The maximum number of iterations in Phase 1, maximal length of the window in Phase 2 and the launching of the sanity check are set in SETTING. 

The boolean values in the part SAVING determines which formats of the outputs are saved. All outputs are saved in the folder \verb+./ouputs/name/+. The general info about the computatin can be saved in .tex format, the computed weight function and local digit set conversion in .csv format. The inputs setting saved as dictionary can be loaded by the interact interface. The log of the whole computation can be saved as .txt file. There is also an option to save unsolved combinations in Phase 2 in .csv format in the case of the interruption of the program.

According to the boolean values in the part IMAGES, the figures the alphabet, input alphabet, weight coefficients set or part of the lattice of $\Zomega$ with alphabet shifted into points which are divisible by the base $\beta$ are saved in .png format to folder \verb+./ouputs/name/img/+. Optionally, the weight coefficients set is plotted with bound given by the proof of Theorem \ref{thm:suffCondPhase1} The images of individual steps of both phases of the extending window method can be saved, too. For Phase 2, the search for the weight coefficient  is plotted for digits given by \verb+phase2_input+.  

The program print out all inputs and then it computes the weight function $q$ by calling \fun{findWeightFunction}{ max\_iterations, max\_input\_length}. The increments of the weight coefficients set in each iteration of Phase 1 are printed and then also the obtained weight coefficients set $\Q$. The longest inputs given by repetetion of one letter are printed after the computation of \fun{check\_one\_letter\_inputs}{max\_input\_length}. During computing of Phase 2, the current length of window and the number of saved combinations are printed. Finally, the length of window, elapsed time and info about saved outputs are printed.  

It is possible to batch process all input files in one folder by executing the bash script \verb+parAdd_batch <folder_name>+.  

\subsection{Interact in SageMath Cloud}
The graphic user interface is implemented using interact in SageMath Cloud. The parts of the interact are on Figure \ref{fig:interact1}, \ref{fig:interact2} and \ref{fig:interact3} in Appendix \ref{app:interact}. The functionality is basically the same as the shell version. After executing of the cell by Shift+Enter, the parameters of the numeration system are filled in the corresponding spaces or one of the previous settings is loaded by typing its name.  By default, the last inputs are shown in the form. The inputs are submitted by pressing the button Update. Using check-boxes, the formats of output are chosen and the search for the weight function is launched by pressing second button Update.

The printed output is similar to the shell output. In addition, it contains the figures and it is formatted using \LaTeX. Moreover, the sanity check can be run for a given length, the weight coefficient for a tuple of  input digits is returned or the images of individual steps of both phases are shown and saved.