We have tested several bases with different alphabets. The alphabets are chosen such that the contains all representatives modulo $\beta-1$ according to Theorem \ref{thm:moduloAlphabet}?????????????????????????. The sizes of the integer alphabets are given by the lower bound from Theorem \ref{thm:lowerBoundAlphabet}. Since this theorem assumes only an integer alphabet, we have tested also noninteger alphabets of smaller size. The input alphabet $\B$ is always $\A+ \A$.  

 Table \ref{tbl:results} summarizes the tested numeration systems which are described in the sections below. The first column of the table says if the sufficient condition given by Theorem \ref{thm:suffCondPhase1} is satisfied. The second one shows whether Phase 1 was successful (\checkmark) or not (\xmark) for the given numeration system. The third column is the control of the necessary condition for the convergence of Phase 2 by Algorithm \ref{alg:oneletterSets}, i.e. if there is the output of the weight function $q$ for input digits $b,b,\dots,b$ for all $b\in\B$. The results of Phase 2 are in the last column. Notice, that the next step of the extending window method is not processed without the previous one (--).

The complete results including log files and images can be found on the attached CD  or \url{https://github.com/Legersky/ParallelAddition}.

\begin{table}[!htb]
\centering
  \begin{tabular}{l r|c c c c}
      Name &  Subsec. & Suff. c. & Phase 1 & Necess. c. & Phase 2 \\ \hline
      Eisenstein\_1-block\_complex& \ref{subsec:Eisenstein1-blockcomplex} & yes  & \checkmark & \checkmark & \checkmark \\
      Eisenstein\_1-block\_small\_complex & \ref{subsec:Eisenstein1-blocksmallcomplex} & yes  & \checkmark & \xmark & --\\
      Eisenstein\_1-block\_integer & \ref{subsec:Eisenstein1-blockinteger} & yes  & \checkmark & \xmark & --\\
      Eisenstein\_2-block & \ref{subsec:Eisenstein2-block} & yes  & \checkmark & \xmark & --\\
      Eisenstein\_2-block\_4elements & \ref{subsec:Eisenstein2-block4elements} & yes  & \checkmark & \xmark & --\\ \hline
      Penney\_1-block\_complex & \ref{subsec:Penney1-blockcomplex} & yes & \checkmark & \checkmark & \xmark \\
      Penney\_1-block\_complex\_small &  \ref{subsec:Penney1-blockcomplexsmall} & yes  & \checkmark & \xmark & --\\
      Penney\_1-block\_integer &  \ref{subsec:Penney1-blockinteger} & yes  & \checkmark & \xmark & --\\
      Penney\_2-block\_integer &  \ref{subsec:Penney2-blockinteger} & yes  & \checkmark & \checkmark & \checkmark \\ \hline
      Quadratic+1-2+2-1\_1-block\_complex & \ref{subsec:Quadratic+1-2+2-1blockcomplex} & yes  &\checkmark & \checkmark & \checkmark \\
      Quadratic+1-2+2-1\_1-block\_integer & \ref{subsec:Quadratic+1-2+2-1blockinteger} & yes  & \checkmark & \xmark & --\\ \hline
      Quadratic+1+4+5\_1-block\_complex & \ref{subsec:Quadratic+1+4+51-blockcomplex} & yes  & \checkmark & \checkmark & \checkmark \\ \hline
      Quadratic+1+3+5\_1-block\_complex & \ref{subsec:Quadratic+1+3+51-blockcomplex} & yes  & \checkmark & \xmark & --\\ \hline
      Quadratic+1-5+3\_1-block\_integer  &\ref{subsec:Quadratic+1-5+31-blockinteger} & no  & \xmark & -- & --\\ \hline
      Quadratic+1-5+5\_1-block\_real  &\ref{subsec:Quadratic+1-5+51-blockreal} & no  & \checkmark & \xmark & --\\ \hline
      base\_2 & \ref{subsec:base2} & yes  & \checkmark & \checkmark & \checkmark \\
      base\_4 & \ref{subsec:base4} & yes  & \checkmark & \checkmark & \checkmark \\ \hline
      Cubic+1+1-5+5\_complex & \ref{subsec:Cubic+1+1-5+5complex} & no & \checkmark & \xmark & --\\
      Cubic+1+1-1+1\_complex & \ref{subsec:Cubic+1+1-1+1complex} & no & \xmark & -- & --\\
  \end{tabular}
  \caption{Results of extending window method.}
  \label{tbl:results}
\end{table} 

\section{\texorpdfstring{Eisenstein base $\beta = -\frac{3}{2} + \frac{\imath \sqrt{3}}{2}$}{Eisenstein base beta = -3/2 + i sqrt(3)/2}}
Eisenstein base $\beta$ equals $\omega - 1$ with $\omega =-\frac{1}{2} + \frac{\imath \sqrt{3}}{2}$. The minimal polynomial of the generator $\omega$ is $x^2 + x+1$ and the minimal polynomial of the base $\beta$ is $x^2 + 3x+3$. We have tested the complex, smaller complex and integer alphabet.
 
\subsection{ Eisenstein\_1-block\_complex }

\label{subsec:Eisenstein1-blockcomplex}

Parameters:
\begin{itemize}
    \item Minimal polynomial of $\omega$: $ t^{2} + t + 1 $
    \item Base $\beta= \omega - 1 $
    \item Minimal polynomial of base: $ x^{2} + 3x + 3 $
    \item Alphabet $\mathcal{A} =\left\{0, 1, -1, \omega, -\omega, -\omega - 1, \omega + 1\right\}$
    \item Input alphabet $\mathcal{B} =\mathcal{A}+ \mathcal{A}$
\end{itemize}


\noindent Extending window method:
\begin{enumerate}
    \item Phase 1 was succesfull.
The number of elements in the weight coefficient set $\mathcal{Q}$ is $19$.

    \item There is a unique weight coefficient for input $b,b,\dots,b$ for all $b\in\mathcal{B}$.

    \item Phase 2 was succesfull.
The lenght of window $m$ of the weight function $q$ is 3.
\end{enumerate}
%Eisenstein\_1-block\_complex & \checkmark & \checkmark & \checkmark \\

\subsection{ Eisenstein\_1-block\_small\_complex }

\label{subsec:Eisenstein1-blocksmallcomplex}

The alphabet $\mathcal{A} =\left\{0, 1, \omega, \omega + 1\right\}$.

\noindent The result of the extending window method is:
\begin{enumerate}
    \item Phase 1 was succesfull.
The number of elements in the weight coefficient set $\mathcal{Q}$ is $17$.

    \item There is not unique weight coefficient for input $b,b,\dots,b$ for the $b= \omega + 2 $ for fixed length of window. Thus Phase 2 does not converge.

\end{enumerate}
%Eisenstein\_1-block\_small\_complex & \ref{subsec:Eisenstein1-blocksmallcomplex} & \checkmark & \xmark & --\\

\subsection{ Eisenstein\_1-block\_integer }

\label{subsec:Eisenstein1-blockinteger}

The alphabet $\mathcal{A} =\left\{0, 1, -1, 2, -2, 3, -3\right\}$.

\noindent The result of the extending window method is:
\begin{enumerate}
    \item Phase 1 was succesfull.
The number of elements in the weight coefficient set $\mathcal{Q}$ is $53$.

    \item There is not unique weight coefficient for input $b,b,\dots,b$ for the $b= 4 $ for fixed length of window. Thus Phase 2 does not converge.

\end{enumerate}
%Eisenstein\_1-block\_integer & \ref{subsec:Eisenstein1-blockinteger} & \checkmark & \xmark & --\\


We may also study so called 2-block parallel addition. Roughly speaking, we consider two digits after each other in a $(\beta,\A)$-representation as one digit of the $(\beta^2,\A+\beta\A)$-representation of the same number. So we shift from base $\beta$ to $\beta^2=-3-3\beta=-3\omega$ which has the minimal polynomial $x^2-3x+9$. We have tested the shifted alphabets $\{0,\pm 1\}+\beta \{0,\pm 1\}$ and $\{0,1, \omega, \omega +1\}+\beta \{0,1, \omega, \omega +1\}$.
  
\begin{exmp}
\textbf{ Eisenstein\_2-block }

\label{ex:Eisenstein2-block}

The alphabet $\mathcal{A} =\left\{0, 1, -1, \omega, -\omega, \omega - 1, -\omega + 1, \omega - 2, -\omega + 2\right\}$.

The elements $ \left\{2\omega - 1, 2\omega, \omega + 1, -\omega - 1, -2\omega, -2\omega + 1\right\} $ have no representative  modulo $\beta-1$ in the alphabet $\mathcal{A}$.
%Eisenstein\_2-block & \ref{ex:Eisenstein2-block} &no & -- & -- & -- & -- &\
\end{exmp}
\subsection{ Eisenstein\_2-block\_4elements }

\label{subsec:Eisenstein2-block4elements}

Parameters:
\begin{itemize}
    \item Minimal polynomial of $\omega$: $ t^{2} + t + 1 $
    \item Base $\beta= -3\omega $
    \item Minimal polynomial of base: $ x^{2} - 3x + 9 $
    \item Alphabet $\mathcal{A} =\left\{0, 1, -1, \omega, -\omega, \omega + 1, -\omega - 1, \omega - 1, 2\omega - 1, 2\omega, -2\omega, -2\omega - 1, -2, -\omega - 2\right\}$
    \item Input alphabet $\mathcal{B} =\mathcal{A}+ \mathcal{A}$
\end{itemize}

\noindent Extending window method:
\begin{enumerate}
    \item Phase 1 was succesfull.
The number of elements in the weight coefficient set $\mathcal{Q}$ is $17$.

    \item There is not unique weight coefficient for input $b,b,\dots,b$ for some $b\in\mathcal{B}$ for fixed length of window.

\end{enumerate}
%Eisenstein\_2-block\_4elements & \checkmark & \xmark & --\\


\section{\texorpdfstring{Penney base $\beta = -1 + \imath$}{Penney base beta = -1 + i}}
Penney base $\beta = -1 + \omega$ where $\omega=\imath$. The minimal polynomial of the base $\beta$ is $x^2 + 2x+2$. We have tested the complex, smaller complex and integer alphabet.
\subsection{ Penney\_1-block\_complex }

\label{subsec:Penney1-blockcomplex}

The alphabet $\mathcal{A} =\left\{0, 1, -1, \omega, -\omega\right\}$.

\noindent The result of the extending window method is:
\begin{enumerate}
    \item Phase 1 was succesfull.
The number of elements in the weight coefficient set $\mathcal{Q}$ is $45$.

    \item There is a unique weight coefficient for input $b,b,\dots,b$ for all $b\in\mathcal{B}$.

    \item Phase 2 was not succesfull.

\end{enumerate}
%Penney\_1-block\_complex & \ref{subsec:Penney1-blockcomplex} & \checkmark & \checkmark & \xmark \\

\subsection{ Penney\_1-block\_complex\_small }

\label{subsec:Penney1-blockcomplexsmall}

The alphabet $\mathcal{A} =\left\{0, 1, \omega\right\}$.

\noindent The result of the extending window method is:
\begin{enumerate}
    \item Phase 1 was succesfull.
The number of elements in the weight coefficient set $\mathcal{Q}$ is $22$.

    \item There is not unique weight coefficient for input $b,b,\dots,b$ for the $b= \omega + 1 $ for fixed length of window. Thus Phase 2 does not converge.

\end{enumerate}
%Penney\_1-block\_complex\_small & \ref{subsec:Penney1-blockcomplexsmall} & \checkmark & \xmark & --\\

\begin{exmp}
\textbf{ Penney\_1-block\_integer }

\label{ex:Penney1-blockinteger}

The alphabet $\mathcal{A} =\left\{0, 1, -1, 2, -2\right\}$.

The result of the extending window method is:
\begin{enumerate}
    \item Phase 1 was succesful.
The number of elements in the weight coefficient set $\mathcal{Q}$ is $47$.

    \item There is not unique weight coefficient for input $b,b,\dots,b$ for the $b= 1 $ for fixed length of window. Thus Phase 2 does not converge.

\end{enumerate}
\end{exmp}
%Penney\_1-block\_integer & \ref{ex:Penney1-blockinteger} & \checkmark & \xmark & --\\


For 2-block parallel addition, we shift from base $\beta$ to $\beta^2=-2-2\beta=-2\omega$ which has the minimal polynomial $x^{2} + 4$. We have tested the shifted alphabet $\{0,\pm 1\}+\beta \{0,\pm 1\}$.

\subsection{ Penney\_2-block\_integer }

\label{subsec:Penney2-blockinteger}

The alphabet $\mathcal{A} =\left\{0, 1, -1, \omega, -\omega, \omega - 1, -\omega + 1, \omega - 2, -\omega + 2\right\}$.

\noindent The result of the extending window method is:
\begin{enumerate}
    \item Phase 1 was succesfull.
The number of elements in the weight coefficient set $\mathcal{Q}$ is $27$.

    \item There is a unique weight coefficient for input $b,b,\dots,b$ for all $b\in\mathcal{B}$.

    \item Phase 2 was succesfull.
The lenght of window $m$ of the weight function $q$ is 5.
\end{enumerate}
%Penney\_2-block\_integer & \ref{subsec:Penney2-blockinteger} & \checkmark & \checkmark & \checkmark \\


\section{\texorpdfstring{Base $\beta = 1 + \imath$}{Base beta = 1 + i}}
The following numeration systems have the base $\beta =1 + \imath$ with the minimal polynomial $x^2-2x+2$. We have tested the complex and integer alphabet.

\subsection{ Quadratic+1-2+2-1\_block\_complex }

\label{subsec:Quadratic+1-2+2-1blockcomplex}

Parameters:
\begin{itemize}
    \item Minimal polynomial of $\omega$: $ t^{2} - 2t + 2 $
    \item Base $\beta= \omega $
    \item Minimal polynomial of base: $ x^{2} - 2x + 2 $
    \item Alphabet $\mathcal{A} =\left\{0, 1, -1, \omega - 1, -\omega + 1\right\}$
    \item Input alphabet $\mathcal{B} =\mathcal{A}+ \mathcal{A}$
\end{itemize}

\noindent Extending window method:
\begin{enumerate}
    \item Phase 1 was succesfull.
The number of elements in the weight coefficient set $\mathcal{Q}$ is $45$.

    \item There is a unique weight coefficient for input $b,b,\dots,b$ for all $b\in\mathcal{B}$.

    \item Phase 2 was succesfull.
The lenght of window $m$ of the weight function $q$ is 6.
\end{enumerate}
%Quadratic+1-2+2-1\_block\_complex & \checkmark & \checkmark & \checkmark \\

\subsection{ Quadratic+1-2+2-1\_block\_integer }

\label{subsec:Quadratic+1-2+2-1blockinteger}

The alphabet $\mathcal{A} =\left\{0, 1, -1, 2, -2\right\}$
\noindent Extending window method:
\begin{enumerate}
    \item Phase 1 was succesfull.
The number of elements in the weight coefficient set $\mathcal{Q}$ is $46$.

    \item There is not unique weight coefficient for input $b,b,\dots,b$ for the $b= 0 $ for fixed length of window. Thus Phase 2 does not converge.

\end{enumerate}
%Quadratic+1-2+2-1\_block\_integer & \ref{subsec:Quadratic+1-2+2-1blockinteger} & \checkmark & \xmark & --\\

 

\section{\texorpdfstring{Base $\beta = -2 + \imath$}{Base beta = -2 + i}}
The base  $\beta = -2 + \imath$ has the minimal polynomial $x^2+4x +5$.
\input{testedExamples/Quadratic+1+4+5_1-block_complex.tex}
% 
\section{\texorpdfstring{Base $\beta = -\frac{3}{2} + \frac{\imath \sqrt{11}}{2}$}{Base beta = -{3}/{2} + i sqrt(11)/{2}}}
The base $\beta = -\frac{3}{2} + \frac{\imath \sqrt{11}}{2}$ has the minimal polynomial  $x^2+3x +5$.
\subsection{ Quadratic+1+3+5\_1-block\_complex }

\label{subsec:Quadratic+1+3+51-blockcomplex}

The alphabet $\mathcal{A} =\left\{0, 1, -1, \omega + 1, -\omega - 1, \omega + 2, -\omega - 2, \omega + 3, -\omega - 3\right\}$.

\noindent The result of the extending window method is:
\begin{enumerate}
    \item Phase 1 was succesfull.
The number of elements in the weight coefficient set $\mathcal{Q}$ is $11$.

    \item There is not unique weight coefficient for input $b,b,\dots,b$ for the $b= 2\omega + 2 $ for fixed length of window. Thus Phase 2 does not converge.

\end{enumerate}
%Quadratic+1+3+5\_1-block\_complex & \ref{subsec:Quadratic+1+3+51-blockcomplex} & \checkmark & \xmark & --\\


\section{\texorpdfstring{Base $\beta = \frac{5}{2} + \frac{\imath \sqrt{13}}{2}$}{Base beta = {5}/{2} + i sqrt(13)/2}}
The minimal polynomial of the base $\beta = \frac{5}{2} + \frac{\imath \sqrt{13}}{2}$ is $x^2 -5x+3$.
\subsection{ Quadratic+1-5+3\_1-block\_integer }

\label{subsec:Quadratic+1-5+31-blockinteger}

Parameters:
\begin{itemize}
    \item Minimal polynomial of $\omega$: $ t^{2} - 5t + 3 $
    \item Base $\beta= \omega $
    \item Minimal polynomial of base: $ x^{2} - 5x + 3 $
    \item Alphabet $\mathcal{A} =\left\{0, 1, 2, 3, 4, 5, 6\right\}$
    \item Input alphabet $\mathcal{B} =\mathcal{A}+ \mathcal{A}$
\end{itemize}

\noindent Extending window method:
\begin{enumerate}
    \item Phase 1 was not succesfull. 

\end{enumerate}
%Quadratic+1-5+3\_1-block\_integer & \xmark & -- & --\\


\section{\texorpdfstring{Base $\beta = \frac{5}{2} + \frac{\sqrt{5}}{2}$}{Base beta = {5}/{2} + sqrt(5)/{2}}}
The minimal polynomial of the base $\beta = \frac{5}{2} + \frac{\sqrt{5}}{2}$ is $x^2 -5x+5$.
\subsection{ Quadratic+1-5+5\_1-block\_real }

\label{subsec:Quadratic+1-5+51-blockreal}

The alphabet $\mathcal{A} =\left\{0, \omega + 1, -\omega - 1, \omega + 2, -\omega - 2\right\}$.

\noindent The result of the extending window method is:
\begin{enumerate}
    \item Phase 1 was succesfull.
The number of elements in the weight coefficient set $\mathcal{Q}$ is $127$.

    \item There is not unique weight coefficient for input $b,b,\dots,b$ for the $b= 0 $ for fixed length of window. Thus Phase 2 does not converge.

\end{enumerate}
%Quadratic+1-5+5\_1-block\_real & \ref{subsec:Quadratic+1-5+51-blockreal} & \checkmark & \xmark & --\\


\section{Integer bases}
We have also tested integer bases 2, resp. 4 with the alphabets $\{0,\pm 1\}$, resp. $\{0,\pm 1, \pm 2\}$.
\subsection{ base\_2 }

\label{subsec:base2}

The alphabet $\mathcal{A} =\left\{0, 1, -1\right\}$.

\noindent The result of the extending window method is:
\begin{enumerate}
    \item Phase 1 was succesfull.
The number of elements in the weight coefficient set $\mathcal{Q}$ is $3$.

    \item There is a unique weight coefficient for input $b,b,\dots,b$ for all $b\in\mathcal{B}$.

    \item Phase 2 was succesfull.
The lenght of window $m$ of the weight function $q$ is 2.
\end{enumerate}
%base\_2 & \ref{subsec:base2} & \checkmark & \checkmark & \checkmark \\

\begin{exmp}
\textbf{ base\_4 }

\label{ex:base4}

The alphabet $\mathcal{A} =\left\{0, 1, -1, 2, -2\right\}$.

The result of the extending window method is:
\begin{enumerate}
    \item Phase 1 was succesful.
The number of elements in the weight coefficient set $\mathcal{Q}$ is $3$.

    \item There is a unique weight coefficient for input $b,b,\dots,b$ for all $b\in\mathcal{B}$.

    \item Phase 2 was succesful.
The lenght of window $m$ of the weight function $q$ is 2.
\end{enumerate}
\end{exmp}
%base\_4 & \ref{ex:base4} & \checkmark & \checkmark & \checkmark \\


\section{Cubic bases}
\subsection{ Cubic+1+1-5+5\_complex }

\label{subsec:Cubic+1+1-5+5complex}

The alphabet $\mathcal{A} =\left\{0, \omega + 1, \omega + 2, -\omega - 1, -\omega - 2\right\}$.

\noindent The result of the extending window method is:
\begin{enumerate}
    \item Phase 1 was succesfull.
The number of elements in the weight coefficient set $\mathcal{Q}$ is $345$.

    \item There is not unique weight coefficient for input $b,b,\dots,b$ for the $b= 0 $ for fixed length of window. Thus Phase 2 does not converge.

\end{enumerate}
%Cubic+1+1-5+5\_complex & \ref{subsec:Cubic+1+1-5+5complex} & \checkmark & \xmark & --\\

\begin{exmp}
\textbf{ Cubic+1+1-1+1\_complex }

\label{ex:Cubic+1+1-1+1complex}

The alphabet $\mathcal{A} =\left\{0, \omega + 1, \omega + 2, -\omega - 1, -\omega - 2\right\}$.

The result of the extending window method is:
\begin{enumerate}
    \item Phase 1 was not successful. The size of the last intermediate weight coefficients set was 30 636 when stopped. 

\end{enumerate}
\end{exmp}
%Cubic+1+1-1+1\_complex & \ref{ex:Cubic+1+1-1+1complex} & yes & no & \xmark & -- & --\\
