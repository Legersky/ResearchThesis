We have tested several bases with different alphabets. The bases are chosen such they have no conjugates of modulo 1 according to Theorem \ref{thm:suffConjugates}. The sizes of the integer alphabets are given by the lower bound from Theorem \ref{thm:lowerBoundAlphabet}. Since this theorem assumes only an integer alphabet, we have tested also non-integer alphabets of smaller size. The input alphabet $\B$ is always $\A+ \A$.  

 Table \ref{tbl:results} summarizes the tested numeration systems which are described in the sections below. It is marked in the first column of the table whether the alphabet satisfies the necessary conditions described in Section \ref{sec:alphabet}. The second column says if the sufficient condition given by Theorem \ref{thm:suffCondPhase1} holds. The third one shows whether Phase 1 was successful (\checkmark) or not (\xmark) for the given numeration system. By not successful we mean that the size of the intermediate weight coefficients sets was steeply increasing and the computation was interrupted because of hardware limits. The fourth column is the control of the necessary condition for the convergence of Phase 2 by Algorithm \ref{alg:oneletterSets}, i.e. if there is the output of the weight function $q$ for input digits $b,b,\dots,b$ for all $b\in\B$. The results of Phase 2 are in the last column. Again, (\xmark) means that the computation was stopped because of hardware limits. Notice, that the next step of the extending window method is not processed without the previous one~(--).

We see that Phase 1 was successful in all cases when the sufficient condition holds, as we expected. Moreover, we have examples when it converges without the sufficient condition. We remark that the absolute values of all conjugates of the base are greater than 1 in Examples \ref{ex:Quadratic+1-5+51-blockreal} and \ref{ex:Cubic+1+1-5+5complex}, whereas there is a conjugate whose modulus is smaller than 1 in Examples \ref{ex:Quadratic+1-5+31-blockinteger} and \ref{ex:Cubic+1+1-1+1complex}.


% The implemention provides weight function with the same length of window as manually found ones for Examples \ref{ex:Eisenstein1-blockcomplex}, \ref{ex:Penney1-blockcomplex}, \ref{ex:Quadratic+1-2+21-blockcomplex} and \ref{ex:Quadratic+1+3+51-blockcomplex}. It also provides parallel addition algorithm for integer bases.
 
The possible reasoning of non-convergence of Phase 2 for the quadratic bases with integer alphabet is that the coefficients of the rewriting rule $x-\beta$ are not integers.  

The complete results including log files and images can be found on the attached CD  or \url{https://github.com/Legersky/ParallelAddition}.


\begin{table}[!htb]
\centering
  \begin{tabular}{l r|c cc c c}
      Name &  Ex. & Alph. & Suff. c. & Phase 1 & Necess. c. & Phase 2 \\ \hline
      Eisenstein\_1-block\_complex & \ref{ex:Eisenstein1-blockcomplex} & yes & yes & \checkmark & \checkmark & \checkmark \\
      Eisenstein\_1-block\_integer & \ref{ex:Eisenstein1-blockinteger} & yes & yes & \checkmark & \xmark & --\\
      Eisenstein\_1-block\_small\_complex & \ref{ex:Eisenstein1-blocksmallcomplex} &no & -- & -- & -- & -- \\
      Eisenstein\_2-block & \ref{ex:Eisenstein2-block} &no & -- & -- & -- & -- \\
      Eisenstein\_2-block\_4elements & \ref{ex:Eisenstein2-block4elements} & yes & yes & \checkmark & \xmark & --\\
      \hline
      Penney\_1-block\_complex & \ref{ex:Penney1-blockcomplex} & yes & yes & \checkmark & \checkmark & \checkmark \\
      Penney\_1-block\_small\_complex & \ref{ex:Penney1-blockcomplexsmall} &no & -- & -- & -- & -- \\
      Penney\_1-block\_integer & \ref{ex:Penney1-blockinteger} & yes & yes & \checkmark & \xmark & --\\
      Penney\_2-block\_integer & \ref{ex:Penney2-blockinteger} & yes & yes & \checkmark & \checkmark & \checkmark \\
      \hline
      Quadratic+1-2+2\_1-block\_complex & \ref{ex:Quadratic+1-2+21-blockcomplex} & yes & yes& \checkmark & \checkmark & \checkmark \\
        Quadratic+1-2+2\_1-block\_integer & \ref{ex:Quadratic+1-2+21-blockinteger} & yes & yes & \checkmark & \xmark & --\\
      \hline
      Quadratic+1+4+5\_1-block\_complex & \ref{ex:Quadratic+1+4+51-blockcomplex} & yes & yes & \checkmark & \checkmark & \checkmark \\ 
      \hline
      Quadratic+1+3+5\_1-block\_complex & \ref{ex:Quadratic+1+3+51-blockcomplex} & yes & yes & \checkmark & \checkmark & \xmark \\
      \hline
      Quadratic+1-5+3\_1-block\_integer & \ref{ex:Quadratic+1-5+31-blockinteger} & yes & no & \xmark & -- & --\\
      \hline
      Quadratic+1-5+5\_1-block\_real & \ref{ex:Quadratic+1-5+51-blockreal} & yes & no & \checkmark & \xmark & --\\
      \hline
      base\_2 & \ref{ex:base2} & yes & yes & \checkmark & \checkmark & \checkmark \\
        base\_4 & \ref{ex:base4} & yes & yes & \checkmark & \checkmark & \checkmark \\
      \hline
      Cubic+1+1-1+1\_complex & \ref{ex:Cubic+1+1-1+1complex} & yes & no & \xmark & -- & --\\
        Cubic+1+1-5+5\_complex & \ref{ex:Cubic+1+1-5+5complex} & yes & no & \checkmark & \xmark & --\\
  \end{tabular}
  \caption{Results of extending window method.}
  \label{tbl:results}
\end{table} 

\newpage
\section{\texorpdfstring{Eisenstein base $\beta = -\frac{3}{2} + \frac{\imath \sqrt{3}}{2}$}{Eisenstein base beta = -3/2 + i sqrt(3)/2}}
Eisenstein base $\beta$ equals $\omega - 1$ with $\omega =-\frac{1}{2} + \frac{\imath \sqrt{3}}{2}$. The minimal polynomial of the generator $\omega$ is $x^2 + x+1$ and the minimal polynomial of the base $\beta$ is $x^2 + 3x+3$. Thus, the lower bound for the size of the integer alphabet given by Theorem \ref{thm:lowerBoundAlphabet} is 7. We have tested the base with three alphabets -- two complex and one integer.
  
\subsection{ Eisenstein\_1-block\_complex }

\label{subsec:Eisenstein1-blockcomplex}

Parameters:
\begin{itemize}
    \item Minimal polynomial of $\omega$: $ t^{2} + t + 1 $
    \item Base $\beta= \omega - 1 $
    \item Minimal polynomial of base: $ x^{2} + 3x + 3 $
    \item Alphabet $\mathcal{A} =\left\{0, 1, -1, \omega, -\omega, -\omega - 1, \omega + 1\right\}$
    \item Input alphabet $\mathcal{B} =\mathcal{A}+ \mathcal{A}$
\end{itemize}

\noindent Extending window method:
\begin{enumerate}
    \item Phase 1 was succesfull.
The number of elements in the weight coefficient set $\mathcal{Q}$ is $19$.

    \item There is a unique weight coefficient for input $b,b,\dots,b$ for all $b\in\mathcal{B}$.

    \item Phase 2 was succesfull.
The lenght of window $m$ of the weight function $q$ is 3.
\end{enumerate}
%Eisenstein\_1-block\_complex & \checkmark & \checkmark & \checkmark \\

\begin{exmp}
\textbf{ Eisenstein\_1-block\_small\_complex }

\label{ex:Eisenstein1-blocksmallcomplex}

The alphabet $\mathcal{A} =\left\{0, 1, \omega, \omega + 1\right\}$.

The result of the extending window method is:
\begin{enumerate}
    \item Phase 1 was succesful.
The number of elements in the weight coefficient set $\mathcal{Q}$ is $17$.

    \item There is not unique weight coefficient for input $b,b,\dots,b$ for the $b= \omega + 2 $ for fixed length of window. Thus Phase 2 does not converge.

\end{enumerate}
\end{exmp}
%Eisenstein\_1-block\_small\_complex & \ref{ex:Eisenstein1-blocksmallcomplex} & \checkmark & \xmark & --\\

\begin{exmp}
\textbf{ Eisenstein\_1-block\_integer }

\label{ex:Eisenstein1-blockinteger}

The alphabet $\mathcal{A} =\left\{0, 1, -1, 2, -2, 3, -3\right\}$.

The result of the extending window method is:
\begin{enumerate}
    \item Phase 1 was successful.
The number of elements in the weight coefficient set $\mathcal{Q}$ is $53$.

    \item There is not unique weight coefficient for input $b,b,\dots,b$ for $b\in\left\{1, 2, 3, 5, 6, -6, -4, -3\right\}$ for fixed length of window. Thus Phase 2 does not converge.

\end{enumerate}
\end{exmp}
%Eisenstein\_1-block\_integer & \ref{ex:Eisenstein1-blockinteger} & yes & yes & \checkmark & \xmark & --\\


We may also study so-called 2-block parallel addition. Roughly speaking, we consider two digits after each other in a $(\beta,\A)$-representation as one digit of the $(\beta^2,\A+\beta\A)$-representation of the same number. So we shift from base $\beta$ to $\beta^2=-3-3\beta=-3\omega$ which has the minimal polynomial $x^2-3x+9$. We have tested the shifted alphabets $\{0,\pm 1\}+\beta \{0,\pm 1\}$ and $\{0,1, \omega, \omega +1\}+\beta \{0,1, \omega, \omega +1\}$.
  
\begin{exmp}
\textbf{ Eisenstein\_2-block }

\label{ex:Eisenstein2-block}

The alphabet $\mathcal{A} =\left\{0, 1, -1, \omega, -\omega, \omega - 1, -\omega + 1, \omega - 2, -\omega + 2\right\}$.

The elements $ \left\{2\omega - 1, 2\omega, \omega + 1, -\omega - 1, -2\omega, -2\omega + 1\right\} $ have no representative  modulo $\beta-1$ in the alphabet $\mathcal{A}$.
%Eisenstein\_2-block & \ref{ex:Eisenstein2-block} &no & -- & -- & -- & -- &\
\end{exmp}
\begin{exmp}
\textbf{ Eisenstein\_2-block\_4elements }

\label{ex:Eisenstein2-block4elements}

The alphabet $\mathcal{A} =\left\{0, 1, -1, \omega, -\omega, \omega + 1, -\omega - 1, \omega - 1, 2\omega - 1, 2\omega, -2\omega, -2\omega - 1, -2, -\omega - 2\right\}$.

The result of the extending window method is:
\begin{enumerate}
    \item Phase 1 was successful.
The number of elements in the weight coefficient set $\mathcal{Q}$ is $17$.

    \item There is not unique weight coefficient for input $b,b,\dots,b$ for $b\in\left\{2\omega - 1, \omega + 1, -2\omega, -\omega - 2, -4\right\}$ for fixed length of window. Thus Phase 2 does not converge.

\end{enumerate}
\end{exmp}
%Eisenstein\_2-block\_4elements & \ref{ex:Eisenstein2-block4elements} & yes & yes & \checkmark & \xmark & --\\


\section{\texorpdfstring{Penney base $\beta = -1 + \imath$}{Penney base beta = -1 + i}}
Penney base $\beta = -1 + \omega$ where $\omega=\imath$. The minimal polynomial of the base $\beta$ is $x^2 + 2x+2$. We have tested the base with three alphabets -- two complex and one integer. The lower bound for the size of the integer alphabet is 5.
\begin{exmp}
\textbf{ Penney\_1-block\_complex }

\label{ex:Penney1-blockcomplex}

The alphabet $\mathcal{A} =\left\{0, 1, -1, \omega, -\omega\right\}$.

The result of the extending window method is:
\begin{enumerate}
    \item Phase 1 was succesful.
The number of elements in the weight coefficient set $\mathcal{Q}$ is $45$.

    \item There is a unique weight coefficient for input $b,b,\dots,b$ for all $b\in\mathcal{B}$.

    \item Phase 2 was succesful.
The length of window $m$ of the weight function $q$ is 6.
\end{enumerate}
\end{exmp}
%Penney\_1-block\_complex & \ref{ex:Penney1-blockcomplex} & \checkmark & \checkmark & \checkmark \\

\begin{exmp}
\textbf{ Penney\_1-block\_complex\_small }

\label{ex:Penney1-blockcomplexsmall}

The alphabet $\mathcal{A} =\left\{0, 1, \omega\right\}$.

The result of the extending window method is:
\begin{enumerate}
    \item Phase 1 was succesful.
The number of elements in the weight coefficient set $\mathcal{Q}$ is $22$.

    \item There is not unique weight coefficient for input $b,b,\dots,b$ for the $b= \omega + 1 $ for fixed length of window. Thus Phase 2 does not converge.

\end{enumerate}
\end{exmp}
%Penney\_1-block\_complex\_small & \ref{ex:Penney1-blockcomplexsmall} & \checkmark & \xmark & --\\

\begin{exmp}
\textbf{ Penney\_1-block\_integer }

\label{ex:Penney1-blockinteger}

The alphabet $\mathcal{A} =\left\{0, 1, -1, 2, -2\right\}$.

The result of the extending window method is:
\begin{enumerate}
    \item Phase 1 was successful.
The number of elements in the weight coefficient set $\mathcal{Q}$ is $47$.

    \item There is not a unique weight coefficient for input $b,b,\dots,b$ for $b\in\left\{2, 3, 4, -4, -3, -2\right\}$ for some fixed length of window. Thus Phase 2 does not converge.

\end{enumerate}
\end{exmp}
%Penney\_1-block\_integer & \ref{ex:Penney1-blockinteger} & yes & yes & \checkmark & \xmark & --\\


For 2-block parallel addition, we shift from base $\beta$ to $\beta^2=-2-2\beta=-2\omega$ which has the minimal polynomial $x^{2} + 4$. We have tested the shifted alphabet $\{0,\pm 1\}+\beta \{0,\pm 1\}$.

\begin{exmp}
\textbf{ Penney\_2-block\_integer }

\label{ex:Penney2-blockinteger}

The alphabet $\mathcal{A} =\left\{0, 1, -1, \omega, -\omega, \omega - 1, -\omega + 1, \omega - 2, -\omega + 2\right\}$.

The result of the extending window method is:
\begin{enumerate}
    \item Phase 1 was succesful.
The number of elements in the weight coefficient set $\mathcal{Q}$ is $27$.

    \item There is a unique weight coefficient for input $b,b,\dots,b$ for all $b\in\mathcal{B}$.

    \item Phase 2 was succesful.
The lenght of window $m$ of the weight function $q$ is 5.
\end{enumerate}
\end{exmp}
%Penney\_2-block\_integer & \ref{ex:Penney2-blockinteger} & \checkmark & \checkmark & \checkmark \\


\section{\texorpdfstring{Base $\beta = 1 + \imath$}{Base beta = 1 + i}}
The following numeration systems have the base $\beta =1 + \imath$ with the minimal polynomial $x^2-2x+2$. We have tested the complex and integer alphabet. 

\begin{exmp}
\textbf{ Quadratic+1-2+2\_1-block\_complex }

\label{ex:Quadratic+1-2+21-blockcomplex}

The alphabet $\mathcal{A} =\left\{0, 1, -1, \omega - 1, -\omega + 1\right\}$.

The result of the extending window method is:
\begin{enumerate}
    \item Phase 1 was succesful.
The number of elements in the weight coefficient set $\mathcal{Q}$ is $45$.

    \item There is a unique weight coefficient for input $b,b,\dots,b$ for all $b\in\mathcal{B}$.

    \item Phase 2 was succesful.
The lenght of window $m$ of the weight function $q$ is 6.
\end{enumerate}
\end{exmp}
%Quadratic+1-2+2\_1-block\_complex & \ref{ex:Quadratic+1-2+21-blockcomplex} & \checkmark & \checkmark & \checkmark \\

\begin{exmp}
\textbf{ Quadratic+1-2+2\_1-block\_integer }

\label{ex:Quadratic+1-2+21-blockinteger}

The alphabet $\mathcal{A} =\left\{0, 1, -1, 2, -2\right\}$.

The result of the extending window method is:
\begin{enumerate}
    \item Phase 1 was successful.
The number of elements in the weight coefficient set $\mathcal{Q}$ is $46$.

    \item There is not a unique weight coefficient for input $b,b,\dots,b$ for $b\in\left\{2, -1, -2\right\}$ for some fixed length of window. Thus Phase 2 does not converge.

\end{enumerate}
\end{exmp}
%Quadratic+1-2+2\_1-block\_integer & \ref{ex:Quadratic+1-2+21-blockinteger} & yes & yes & \checkmark & \xmark & --\\

 

\section{\texorpdfstring{Base $\beta = -2 + \imath$}{Base beta = -2 + i}}
The base  $\beta = -2 + \imath$ has the minimal polynomial $x^2+4x +5$.
\subsection{ Quadratic\_1+4+5\_1-block\_complex }

\label{subsec:Quadratic+1+4+51-blockcomplex}

The alphabet $\mathcal{A} =\left\{0, 1, -1, \omega, -\omega, \omega + 1, -\omega - 1, \omega - 1, -\omega - 2, -2\right\}$.

\noindent The result of the extending window method is:
\begin{enumerate}
    \item Phase 1 was succesfull.
The number of elements in the weight coefficient set $\mathcal{Q}$ is $17$.

    \item There is a unique weight coefficient for input $b,b,\dots,b$ for all $b\in\mathcal{B}$.

    \item Phase 2 was succesfull.
The lenght of window $m$ of the weight function $q$ is 3.
\end{enumerate}
%Quadratic\_1+4+5\_1-block\_complex & \ref{subsec:Quadratic1+4+51-blockcomplex} & \checkmark & \checkmark & \checkmark \\

% 
\section{\texorpdfstring{Base $\beta = -\frac{3}{2} + \frac{\imath \sqrt{11}}{2}$}{Base beta = -{3}/{2} + i sqrt(11)/{2}}}
The base $\beta = -\frac{3}{2} + \frac{\imath \sqrt{11}}{2}$ has the minimal polynomial  $x^2+3x +5$.
\begin{exmp}
\textbf{ Quadratic+1+3+5\_1-block\_complex }

\label{ex:Quadratic+1+3+51-blockcomplex}

The alphabet $\mathcal{A} =\left\{0, 1, -1, \omega + 1, -\omega - 1, \omega + 2, -\omega - 2, \omega + 3, -\omega - 3\right\}$.

The result of the extending window method is:
\begin{enumerate}
    \item Phase 1 was successful.
The number of elements in the weight coefficient set $\mathcal{Q}$ is $11$.

    \item There is a unique weight coefficient for input $b,b,\dots,b$ for all $b\in\mathcal{B}$.

    \item Phase 2 was not successful. Maximal tested length of window was 10.

\end{enumerate}
\end{exmp}
%Quadratic+1+3+5\_1-block\_complex & \ref{ex:Quadratic+1+3+51-blockcomplex} & yes & yes & \checkmark & \checkmark & \xmark \\


\section{\texorpdfstring{Base $\beta = \frac{5}{2} + \frac{\imath \sqrt{13}}{2}$}{Base beta = {5}/{2} + i sqrt(13)/2}}
The minimal polynomial of the base $\beta = \frac{5}{2} + \frac{\imath \sqrt{13}}{2}$ is $x^2 -5x+3$.
\subsection{ Quadratic+1-5+3\_1-block\_integer }

\label{subsec:Quadratic+1-5+31-blockinteger}

Parameters:
\begin{itemize}
    \item Minimal polynomial of $\omega$: $ t^{2} - 5t + 3 $
    \item Base $\beta= \omega $
    \item Minimal polynomial of base: $ x^{2} - 5x + 3 $
    \item Alphabet $\mathcal{A} =\left\{0, 1, 2, 3, 4, 5, 6\right\}$
    \item Input alphabet $\mathcal{B} =\mathcal{A}+ \mathcal{A}$
\end{itemize}

\noindent Extending window method:
\begin{enumerate}
    \item Phase 1 was not succesfull. 

\end{enumerate}
%Quadratic+1-5+3\_1-block\_integer & \xmark & -- & --\\


\section{\texorpdfstring{Base $\beta = \frac{5}{2} + \frac{\sqrt{5}}{2}$}{Base beta = {5}/{2} + sqrt(5)/{2}}}
The minimal polynomial of the base $\beta = \frac{5}{2} + \frac{\sqrt{5}}{2}$ is $x^2 -5x+5$.
\begin{exmp}
\textbf{ Quadratic+1-5+5\_1-block\_real }

\label{ex:Quadratic+1-5+51-blockreal}

The alphabet $\mathcal{A} =\left\{0, \omega + 1, -\omega - 1, \omega + 2, -\omega - 2\right\}$.

The result of the extending window method is:
\begin{enumerate}
    \item Phase 1 was succesful.
The number of elements in the weight coefficient set $\mathcal{Q}$ is $127$.

    \item There is not unique weight coefficient for input $b,b,\dots,b$ for the $b= 0 $ for fixed length of window. Thus Phase 2 does not converge.

\end{enumerate}
\end{exmp}
%Quadratic+1-5+5\_1-block\_real & \ref{ex:Quadratic+1-5+51-blockreal} & \checkmark & \xmark & --\\


\section{Integer bases}
We have also tested integer bases 2, resp. 4 with the alphabets $\{0,\pm 1\}$, resp. $\{0,\pm 1, \pm 2\}$. Since the minimal polynomial is $x-2$, resp. $x-4$, the minimal size of the integer alphabet given by Theorem \ref{thm:lowerBoundAlphabet} is $|1-2|+2=3$, resp. 4. 
\subsection{ base\_2 }

\label{subsec:base2}

The alphabet $\mathcal{A} =\left\{0, 1, -1\right\}$.

\noindent The result of the extending window method is:
\begin{enumerate}
    \item Phase 1 was succesfull.
The number of elements in the weight coefficient set $\mathcal{Q}$ is $3$.

    \item There is a unique weight coefficient for input $b,b,\dots,b$ for all $b\in\mathcal{B}$.

    \item Phase 2 was succesfull.
The lenght of window $m$ of the weight function $q$ is 2.
\end{enumerate}
%base\_2 & \ref{subsec:base2} & \checkmark & \checkmark & \checkmark \\

\subsection{ base\_4 }

\label{subsec:base4}

The alphabet $\mathcal{A} =\left\{0, 1, -1, 2, -2\right\}$.

\noindent The result of the extending window method is:
\begin{enumerate}
    \item Phase 1 was succesfull.
The number of elements in the weight coefficient set $\mathcal{Q}$ is $3$.

    \item There is a unique weight coefficient for input $b,b,\dots,b$ for all $b\in\mathcal{B}$.

    \item Phase 2 was succesfull.
The lenght of window $m$ of the weight function $q$ is 2.
\end{enumerate}
%base\_4 & \ref{subsec:base4} & \checkmark & \checkmark & \checkmark \\


\section{Cubic bases}
The base is chosen as the zero of cubic polynomial $x^3+x-5x+5$, resp. $x^3+x-x+1$, which is the greatest one in modulus. 
\subsection{ Cubic+1+1-5+5\_complex }

\label{subsec:Cubic+1+1-5+5complex}

The alphabet $\mathcal{A} =\left\{0, \omega + 1, \omega + 2, -\omega - 1, -\omega - 2\right\}$.

\noindent The result of the extending window method is:
\begin{enumerate}
    \item Phase 1 was succesfull.
The number of elements in the weight coefficient set $\mathcal{Q}$ is $345$.

    \item There is not unique weight coefficient for input $b,b,\dots,b$ for the $b= 0 $ for fixed length of window. Thus Phase 2 does not converge.

\end{enumerate}
%Cubic+1+1-5+5\_complex & \ref{subsec:Cubic+1+1-5+5complex} & \checkmark & \xmark & --\\

\begin{exmp}
\textbf{ Cubic+1+1-1+1\_complex }

\label{ex:Cubic+1+1-1+1complex}

The alphabet $\mathcal{A} =\left\{0, \omega + 1, \omega + 2, -\omega - 1, -\omega - 2\right\}$.

The result of the extending window method is:
\begin{enumerate}
    \item Phase 1 was not successful. The size of the last intermediate weight coefficients set was 30 636 when stopped. 

\end{enumerate}
\end{exmp}
%Cubic+1+1-1+1\_complex & \ref{ex:Cubic+1+1-1+1complex} & yes & no & \xmark & -- & --\\
