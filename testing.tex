We have tested several bases with different alphabets. Table \ref{tbl:results} summarizes the tested numeration systems which are described in the sections below. The first column of the table shows whether Phase 1 was successful (\checkmark) or not (\xmark) for the given numeration system. The second column is the control of the necessary condition for the convergence of Phase 2, i.e. if there is output of the weight function $q$ for input digits $b,b,\dots,b$ for all $b\in\B$. The results of Phase 2 are in the last column.
\begin{table}
\centering
  \begin{tabular}{l |c c c}
      Name & Phase 1 & Necessary condition & Phase 2 \\ \hline
      Eisenstein\_1-block\_complex & \checkmark & \checkmark & \checkmark \\
      Eisenstein\_1-block\_integer & \checkmark & \xmark & --
  \end{tabular}
  \caption{Results of extending window method.}
  \label{tbl:results}
\end{table} 
\section{Eisenstein numeration system}

\subsection{ Eisenstein\_1-block\_complex }

\label{subsec:Eisenstein1-blockcomplex}

Parameters:
\begin{itemize}
    \item Minimal polynomial of $\omega$: $ t^{2} + t + 1 $
    \item Base $\beta= \omega - 1 $
    \item Minimal polynomial of base: $ x^{2} + 3x + 3 $
    \item Alphabet $\mathcal{A} =\left\{0, 1, -1, \omega, -\omega, -\omega - 1, \omega + 1\right\}$
    \item Input alphabet $\mathcal{B} =\mathcal{A}+ \mathcal{A}$
\end{itemize}

\noindent Extending window method:
\begin{enumerate}
    \item Phase 1 was succesfull.
The number of elements in the weight coefficient set $\mathcal{Q}$ is $19$.

    \item There is a unique weight coefficient for input $b,b,\dots,b$ for all $b\in\mathcal{B}$.

    \item Phase 2 was succesfull.
The lenght of window $m$ of the weight function $q$ is 3.
\end{enumerate}
%Eisenstein\_1-block\_complex & \checkmark & \checkmark & \checkmark \\

\begin{exmp}
\textbf{ Eisenstein\_1-block\_integer }

\label{ex:Eisenstein1-blockinteger}

The alphabet $\mathcal{A} =\left\{0, 1, -1, 2, -2, 3, -3\right\}$.

The result of the extending window method is:
\begin{enumerate}
    \item Phase 1 was successful.
The number of elements in the weight coefficient set $\mathcal{Q}$ is $53$.

    \item There is not unique weight coefficient for input $b,b,\dots,b$ for $b\in\left\{1, 2, 3, 5, 6, -6, -4, -3\right\}$ for fixed length of window. Thus Phase 2 does not converge.

\end{enumerate}
\end{exmp}
%Eisenstein\_1-block\_integer & \ref{ex:Eisenstein1-blockinteger} & yes & yes & \checkmark & \xmark & --\\

% \begin{exmp}
\textbf{ Eisenstein\_1-block\_small\_complex }

\label{ex:Eisenstein1-blocksmallcomplex}

The alphabet $\mathcal{A} =\left\{0, 1, \omega, \omega + 1\right\}$.

The result of the extending window method is:
\begin{enumerate}
    \item Phase 1 was succesful.
The number of elements in the weight coefficient set $\mathcal{Q}$ is $17$.

    \item There is not unique weight coefficient for input $b,b,\dots,b$ for the $b= \omega + 2 $ for fixed length of window. Thus Phase 2 does not converge.

\end{enumerate}
\end{exmp}
%Eisenstein\_1-block\_small\_complex & \ref{ex:Eisenstein1-blocksmallcomplex} & \checkmark & \xmark & --\\

% \begin{exmp}
\textbf{ Eisenstein\_2-block }

\label{ex:Eisenstein2-block}

The alphabet $\mathcal{A} =\left\{0, 1, -1, \omega, -\omega, \omega - 1, -\omega + 1, \omega - 2, -\omega + 2\right\}$.

The elements $ \left\{2\omega - 1, 2\omega, \omega + 1, -\omega - 1, -2\omega, -2\omega + 1\right\} $ have no representative  modulo $\beta-1$ in the alphabet $\mathcal{A}$.
%Eisenstein\_2-block & \ref{ex:Eisenstein2-block} &no & -- & -- & -- & -- &\
\end{exmp}
% \begin{exmp}
\textbf{ Eisenstein\_2-block\_4elements }

\label{ex:Eisenstein2-block4elements}

The alphabet $\mathcal{A} =\left\{0, 1, -1, \omega, -\omega, \omega + 1, -\omega - 1, \omega - 1, 2\omega - 1, 2\omega, -2\omega, -2\omega - 1, -2, -\omega - 2\right\}$.

The result of the extending window method is:
\begin{enumerate}
    \item Phase 1 was successful.
The number of elements in the weight coefficient set $\mathcal{Q}$ is $17$.

    \item There is not unique weight coefficient for input $b,b,\dots,b$ for $b\in\left\{2\omega - 1, \omega + 1, -2\omega, -\omega - 2, -4\right\}$ for fixed length of window. Thus Phase 2 does not converge.

\end{enumerate}
\end{exmp}
%Eisenstein\_2-block\_4elements & \ref{ex:Eisenstein2-block4elements} & yes & yes & \checkmark & \xmark & --\\


\subsection{Penney base $\beta = -1 + \imath$ with $\imath = \exp{\frac{2 \pi \imath}{4}}$}

% \begin{exmp}
\textbf{ Penney\_1-block\_complex }

\label{ex:Penney1-blockcomplex}

The alphabet $\mathcal{A} =\left\{0, 1, -1, \omega, -\omega\right\}$.

The result of the extending window method is:
\begin{enumerate}
    \item Phase 1 was succesful.
The number of elements in the weight coefficient set $\mathcal{Q}$ is $45$.

    \item There is a unique weight coefficient for input $b,b,\dots,b$ for all $b\in\mathcal{B}$.

    \item Phase 2 was succesful.
The length of window $m$ of the weight function $q$ is 6.
\end{enumerate}
\end{exmp}
%Penney\_1-block\_complex & \ref{ex:Penney1-blockcomplex} & \checkmark & \checkmark & \checkmark \\

% \begin{exmp}
\textbf{ Penney\_1-block\_complex\_small }

\label{ex:Penney1-blockcomplexsmall}

The alphabet $\mathcal{A} =\left\{0, 1, \omega\right\}$.

The result of the extending window method is:
\begin{enumerate}
    \item Phase 1 was succesful.
The number of elements in the weight coefficient set $\mathcal{Q}$ is $22$.

    \item There is not unique weight coefficient for input $b,b,\dots,b$ for the $b= \omega + 1 $ for fixed length of window. Thus Phase 2 does not converge.

\end{enumerate}
\end{exmp}
%Penney\_1-block\_complex\_small & \ref{ex:Penney1-blockcomplexsmall} & \checkmark & \xmark & --\\

% \begin{exmp}
\textbf{ Penney\_1-block\_integer }

\label{ex:Penney1-blockinteger}

The alphabet $\mathcal{A} =\left\{0, 1, -1, 2, -2\right\}$.

The result of the extending window method is:
\begin{enumerate}
    \item Phase 1 was successful.
The number of elements in the weight coefficient set $\mathcal{Q}$ is $47$.

    \item There is not a unique weight coefficient for input $b,b,\dots,b$ for $b\in\left\{2, 3, 4, -4, -3, -2\right\}$ for some fixed length of window. Thus Phase 2 does not converge.

\end{enumerate}
\end{exmp}
%Penney\_1-block\_integer & \ref{ex:Penney1-blockinteger} & yes & yes & \checkmark & \xmark & --\\

% \begin{exmp}
\textbf{ Penney\_2-block\_integer }

\label{ex:Penney2-blockinteger}

The alphabet $\mathcal{A} =\left\{0, 1, -1, \omega, -\omega, \omega - 1, -\omega + 1, \omega - 2, -\omega + 2\right\}$.

The result of the extending window method is:
\begin{enumerate}
    \item Phase 1 was succesful.
The number of elements in the weight coefficient set $\mathcal{Q}$ is $27$.

    \item There is a unique weight coefficient for input $b,b,\dots,b$ for all $b\in\mathcal{B}$.

    \item Phase 2 was succesful.
The lenght of window $m$ of the weight function $q$ is 5.
\end{enumerate}
\end{exmp}
%Penney\_2-block\_integer & \ref{ex:Penney2-blockinteger} & \checkmark & \checkmark & \checkmark \\


\subsection{Base $\beta = 1 + \imath$}
% \input{testedExamples/Quadratic+1-2+2-block_complex.tex}
% \input{testedExamples/Quadratic+1-2+2-block_small_complex.tex}
% \input{testedExamples/Quadratic+1-2+2-block_integer.tex}
% 
% \subsection{Base $\beta = -\frac{3}{2} + \frac{\imath \sqrt{11}}{2}$}
% \begin{exmp}
\textbf{ Quadratic+1+3+5\_1-block\_complex }

\label{ex:Quadratic+1+3+51-blockcomplex}

The alphabet $\mathcal{A} =\left\{0, 1, -1, \omega + 1, -\omega - 1, \omega + 2, -\omega - 2, \omega + 3, -\omega - 3\right\}$.

The result of the extending window method is:
\begin{enumerate}
    \item Phase 1 was successful.
The number of elements in the weight coefficient set $\mathcal{Q}$ is $11$.

    \item There is a unique weight coefficient for input $b,b,\dots,b$ for all $b\in\mathcal{B}$.

    \item Phase 2 was not successful. Maximal tested length of window was 10.

\end{enumerate}
\end{exmp}
%Quadratic+1+3+5\_1-block\_complex & \ref{ex:Quadratic+1+3+51-blockcomplex} & yes & yes & \checkmark & \checkmark & \xmark \\

% 
% \subsection{Base $\beta = -2 + \imath$}
% \subsection{ Quadratic\_1+4+5\_1-block\_complex }

\label{subsec:Quadratic1+4+51-blockcomplex}

Parameters:
\begin{itemize}
    \item Minimal polynomial of $\omega$: $ t^{2} + 1 $
    \item Base $\beta= \omega - 2 $
    \item Minimal polynomial of base: $ x^{2} + 4x + 5 $
    \item Alphabet $\mathcal{A} =\left\{0, 1, -1, \omega, -\omega, \omega + 1, -\omega - 1, \omega - 1, -\omega - 2, -2\right\}$
    \item Input alphabet $\mathcal{B} =\mathcal{A}+ \mathcal{A}$
\end{itemize}

\noindent Extending window method:
\begin{enumerate}
    \item Phase 1 was succesfull.
The number of elements in the weight coefficient set $\mathcal{Q}$ is $17$.

    \item There is a unique weight coefficient for input $b,b,\dots,b$ for all $b\in\mathcal{B}$.

    \item Phase 2 was succesfull.
The lenght of window $m$ of the weight function $q$ is 3.
\end{enumerate}
%Quadratic\_1+4+5\_1-block\_complex & \checkmark & \checkmark & \checkmark \\

% 
% \section{Integer bases}
% \subsection{ base\_2 }

\label{subsec:base2}

The alphabet $\mathcal{A} =\left\{0, 1, -1\right\}$.

\noindent The result of the extending window method is:
\begin{enumerate}
    \item Phase 1 was succesfull.
The number of elements in the weight coefficient set $\mathcal{Q}$ is $3$.

    \item There is a unique weight coefficient for input $b,b,\dots,b$ for all $b\in\mathcal{B}$.

    \item Phase 2 was succesfull.
The lenght of window $m$ of the weight function $q$ is 2.
\end{enumerate}
%base\_2 & \ref{subsec:base2} & \checkmark & \checkmark & \checkmark \\

% \subsection{ base\_4 }

\label{subsec:base4}

The alphabet $\mathcal{A} =\left\{0, 1, -1, 2, -2\right\}$.

\noindent The result of the extending window method is:
\begin{enumerate}
    \item Phase 1 was succesfull.
The number of elements in the weight coefficient set $\mathcal{Q}$ is $3$.

    \item There is a unique weight coefficient for input $b,b,\dots,b$ for all $b\in\mathcal{B}$.

    \item Phase 2 was succesfull.
The lenght of window $m$ of the weight function $q$ is 2.
\end{enumerate}
%base\_4 & \ref{subsec:base4} & \checkmark & \checkmark & \checkmark \\

% 
% \subsection{Base ???}
% \subsection{ Quadratic+1-5+3\_1-block\_integer }

\label{subsec:Quadratic+1-5+31-blockinteger}

Parameters:
\begin{itemize}
    \item Minimal polynomial of $\omega$: $ t^{2} - 5t + 3 $
    \item Base $\beta= \omega $
    \item Minimal polynomial of base: $ x^{2} - 5x + 3 $
    \item Alphabet $\mathcal{A} =\left\{0, 1, 2, 3, 4, 5, 6\right\}$
    \item Input alphabet $\mathcal{B} =\mathcal{A}+ \mathcal{A}$
\end{itemize}

\noindent Extending window method:
\begin{enumerate}
    \item Phase 1 was not succesfull. 

\end{enumerate}
%Quadratic+1-5+3\_1-block\_integer & \xmark & -- & --\\
