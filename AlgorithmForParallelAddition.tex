The constructor of \emph{AlgorithmForParallelAddition}:
\begin{method}{\_\_init\_\_}{minPol\_str, embd, alphabet, base, name='NumerationSystem', inputAlphabet='', printLog=True, printLogLatex=False, verbose=0}
Take \var{minPol\_str} which is symbolic expression in the variable $x$ of minimal polynomial $p$. The closest root of  \var{minPol\_str} to \var{embd} is used as the ring generator $\omega$. The structures for $\Zomega$ are constructed as described above. Setters \fun{setAlphabet}{alphabet}, \fun{setInputAlphabet} and \fun{setBase}{base} are called. Messages saved to logfile during existence of an instance are printed (using \LaTeX) on standard output depending on \var{printLog} and \var{printLogLatex}. The level of messages for a development is set by \var{verbose}. 
\end{method}

Setters are following:
\begin{method}{setAlphabet}{A}
Check if \var{A} is subset of $\Zomega$. Set alphabet $\A$:=\var{A}
\end{method}

\begin{method}{setInputAlphabet}{B}
If \var{B} is empty, $\A+\A$ is used. Set the input alphabet $\B$:=\var{B}. Check if $\A\subsetneq\B\subset\A+\A$. 
\end{method}

\begin{method}{setBase}{base}
Set $\beta$:=\var{base}.
\end{method}

Getters are following:
\begin{method}{getRingGenerator}{}
Return ring generator $\omega$.
\end{method}


\begin{method}{getBase}{}
Return the base $\beta$.
\end{method}


\begin{method}{getBaseCC}{}
Return the base $\beta$ as a complex number.
\end{method}


\begin{method}{getAlphabet}{}
Return the alphabet $\A$.
\end{method}


\begin{method}{getInputAlphabet}{}
Return the input alphabet $\B$.
\end{method}


\begin{method}{getMinPolynomial}{}
Return the minimal polynomial $p$ of ring generator $\omega$.
\end{method}


\begin{method}{getMinPolynomialOfBase}{}
Return the minimal polynomial $p$ of base $\beta$.
\end{method}


\begin{method}{getWeightCoefSet}{}
Return weight coefficients set $\Q$. If it is not computed, return \var{None}.
\end{method}


\begin{method}{getWeightFunction}{}
Return weight function $q$. If it is not computed, return \var{None}.
\end{method}


\begin{method}{getName}{}
Return the name of the numeration system.
\end{method}


\begin{method}{getDictOfSetting}{}
Return dictionary attributes of numeration system.
\end{method}


-----------------------------EXTENDING WINDOW METHOD------------------------------------------------------------------------

\begin{method}{\_findWeightCoefSet}{ max\_iterations, method}

\end{method}


\begin{method}{addWeightCoefSetIncrement}{ increment}

\end{method}


\begin{method}{\_findWeightFunction}{ max\_input\_length,method}

\end{method}


\begin{method}{findWeightFunction}{ max\_iterations, max\_input\_length, method\_weightCoefSet=2, method\_weightFunSearch=4}

\end{method}


-----------------------------PARALLEL ADDITION AND CONVERSION---------------------------------------------------------------

\begin{method}{addParallel}{a,b}

\end{method}


\begin{method}{parallelConversion}{\_w}

\end{method}


\begin{method}{localConversion}{w}

\end{method}


\begin{method}{parallelConversion\_using\_localConversion}{w}

\end{method}


-----------------------------SANITY CHECK-----------------------------------------------------------------------------------

\begin{method}{sanityCheck\_addition}{ num\_digits}

\end{method}


\begin{method}{sanityCheck\_conversion}{ num\_digits}

\end{method}


-----------------------------RING CONVERSIONS, AUXILIARY RING FUNCTIONS-----------------------------------------------------

\begin{method}{list2BaseRing}{ \_digits}

\end{method}


\begin{method}{list2Ring}{ \_digits}

\end{method}


\begin{method}{ring2NumberField}{ num\_from\_ring}

\end{method}


\begin{method}{ring2CC}{ num\_from\_ring}

\end{method}


\begin{method}{getCoordinates}{ num}

\end{method}


\begin{method}{sumOfSets}{A,B}

\end{method}


\begin{method}{\_computeInverseBaseCompanionMatrix}{}

\end{method}


\begin{method}{divideByBase}{divided\_number}

\end{method}


-----------------------------PRINT FUNCTIONS--------------------------------------------------------------------------------

\begin{method}{addLog}{\_log, latex=False}

\end{method}


\begin{method}{printWeightFunction}{}

\end{method}


\begin{method}{printWeightFunctionInfo}{}

\end{method}


\begin{method}{printWeightCoefSet}{}

\end{method}


\begin{method}{printLatexInfo}{}

\end{method}


-----------------------------PLOT FUNCTIONS---------------------------------------------------------------------------------

\begin{method}{plot}{ nums\_from\_ring, labeled=True, color='red', size=20, fontsize=10}

\end{method}


\begin{method}{plotAlphabet}{}

\end{method}


\begin{method}{plotWeightCoefSet}{estimation=False}

\end{method}


\begin{method}{plotLattice}{}

\end{method}


\begin{method}{polygon\_shifted}{    def polygon\_shifted(points,shift=0, enlargement=1.2, color='green'}

\end{method}


\begin{method}{plotPhase1}{legend\_xshift=8,}

\end{method}


\begin{method}{polygon\_shifted}{    def polygon\_shifted(points,shift=0, enlargement=1, color='green'}

\end{method}


\begin{method}{legend}{    def legend(k,covered, new, alphabet}

\end{method}


\begin{method}{plotPhase2}{ digits,}

\end{method}


\begin{method}{polygon\_shifted2}{    def polygon\_shifted2(points,shift=0, enlargement=1, color='green'}

\end{method}


-----------------------------SAVE FUNCTIONS---------------------------------------------------------------------------------

\begin{method}{saveInfoToTexFile}{ filename, header=True}

\end{method}


\begin{method}{saveLog}{ filename}

\end{method}


\begin{method}{saveWeightFunctionToTexFile}{ filename}

\end{method}


\begin{method}{saveWeightFunctionToCsvFile}{ filename}

\end{method}


\begin{method}{saveLocalConversionToCsvFile}{ filename}

\end{method}


\begin{method}{saveUnsolvedInputsToCsv}{ filename}

\end{method}


\begin{method}{inputSettingToSageFile}{ filename}

\end{method}


\begin{method}{saveImages}{images, folder,name, img\_size=10}

\end{method}
