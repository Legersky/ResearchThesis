This class constructs neccessary structures for computation in $\Zomega$. It is \emph{PolynomialQuotientRing} obtained as a \emph{PolynomialRing} over integers factored by polynomial $p$. This is used for representation of elements of $\Zomega$ and arithmetics. We remark that it is independent on the choice of root of the  minimal polynomial $p$. But as we need also comparisons of numbers in $\Zomega$ in modulus, we specify $\omega$ by its approximate complex value and we form a factor ring of rational polynomials by using class \emph{NumberField}. This enables us to get absolute values of elements of $\Zomega$ which can be then compared.

Method \textbf{findWeightFunction()} links together both phases of the extending window method to find the weight function $q$. That is used in the methods for addition and digit set conversion to process them as local functions. There are also verification methods.

Moreover, many methods for printing, plotting and saving outputs are implemented.

The constructor of class \emph{AlgorithmForParallelAddition} is 

\begin{method}{\_\_init\_\_}{minPol\_str, embd, alphabet, base, name='NumerationSystem', inputAlphabet=' ',\\
 printLog=True, printLogLatex=False, verbose=0}
Take \var{minPol\_str} which is a string of symbolic expression in the variable $x$ of minimal polynomial~$p$. The closest root of  \var{minPol\_str} to \var{embd} is used as the ring generator $\omega$ (see more in documentation of \emph{NumberField} in SageMath \cite{sage}). The structures for $\Zomega$ are constructed as described above. Setters \fun{setAlphabet}{alphabet}, \fun{setInputAlphabet}{inputAlphabet} and \fun{setBase}{base} are called. The alphabet is checked according to Section \ref{sec:alphabet}. Messages saved to logfile during existence of an instance are printed (using \LaTeX) on standard output depending on \var{printLog} and \var{printLogLatex}. The level of messages for a development is set by \var{verbose}. 
\end{method}

Methods for the construction of an addition algorithm which is computable in parallel by the designed extending window method are the following:

\begin{method}{\_findWeightCoefSet}{ max\_iterations, method\_number}
Create an instance of \emph{WeightCoefficientsSetSearch(method\_number)} and call its method \fun{findWeightCoefficientsSet}{max\_iterations} to obtain a weight coefficients set $\Q$.
\end{method}

\begin{method}{\_findWeightFunction}{ max\_input\_length,method\_number}
Create an instance of \emph{WeightFunctionSearch(method\_number)} and call its methods \fun{check\_one\_letter\_inputs}{max\_input\_length} and \\ \fun{findWeightFunction}{max\_input\_length} to obtain a weight function $q$.
\end{method}


\begin{method}{findWeightFunction}{ max\_iterations, max\_input\_length, method\_weightCoefSet=2,\\ method\_weightFunSearch=4}
Return the weight function $q$ obtained by calling \fun{\_findWeightCoefSet}{max\_iterations, method\_weightCoefSet} and \fun{\_findWeightFunction}{max\_input\_length,\\ method\_weightFunSearch}.
\end{method}

The important function for the searching for possible weight coefficients is

\begin{method}{divideByBase}{divided\_number}
Using Theorem \ref{thm:divisibility}, check if \var{divided\_number} is divisible by the base $\beta$. If it is so, return the result of division, else return \var{None}.
\end{method}


Methods for the addition and the digit set conversion computable in parallel are following:

\begin{method}{addParallel}{a,b}
Sum up numbers represented by lists of digits \var{a} and \var{b} digitwise and convert the result by \fun{parallelConversion}{}. 
\end{method}


\begin{method}{parallelConversion}{\_w}
Return $(\beta,\A)$-representation of number represented by list \var{\_w} of digits in input alphabet $\B$. It is computed locally according to the equation \eqref{eq:conversionFormula} and using weight function $q$.
\end{method}


\begin{method}{localConversion}{w}
Return converted digit according to equation \eqref{eq:conversionFormula} for list of input digits \var{w}.
\end{method}


The correcteness of the implementation of the extending window for a given numeration system can be verified by
 
\begin{method}{sanityCheck\_conversion}{ num\_digits}
Check whether the values of all possible numbers of the length \var{num\_digits} with digits in the input alphabet $\B$ are the same as their $(\beta, \A)$-representation obtained by \fun{parallelConversion}{}.   
\end{method}


