Constructor:

\begin{method}{\_\_init\_\_}{minPol\_str, embd, alphabet, base, name='NumerationSystem', inputAlphabet='', printLog=True, printLogLatex=False, verbose=0}
Take \var{minPol\_str} which is symbolic expression in the variable $x$ of minimal polynomial $p$. The closest root of  \var{minPol\_str} to \var{embd} is used as the ring generator $\omega$. The structures for $\Zomega$ are constructed as described above. Setters \fun{setAlphabet}{alphabet}, \fun{setInputAlphabet}{A} and \fun{setBase}{base} are called. Messages saved to logfile during existence of an instance are printed (using \LaTeX) on standard output depending on \var{printLog} and \var{printLogLatex}. The level of messages for a development is set by \var{verbose}. 
\end{method}

Setters and getters:

\begin{method}{setAlphabet}{A}
Check if \var{A} is subset of $\Zomega$. Set alphabet $\A$:=\var{A}
\end{method}

\begin{method}{setInputAlphabet}{B}
If \var{B} is empty, $\A+\A$ is used. Set the input alphabet $\B$:=\var{B}. Check if $\A\subsetneq\B\subset\A+\A$. 
\end{method}


-----------------------------EXTENDING WINDOW METHOD------------------------------------------------------------------------

\begin{method}{\_findWeightCoefSet}{ max\_iterations, method\_number}
Create an instance of \emph{WeightCoefficientsSetSearch(method\_number)} and call its method \fun{findWeightCoefficientsSet}{max\_iterations} to obtain a weight coefficients set $\Q$.
\end{method}

\begin{method}{addWeightCoefSetIncrement}{ increment}
Save \var{increment} from extending intermediate weight coefficients set $\Q_{k}$ to $\Q_{k+1}$.
\end{method}

\begin{method}{\_findWeightFunction}{ max\_input\_length,method\_number}
Create an instance of \emph{WeightFunctionSearch(method\_number)} and call its methods \fun{check\_one\_letter\_inputs}{max\_input\_length} and \fun{findWeightFunction}{max\_input\_length} to obtain a weight function $q$.
\end{method}


\begin{method}{findWeightFunction}{ max\_iterations, max\_input\_length, method\_weightCoefSet=2, method\_weightFunSearch=4}
Return the weight function $q$ obtaind by calling \fun{\_findWeightCoefSet}{max\_iterations,method\_weightCoefSet} and \fun{\_findWeightFunction}{max\_input\_length, method\_weightFunSearch}.
\end{method}


-----------------------------PARALLEL ADDITION AND CONVERSION---------------------------------------------------------------

\begin{method}{addParallel}{a,b}
Sum up numbers represented by lists of digits \var{a} and \var{b} digitwise and convert the result by \fun{parallelConversion}{}. 
\end{method}


\begin{method}{parallelConversion}{\_w}
Return $(\beta,\A)$-representation of number represented by list \var{\_w} of digits in input alphabet $\B$. It is computed locally according to the equation \ref{eq:conversionFormula} and using weight function $q$.
\end{method}


\begin{method}{localConversion}{w}

\end{method}


-----------------------------SANITY CHECK-----------------------------------------------------------------------------------

\begin{method}{sanityCheck\_conversion}{ num\_digits}

\end{method}


-----------------------------RING CONVERSIONS, AUXILIARY RING FUNCTIONS-----------------------------------------------------

\begin{method}{\_computeInverseBaseCompanionMatrix}{}

\end{method}


\begin{method}{divideByBase}{divided\_number}

\end{method}

